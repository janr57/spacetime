% st-prologue.tex
%
% Copyright (C) 2020 José A. Navarro Ramón <josea.navarro1@gmail.com>

\chapter{Preface}
These class notes are taken from the twenty-four amazing lectures,
\emph{Central Lecture Course on Spacetime} that you can find in
\href{https://www.youtube.com/playlist?list=PLFeEvEPtX_0S6vxxiiNPrJbLu9aK1UVC_}{YouTube},
thanks to, both the WE-Heraeus International Winter School on Gravity and Light and professor
Frederic Schuller. They are written using Lua\LaTex\ .

The main goal is to understand the basic mathematics and physics concepts related with modern
spacetime so that I can later get a more in-depth knowledge of this subject.

I would not consider these notes finished anytime soon. Every time I read them I find errors,
ianccuracies and bad or, at least, not well written sentences. This is because English is not
my mother tongue.

The topic for this course is \emph{the structure of spacetime} and more precisely from a
physical point of view the structure of gravity theory which, if relativistic, is general
relativity and the theory of relativistic matter.

The Einstein equations of general relativity connect the matter contents of the universe with
gravity. Of course matter gravitates. The gravitational effect of matter is encoded in the
structure of spacetime as curvature. One side of the Einstein equations talks about matter and
the side talks about gravity.

But there is an underline notion that we need to understand before start talking about gravity
and matter. This notion is \emph{spacetime}. Everybody can have an intuitive notion of spacetime
as something that's made up of space and time and somehow forms a unique entity. But this notion
is not good enough to talk precisely about it. See figure~\ref{fig:prologue-spacetime}.
\begin{figure}[ht]
  \centering
  % Scale
  \def\scl{1}
  % SCALE FACTOR
  \pgfmathsetmacro{\SCALE}{.60}
  % LENGTHS
  % Horizontal distance
  \pgfmathsetmacro{\XLEN}{1*\SCALE}
  % Vertical distance
  \pgfmathsetmacro{\YLEN}{1*\SCALE}
  % Matter rectangle
  \pgfmathsetmacro{\XLENMATTER}{10*\SCALE}
  \pgfmathsetmacro{\YLENMATTER}{.9*\XLENMATTER*\SCALE}
  % Gravity rectangle
  \pgfmathsetmacro{\XLENGRAVITY}{\XLENMATTER}
  \pgfmathsetmacro{\YLENGRAVITY}{\YLENMATTER}
  % Spacetime rectangle
  \pgfmathsetmacro{\XLENSPACETIME}{\XLENMATTER + \XLEN + \XLENGRAVITY}
  \pgfmathsetmacro{\YLENSPACETIME}{\YLENMATTER}
  % COORDINATES
  % Spacetime
  \pgfmathsetmacro{\XSINI}{0}
  \pgfmathsetmacro{\YSINI}{0}
  \pgfmathsetmacro{\XSEND}{\XSINI + \XLENSPACETIME}
  \pgfmathsetmacro{\YSEND}{\YSINI + \YLENSPACETIME}
  % Matter
  \pgfmathsetmacro{\XMINI}{\XSINI}
  \pgfmathsetmacro{\YMINI}{\YSINI + \YLENSPACETIME + \YLEN}
  \pgfmathsetmacro{\XMEND}{\XMINI + \XLENMATTER}
  \pgfmathsetmacro{\YMEND}{\YMINI + \YLENMATTER}
  % Gravity
  \pgfmathsetmacro{\XGINI}{\XMEND + \XLEN}
  \pgfmathsetmacro{\YGINI}{\YMINI}
  \pgfmathsetmacro{\XGEND}{\XGINI + \XLENGRAVITY}
  \pgfmathsetmacro{\YGEND}{\YGINI + \YLENGRAVITY}
  % Box Text
  \pgfmathsetmacro{\XMTEXT}{\XMINI + .5*\XLENMATTER}
  \pgfmathsetmacro{\YMTEXT}{\YMINI + .5*\YLENMATTER}
  \pgfmathsetmacro{\XGTEXT}{\XGINI + .5*\XLENGRAVITY}
  \pgfmathsetmacro{\YGTEXT}{\YGINI + .5*\YLENGRAVITY}
  \pgfmathsetmacro{\XSTEXT}{\XSINI + .5*\XLENSPACETIME}
  \pgfmathsetmacro{\YSTEXT}{\YSINI + .5*\YLENSPACETIME}
  % Arrow ends
  \pgfmathsetmacro{\XARROWINI}{\XMTEXT}
  \pgfmathsetmacro{\YARROWINI}{\YMEND}
  \pgfmathsetmacro{\XARROWEND}{\XGTEXT}
  \pgfmathsetmacro{\YARROWEND}{\YGEND}

  % 
  \begin{tikzpicture}[%
    scale=\scl,
    baseline,
    every node/.style={black,font=\large},
    box/.style={fill=green!50, draw=green!55!black},
    arrow/.style={{Latex[bend]}-{Latex[bend]}, shorten >=4pt, shorten <=4pt, line width=2.5pt},
    background/.style={
      line width=\bgborderwidth,
      draw=\bgbordercolor,
      fill=\bgcolor,
    },
    ]
    % COORDINATES
    % Origin
    \coordinate (O) at (0,0);
    % Spacetime
    \coordinate (S_INI) at (\XSINI em, \YSINI em);
    \coordinate (S_END) at (\XSEND em, \YSEND em);
    \coordinate (S_TEXT) at (\XSTEXT em, \YSTEXT em);
    % Matter
    \coordinate (M_INI) at (\XMINI em, \YMINI em);
    \coordinate (M_END) at (\XMEND em, \YMEND em);
    \coordinate (M_TEXT) at (\XMTEXT em, \YMTEXT em);
    % Gravity
    \coordinate (G_INI) at (\XGINI em, \YGINI em);
    \coordinate (G_END) at (\XGEND em, \YGEND em);
    \coordinate (G_TEXT) at (\XGTEXT em, \YGTEXT em);
    % Arrow
    \coordinate (A_INI) at (\XARROWINI em, \YARROWINI em);
    \coordinate (A_END) at (\XARROWEND em, \YARROWEND em);
    %
    % DRAWING
    % Boxes
    \filldraw[box] (S_INI) rectangle (S_END);
    \filldraw[box] (M_INI) rectangle (M_END);
    \filldraw[box] (G_INI) rectangle (G_END);
    % Text
    \node at (M_TEXT) {Matter};
    \node at (G_TEXT) {Gravity};
    \node at (S_TEXT) {Spacetime};
    % Arrows
    \draw[arrow, black!60] (A_INI) to[bend left=80] node[above, name=E] {\large Einstein Eqns.} (A_END);
    %
    % Yellow background
    \begin{scope}[on background layer]
      \coordinate (SINI) at (S_INI);
      \coordinate (SEND) at (S_END);
      \coordinate (belowleft) at ($(SINI) - (2em, 2em)$);
      \coordinate (above) at ($(E) + (0em, 2em)$);
      \coordinate (right) at ($(SEND) + (2em, 0)$);
      \node [background, fit= (belowleft) (above) (right)] {};
    \end{scope}
  \end{tikzpicture}
  \caption{Einstein equations relate matter and gravity.
    Spacetime is needed to talk about these concepts.}
  \label{fig:prologue-spacetime}
\end{figure}


A notion that is good enough is the following physical key definition which underlies all modern
physics:
\begin{quote}
  ``Spacetime is a four-\emph{dimensional} \emph{topological} \emph{manifold} with a
  \emph{smooth atlas} carrying a \emph{torsion-free} \emph{connection} compatible with a
  \emph{Lorentzian metric} and a \emph{time orientation} satisfying the Einstein equations.''
\end{quote}

What does this mean?
It is the purpose of these lectures to clarify this definition.






%%% Local Variables:
%%% coding: utf-8
%%% mode: latex
%%% TeX-engine: luatex
%%% TeX-master: "../spacetime.tex"
%%% End:

