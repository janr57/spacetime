% manifolds.tex
%
% Copyright (C) 2020-2025 José A. Navarro Ramón <janr.devel@gmail.com>

\chapter{Topological manifolds}


\section{Introduction}
In the last chapter we introduced topological spaces. It's a fact of life that there are so many
that mathematicians cannot even classify them. If we take all the notions ever invented:
regular space, sequential space, first-countable space, second-countable space, t-space,
Hausdorff space, compact space, \dots, we will find many sets that are not homeomorphic with
them. Ok, but that is a mathematical problem.

A physical problem is that many sets that you give a topology, they resemble what we think of
spacetime so little that they are not deemed to be useful ---at the moment--- as models for
spacetime.

But there is a very special type of topological spaces $(M,\symcal{O}_M)$ in classical spacetime
physics, that can be charted analogously to how the surface of the Earth is charted in an atlas.
These special types of topological spaces are called \emph{topological manifolds}.

So a topological manifold is a topological space with one extra condition.


\section{Definition of topological manifold}
A topological space $(M,\symcal{O}_M)$ is called a d-dimensional topological manifold,
if for every point in the set there exists an open set $U$ containing the point
there exists a map $x$ from $U$ into a certain subset of $\symbb{R}^d$, such that
\[
  \forall p\in M
  \text{, }\exists U\in\symcal{O}_M
  \,|\, p\in U
  \,|\, \exists x: U\subseteq M \xrightarrow{\hspace{2.0em}} x(U)\subseteq \symbb{R}^d 
\]
\begin{enumerate}
\item such that $x$ is invertible ---there is a map $x^{-1}$ that goes from the image $x(U)$
  back to $U$---.
  \[
    \exists x^{-1} : x(U)\subseteq M \xrightarrow{\hspace{2,8em}} U \subseteq \symbb{R}^d
  \]
\item $x$ is continuous (we have to take into account the topologies)
   \vspace{-1.0ex}
 \begin{center}
 %\begin{figure}[ht]
   \def\scl{1}
   % \centering
   \begin{tikzpicture}[%
     scale=\scl,
     curled/.style={%
       green!50!black,
       -{Stealth},
       decorate,
       decoration={snake,post length=4.3pt,amplitude=1.3pt,segment length=3.5pt},
     },
     ]
     \node (x) at (0,0) {$x:$};
     \node[right = .6em of x] (U) at (0,0) {$U\subseteq M$};
     \node[right = -.4em of U] (arrow) {$\xrightarrow{\hspace{3em}}$};
     \node[right = -.4em of arrow] (xU) {$x(U)\subseteq\symbb{R}^d$};
     
     \node[below right = 1.5em and -1.5em of U] (OM) {\small $\symcal{O}_M$};
     \node[below right = 1.5em and -1.8em of xU] (ORd) {\small $\symcal{O}_{\text{std}}$};
     
     \draw[curled] ($(OM) + (-.2em,.6em)$) -- ++(up:2.0em);
     \draw[curled] ($(ORd) + (-0.34em,.6em)$) -- ++(up:2.0em);
   \end{tikzpicture}
 \end{center}
\item and the inverse $x^{-1}$ is also continuous (this is acomplished when we choose
  the subset topology $\symcal{O}|_U$ for U).
  
  We know that if $x$ is invertible and continuous, it doesn't mean that $x^{-1}$ is also
  continuous. But nevertheless, here we require it.
  
  % \end{figure}
\end{enumerate}

\subsubsection{Comments}
\begin{itemize}
\item The beginning of the definition means that, for every point in $M$, we can find an open
  set $U$ in $M$. And there exist a map $x$ wich takes \emph{every point in} $P$ to a certain
  subset of $\symbb{R}^d$.
\item Note that, as $U$ is open and all the conditions 1, 2 and 3 apply, we know that $x(U)$ must
  also be open.
\end{itemize}

\subsubsection{First example}
Let's take the surface $M$ of a doughnut ($M$ is a subset of $\symbb{R}^3$). Next, we choose any
arbitrary point $p$ on the surface. There must exist an open set $U$ which contains the point
($p\in U$). See figure~\ref{fig:mf-torus-pU}
\begin{figure}[ht]
  \def\scl{1}
  \pgfmathsetmacro{\HORIZONTAL}{2.5}
  \pgfmathsetmacro{\VERTICAL}{1.3}
  \centering
  \begin{tikzpicture}[%
    scale=\scl,
    torus/.style={fill=black!12},
    point/.style={fill=black,draw=black},
    background/.style={
      line width=\bgborderwidth,
      draw=\bgbordercolor,
      fill=\bgcolor,
    },
    ]
    \coordinate (O) at (0,0);
    \coordinate (M) at (1.6,.8);
    \coordinate (btop) at (0,\VERTICAL);
    \coordinate (bleft) at (-\HORIZONTAL,0);
    \coordinate (bright) at (\HORIZONTAL,0);
    \coordinate (bbottom) at (0,-\VERTICAL);
    % TORUS
    \pic[torus]{torus={1cm}{3.3mm}{67}};
    % \filldraw[point] (-.4,-.35) circle[radius=.5pt]
    % node[below right=-4pt and -1pt] {\scriptsize $p$};
    % M
    \node at (M) {\footnotesize $M\subset\symbb{R}^3$};
    % 
    % \filldraw[fill=red,draw=black] (btop) circle[radius=1pt];
    % \filldraw[fill=red,draw=black] (bbottom) circle[radius=1pt];
    % \filldraw[fill=green,draw=black] (bleft) circle[radius=1pt];
    % \filldraw[fill=green,draw=black] (bright) circle[radius=1pt];
    % \filldraw[fill opacity=0,draw=black] (O) circle[radius=2pt];    
    % YELLOW BACKGROUND
    \begin{scope}[on background layer]
      \node [background, fit= (bleft) (bright) (btop) (bbottom)] {};
    \end{scope}
  \end{tikzpicture}
  \hspace{2em}
  \begin{tikzpicture}[%
    scale=\scl,
    torus/.style={fill=black!12},
    open set/.style={dotted,fill=green!40,draw=black},
    point/.style={fill=black,draw=black},
    background/.style={
      line width=\bgborderwidth,
      draw=\bgbordercolor,
      fill=\bgcolor,
    },
    ]
    \coordinate (O) at (0,0);
    \coordinate (M) at (1.6,.8);
    \coordinate (btop) at (0,\VERTICAL);
    \coordinate (bleft) at (-\HORIZONTAL,0);
    \coordinate (bright) at (\HORIZONTAL,0);
    \coordinate (bbottom) at (0,-\VERTICAL);
    % TORUS
    \pic[torus]{torus={1cm}{3.3mm}{67}};
    \draw[open set,rotate=-8] (-.4,-.40) ellipse (3.8mm and 2mm)
    node[above left=2pt and 7pt,\greentext] {\scriptsize $U$};
    \filldraw[point] (-.4,-.35) circle[radius=.5pt]
    node[below right=-4pt and -1pt] {\scriptsize $p$};
    % M
    \node at (M) {\footnotesize $M\subset\symbb{R}^3$};
    % 
    % YELLOW BACKGROUND
    \begin{scope}[on background layer]
      \node [background, fit= (bleft) (bright) (bbottom) (btop)] {};
    \end{scope}
  \end{tikzpicture}
  \caption{At the left side, the surface $M$ of a torus is represented.
    At the right side we choose \emph{any} point $p$ on $M$. For every point on $M$,
    we find an open set $U$ which contains the point, $p\in U$.}
  \label{fig:mf-torus-pU}
\end{figure}
We can find an invertible map $x$, which takes \emph{every point} in $U$ into a subset of
$\symbb{R}^2$. Both, $x$ add $x^{-1}$ are continuous. Then, the topological space
$(M,\symcal{O}|_M)$ is a \emph{2-topological manifold}. The pair ($U$, $x$) is called a
\emph{chart}. See figure~\ref{fig:mf-torus-chart}.
\begin{figure}[ht]
  \def\scl{1}
  \pgfmathsetmacro{\BGUP}{1.3}
  \pgfmathsetmacro{\BGRIGHT}{2+.7}
  \pgfmathsetmacro{\BGDOWN}{.6}
  \pgfmathsetmacro{\BGLEFT}{.8}  
  \pgfmathsetmacro{\XAXISLENGTH}{4}
  \pgfmathsetmacro{\YAXISLENGTH}{3}
  \centering
  \begin{tikzpicture}[%
    scale=\scl,
    torus/.style={fill=black!12},
    open set/.style={dotted,fill=green!40,draw=black},
    point/.style={fill=black,draw=black},
    map/.style={-{Latex[round]},shorten >=2pt},
    background/.style={
      line width=\bgborderwidth,
      draw=\bgbordercolor,
      fill=\bgcolor,
    },
    ]
    % torus
    \coordinate (p) at (-.4,-.35);
    \coordinate (l) at (-.84,-.30);
    \coordinate (r) at (-.10,-.41);
    % M
    \coordinate (M) at (1.6,.8);
    % chart
    \coordinate (chart) at (-4,-.7);
    \coordinate (Ochart) at ($(chart)+(0,-3.3)$);
    \coordinate (pchart) at (-2,-2.5);
    \coordinate (lchart) at (-3.4,-2.5);
    \coordinate (rchart) at (-.67,-3.0);
    % 
    \coordinate (btop) at (0,\BGUP);
    \coordinate (bleft) at ($(Ochart)-(\BGLEFT,0)$);
    \coordinate (bright) at (\BGRIGHT,0);
    \coordinate (bbottom) at ($(Ochart)-(0,\BGDOWN)$);
    % 
    % TORUS
    \pic[torus]{torus={1cm}{3.3mm}{67}};
    \draw[open set,rotate=-8] (-.4,-.40) ellipse (3.8mm and 2mm);
    \filldraw[fill=black,draw=black] (-.4,-.35) circle[radius=.5pt]
    node[below right=-4pt and -1pt] {\scriptsize $p$};
    \draw[open set,rotate=-8] (-.4,-.40) ellipse (3.8mm and 2mm)
    node[above left=2pt and 7pt,\greentext] {\scriptsize $U$};
    % CHART
    \path (chart) pic[dotted,fill=green!40]{set01};
    \draw[->] (Ochart) to ++(right:\XAXISLENGTH);
    \draw[->] (Ochart) to ++(up:\YAXISLENGTH);
    % MAP LINES
    \draw[map] (p) to[bend right] (pchart);
    \draw[map] (l) to[bend right] node[above=4pt] {$x$} (lchart);
    \draw[map] (r) to[bend right] (rchart);
    % P CHART
    \filldraw[fill=black,draw=black] (pchart) circle[radius=.5pt]
    node[below right=-2pt and 0pt] {\scriptsize $x(p)$};
    % M
    \node at (M) {\footnotesize $M\subset\symbb{R}^3$};
    % X(U)
    \node[\greentext] at (.1,-2.0) {\footnotesize $x(U)\subseteq\symbb{R}^2$};
    
    % \filldraw[fill=red,draw=black] (l) circle[radius=.5pt];
    % \filldraw[fill=red,draw=black] (r) circle[radius=.5pt];
    % \filldraw[fill=red,draw=black] (lchart) circle[radius=.5pt];
    % \filldraw[fill=red,draw=black] (rchart) circle[radius=.5pt];
    % 
    % \filldraw[fill=red,draw=black] (btop) circle[radius=1pt];
    % \filldraw[fill=red,draw=black] (bbottom) circle[radius=1pt];
    % \filldraw[fill=green,draw=black] (bleft) circle[radius=1pt];
    % \filldraw[fill=green,draw=black] (bright) circle[radius=1pt];
    % \filldraw[fill opacity=0,draw=black] (O) circle[radius=2pt];
    % YELLOW BACKGROUND
    \begin{scope}[on background layer]
      \node [background, fit= (bleft) (bright) (btop) (bbottom)] {};
    \end{scope}
  \end{tikzpicture}
  \caption{We choose an arbitrary map $x$ from \emph{all the points in} $U$ on the surface of the
    torus into a subset of $\symbb{R}^2$. The map $x$ is continuous, invertible, and its inverse is
    also continuous, so $M$ is a \emph{2-topological manifold}. The pair $(U,x)$ is called a
    \emph{chart}.}
  \label{fig:mf-torus-chart}
\end{figure}

Note that $p$ is a point in the \emph{real world}. If $M$ was a sphere, representing the Earth,
$p$ would have been the foot of the Eiffel tower (you could go there and touch it). $U$ could
be the area of Paris, so $x(U)$ would be the image of the area of Paris in the chart.
The coordinates of $x(U)$ have no meaning in the real world. In practice, the constructors of
maps of the Earth write in the chart the longitude and latitude of the Eiffel tower instead its
cartesian coordinates. The set of all the charts of the Earth is called an
\emph{atlas}\footnotemark{}. 
\footnotetext{The publisher could have decided to build an atlas using another set of $U$s and
  another map $x$. By choosing another map he could have distorted the charts in a different
  way.}

It is only required that, for every point in the manifold, there exists a chart which contains
the point, so the whole manifold can be covered by charts.

\subsubsection{Second example}
Let's consider a wire $M$ that forms a loop. The wire lives in three dimensions. We take the
standard topology for $\symbb{R}^3$, $\symcal{O}_{\text{std}}$.
For $M$ we take the inherited subset topology restricted to the set $M$,
$\symcal{O}_{\text{std}}|_M$.

We claim that this is a one-dimensional topological manifold. So, for every point $p$ in $M$,
there exists an open set $U$ which contains the point. See figure~\ref{fig:mf-closed-line-3D}.
\begin{figure}[ht]
  \def\scl{.50}
  \pgfmathsetmacro{\POINTRADIUS}{1.4}
  \pgfmathsetmacro{\OPENENDRADIUS}{2.0}
  \pgfmathsetmacro{\POINTRADIUS}{1.4}
  \pgfmathsetmacro{\RAD}{1.0}
  \pgfmathsetmacro{\RX}{3.0}
  \pgfmathsetmacro{\LX}{-\RX}
  \pgfmathsetmacro{\ANG}{45}
  \pgfmathsetmacro{\BGTOP}{2.6}
  \pgfmathsetmacro{\BGBOTTOM}{-2.3}
  \pgfmathsetmacro{\BGRIGHT}{3.8}
  \pgfmathsetmacro{\BGLEFT}{-3.8}
  \centering
  \begin{tikzpicture}[%
    scale=\scl,
    point/.style={fill=black,draw=black},
    open end/.style={fill=\bgcolor,draw=black},
    open set/.style={green!80!black,line width=1.7pt},
    line/.style={line width=1.3pt,draw=black!50},
    white/.style={line width=7pt,draw=\bgcolor},
    background/.style={
      line width=\bgborderwidth,
      draw=\bgbordercolor,
      fill=\bgcolor,
    },
    ]
    \coordinate (O) at (0,0);
    \coordinate (L) at (\LX,0);
    \coordinate (LT) at (\LX+\RAD,\RAD);
    \coordinate (LB) at (\LX+\RAD,-\RAD);
    \coordinate (R) at (\RX,0);
    \coordinate (RT) at (\RX-\RAD,\RAD);
    \coordinate (RB) at (\RX-\RAD,-\RAD);
    \coordinate (lo) at (-2.5,-.87);
    \coordinate (ro) at (-1.5,-.95);
    \coordinate (Mtext) at (2.5,1.8);
    \coordinate (ptext) at (LB);
    \coordinate (Utext) at (LB);
    % 
    \coordinate (bgtop) at (0,\BGTOP);
    \coordinate (bgleft) at (\BGLEFT,0);
    \coordinate (bgright) at (\BGRIGHT,0);
    \coordinate (bgbottom) at (0,\BGBOTTOM);
    % Lines
    \draw[line] (R) to[out=90,in=0] (RT) to[out=180,in=\ANG] (O);
    \draw[line] (L) to[out=-90,in=180] (LB) to[out=0,in=180+\ANG] (O);
    % 
    \draw[white] (L) to[out=90,in=180] (LT) to[out=0,in=180-\ANG] (O);
    \draw[line] (L) to[out=90,in=180] (LT) to[out=0,in=180-\ANG] (O);
    % 
    \draw[white] (R) to[out=-90,in=0] (RB) to[out=180,in=360-\ANG] (O);
    \draw[line] (R) to[out=-90,in=0] (RB) to[out=180,in=360-\ANG] (O);
    % Open set
    % \draw[open set] (lo) to[out=334,in=180] (LB) to[out=0,in=192] (ro);
    % Left open end
    % \filldraw[open end] (lo) circle[radius=\OPENENDRADIUS pt];
    % Right open end
    % \filldraw[open end] (ro) circle[radius=\OPENENDRADIUS pt];
    % p
    % \filldraw[point] (LB) circle[radius=\POINTRADIUS pt];
    % Text
    \node at (Mtext) {\footnotesize $M\subset\symbb{R}^3$};
    % \node[below] at (ptext) {\scriptsize $p$};
    % \node[above left=0pt and -2pt,green!70!black] at (Utext) {\scriptsize $U$};
    % YELLOW BACKGROUND
    % \filldraw[fill=red,draw=black] (bgtop) circle[radius=1pt];
    % \filldraw[fill=red,draw=black] (bgbottom) circle[radius=1pt];
    % \filldraw[fill=green,draw=black] (bgleft) circle[radius=1pt];
    % \filldraw[fill=green,draw=black] (bgright) circle[radius=1pt];
    % \filldraw[fill opacity=0,draw=black] (O) circle[radius=2pt];
    \begin{scope}[on background layer]
      \node [background, fit= (bgleft) (bgright) (bgtop) (bgbottom)] {};
    \end{scope}
  \end{tikzpicture}
  \hspace{2em}
  \begin{tikzpicture}[%
    scale=\scl,
    point/.style={fill=black,draw=black},
    open end/.style={fill=\bgcolor,draw=black},
    open set/.style={green!80!black,line width=1.7pt},
    line/.style={line width=1.3pt,draw=black!50},
    white/.style={line width=7pt,draw=\bgcolor},
    background/.style={
      line width=\bgborderwidth,
      draw=\bgbordercolor,
      fill=\bgcolor,
    },
    ]
    \coordinate (O) at (0,0);
    \coordinate (L) at (\LX,0);
    \coordinate (LT) at (\LX+\RAD,\RAD);
    \coordinate (LB) at (\LX+\RAD,-\RAD);
    \coordinate (R) at (\RX,0);
    \coordinate (RT) at (\RX-\RAD,\RAD);
    \coordinate (RB) at (\RX-\RAD,-\RAD);
    \coordinate (lo) at (-2.5,-.87);
    \coordinate (ro) at (-1.5,-.95);
    \coordinate (Mtext) at (2.5,1.8);
    \coordinate (ptext) at (LB);
    \coordinate (Utext) at (ro);
    % 
    \coordinate (bgtop) at (0,\BGTOP);
    \coordinate (bgleft) at (\BGLEFT,0);
    \coordinate (bgright) at (\BGRIGHT,0);
    \coordinate (bgbottom) at (0,\BGBOTTOM);
    % Lines
    \draw[line] (R) to[out=90,in=0] (RT) to[out=180,in=\ANG] (O);
    \draw[line] (L) to[out=-90,in=180] (LB) to[out=0,in=180+\ANG] (O);
    % 
    \draw[white] (L) to[out=90,in=180] (LT) to[out=0,in=180-\ANG] (O);
    \draw[line] (L) to[out=90,in=180] (LT) to[out=0,in=180-\ANG] (O);
    % 
    \draw[white] (R) to[out=-90,in=0] (RB) to[out=180,in=360-\ANG] (O);
    \draw[line] (R) to[out=-90,in=0] (RB) to[out=180,in=360-\ANG] (O);
    % Open set
    \draw[open set] (lo) to[out=334,in=180] (LB) to[out=0,in=192] (ro);
    % Left open end
    \filldraw[open end] (lo) circle[radius=\OPENENDRADIUS pt];
    % Right open end
    \filldraw[open end] (ro) circle[radius=\OPENENDRADIUS pt];
    % p
    \filldraw[point] (LB) circle[radius=\POINTRADIUS pt];
    % Text
    \node at (Mtext) {\footnotesize $M\subset\symbb{R}^3$};
    \node[above] at (ptext) {\scriptsize $p$};
    \node[below right,\greentext] at (Utext) {\footnotesize $U\subset M$};
    % Map
    % YELLOW BACKGROUND
    % \filldraw[fill=red,draw=black] (bgtop) circle[radius=1pt];
    % \filldraw[fill=red,draw=black] (bgbottom) circle[radius=1pt];
    % \filldraw[fill=green,draw=black] (bgleft) circle[radius=1pt];
    % \filldraw[fill=green,draw=black] (bgright) circle[radius=1pt];
    % \filldraw[fill opacity=0,draw=black] (O) circle[radius=2pt];
    \begin{scope}[on background layer]
      \node [background, fit= (bgleft) (bgright) (bgtop) (bgbottom)] {};
    \end{scope}
  \end{tikzpicture}
  \caption{At the left side, a loop in three dimensions $M$ is represented.
    At the right we choose \emph{any} point $p$, on $M$. We can always find an arbitrary
    \emph{open subset} $U$ of $M$ which contains the point.}
  \label{fig:mf-closed-line-3D}
\end{figure}

And we can find a \emph{one-to-one} map $x$, from every point in $U$ into a subset of
$\symbb{R}$. This map is invertible, and both $x$ and $x^{-1}$ are continuous.
In the target $x(U)$ we choose the inherited standard topology in $\symbb{R}$ restricted to the
set $X(U)$, $\symcal{O}_{\text{std}}|_{x(U)}$.
See figure~\ref{fig:mf-closed-line-3D-chart}.
\begin{figure}[ht]
  \def\scl{.50}
  \pgfmathsetmacro{\POINTRADIUS}{1.4}
  \pgfmathsetmacro{\OPENENDRADIUS}{2.0}
  \pgfmathsetmacro{\POINTRADIUS}{1.4}
  \pgfmathsetmacro{\RAD}{1.0}
  \pgfmathsetmacro{\RX}{3.0}
  \pgfmathsetmacro{\LX}{-\RX}
  \pgfmathsetmacro{\ANG}{45}
  \pgfmathsetmacro{\BGTOP}{2.3}
  \pgfmathsetmacro{\BGBOTTOM}{-5.6}
  \pgfmathsetmacro{\BGRIGHT}{4.0}
  \pgfmathsetmacro{\BGLEFT}{-5.8}
  \centering
  \begin{tikzpicture}[%
    scale=\scl,
    point/.style={fill=black,draw=black},
    open end/.style={fill=\bgcolor,draw=black},
    open set/.style={green!80!black,line width=1.7pt},
    line/.style={line width=1.3pt,draw=black!50},
    white/.style={line width=7pt,draw=\bgcolor},
    map/.style={-{Latex[round]},shorten >=2pt,shorten <=2pt},
    background/.style={
      line width=\bgborderwidth,
      draw=\bgbordercolor,
      fill=\bgcolor,
    },
    ]
    % Coordinates
    \coordinate (O) at (0,0);
    \coordinate (L) at (\LX,0);
    \coordinate (LT) at (\LX+\RAD,\RAD);
    \coordinate (LB) at (\LX+\RAD,-\RAD);
    \coordinate (R) at (\RX,0);
    \coordinate (RT) at (\RX-\RAD,\RAD);
    \coordinate (RB) at (\RX-\RAD,-\RAD);
    \coordinate (p) at (LB);
    \coordinate (lo) at (-2.5,-.87);
    \coordinate (ro) at (-1.5,-.95);
    \coordinate (Mtext) at (2.5,1.8);
    \coordinate (ptext) at (LB);
    \coordinate (Utext) at (ro);
    \coordinate (xlo) at (-4.5,-4.5);
    \coordinate (xro) at (-2.0,-4.5);
    \coordinate (xp) at (-2.9,-4.5);
    % 
    \coordinate (bgtop) at (0,\BGTOP);
    \coordinate (bgleft) at (\BGLEFT,0);
    \coordinate (bgright) at (\BGRIGHT,0);
    \coordinate (bgbottom) at (0,\BGBOTTOM);
    % Lines
    \draw[line] (R) to[out=90,in=0] (RT) to[out=180,in=\ANG] (O);
    \draw[line] (L) to[out=-90,in=180] (LB) to[out=0,in=180+\ANG] (O);
    % 
    \draw[white] (L) to[out=90,in=180] (LT) to[out=0,in=180-\ANG] (O);
    \draw[line] (L) to[out=90,in=180] (LT) to[out=0,in=180-\ANG] (O);
    % 
    \draw[white] (R) to[out=-90,in=0] (RB) to[out=180,in=360-\ANG] (O);
    \draw[line] (R) to[out=-90,in=0] (RB) to[out=180,in=360-\ANG] (O);
    % Open set
    \draw[open set] (lo) to[out=334,in=180] (LB) to[out=0,in=192] (ro);
    % Left open end
    \filldraw[open end] (lo) circle[radius=\OPENENDRADIUS pt];
    % Right open end
    \filldraw[open end] (ro) circle[radius=\OPENENDRADIUS pt];
    % p
    \filldraw[point] (LB) circle[radius=\POINTRADIUS pt];
    % Text
    \node at (Mtext) {\footnotesize $M\subset\symbb{R}^3$};
    \node[above] at (ptext) {\scriptsize $p$};
    \node[below right=0pt and 0pt,\greentext] at (Utext) {\footnotesize $U\subset M$};
    % Chart
    \draw[open set] (xlo) -- (xro);
    % Right open end
    \filldraw[open end] (xro) circle[radius=\OPENENDRADIUS pt]
    node[right=9pt,\greentext] {\footnotesize $x(U)\subseteq\symbb{R}$};
    % Left open end
    \filldraw[open end] (xlo) circle[radius=\OPENENDRADIUS pt];
    % x(p)
    \filldraw[point] (xp) circle[radius=\POINTRADIUS pt]
    node[below] {\scriptsize $x(p)$};
    % Map
    \draw[map] (lo) to[bend right] (xlo);
    \draw[map] (ro) to[bend right] (xro);
    \draw[map] (p) to[bend right] (xp);
    % YELLOW BACKGROUND
    % 
    % \filldraw[fill=red,draw=black] (bgtop) circle[radius=1pt];
    % \filldraw[fill=red,draw=black] (bgbottom) circle[radius=1pt];
    % \filldraw[fill=green,draw=black] (bgleft) circle[radius=1pt];
    % \filldraw[fill=green,draw=black] (bgright) circle[radius=1pt];
    % \filldraw[fill opacity=0,draw=black] (O) circle[radius=2pt];
    \begin{scope}[on background layer]
      \node [background, fit= (bgleft) (bgright) (bgtop) (bgbottom)] {};
    \end{scope}
  \end{tikzpicture}
  \caption{We can always invent a \emph{one-to-one} map which takes every point in $U$ into some
    subset of $\symbb{R}$. This map is invertible, and with the said topologies, both, the map
    and its inverse are continuous.}
  \label{fig:mf-closed-line-3D-chart}
\end{figure}

\subsubsection{Third example}
Now we take a subset $M$ of $\symbb{R}^2$ as in figure~\ref{fig:mf-three-lines}.
\begin{figure}[ht]
  \def\scl{.55}
  \pgfmathsetmacro{\POINTRADIUS}{1.4}
  \pgfmathsetmacro{\OPENENDRADIUS}{2.0}
  \pgfmathsetmacro{\POINTRADIUS}{1.4}
  \pgfmathsetmacro{\RAD}{1.0}
  \pgfmathsetmacro{\RX}{3.0}
  \pgfmathsetmacro{\LX}{-3.0}
  \pgfmathsetmacro{\LY}{0.4}
  \pgfmathsetmacro{\PX}{-1.4}
  \pgfmathsetmacro{\PY}{0.3}
  \pgfmathsetmacro{\ANG}{45}
  \pgfmathsetmacro{\BGTOP}{2.6}
  \pgfmathsetmacro{\BGBOTTOM}{-2.3}
  \pgfmathsetmacro{\BGRIGHT}{3.8}
  \pgfmathsetmacro{\BGLEFT}{-3.8}
  \centering
  \begin{tikzpicture}[%
    scale=\scl,
    point/.style={fill=black,draw=black},
    open end/.style={fill=\bgcolor,draw=black},
    open set/.style={green!80!black,line width=1.7pt},
    line/.style={line width=1.3pt,draw=black!50},
    white/.style={line width=7pt,draw=\bgcolor},
    background/.style={
      line width=\bgborderwidth,
      draw=\bgbordercolor,
      fill=\bgcolor,
    },
    ]
    \coordinate (O) at (0,0);
    \coordinate (L) at (\LX,\LY);
    \coordinate (p) at (\PX,\PY);
    \coordinate (RT) at (\RX-\RAD,\RAD);
    \coordinate (RB) at (\RX-\RAD,-\RAD);
    
    \coordinate (lo) at (-2.0,0.48);
    \coordinate (ro) at (-0.8,0.08);
    \coordinate (Mtext) at (2.5,1.8);
    % \coordinate (ptext) at (LB);
    % \coordinate (Utext) at (LB);
    %      % 
    \coordinate (bgtop) at (0,\BGTOP);
    \coordinate (bgleft) at (\BGLEFT,0);
    \coordinate (bgright) at (\BGRIGHT,0);
    \coordinate (bgbottom) at (0,\BGBOTTOM);
    % Lines
    % \draw[line] (L) to[out=20,in=170] (O);
    \draw[line] (L) to[out=20,in=160] (p);
    \draw[line] (p) to[out=335,in=180] (O);
    \draw[line] (O) to[out=60,in=170] (RT);
    \draw[line] (O) to[out=350,in=170] (RB);
    % Ends
    \filldraw[open end] (L) circle[radius=\OPENENDRADIUS pt];
    \filldraw[open end] (RT) circle[radius=\OPENENDRADIUS pt];
    \filldraw[open end] (RB) circle[radius=\OPENENDRADIUS pt];
    \filldraw[point] (O) circle[radius=.8pt];
    % \filldraw[point] (p) circle[radius=\OPENENDRADIUS pt]
    % node[above] {\scriptsize $p$};
    %      % Text
    \node at (Mtext) {\footnotesize $M\subset\symbb{R}^2$};
    % YELLOW BACKGROUND
    %      % \filldraw[fill=red,draw=black] (bgtop) circle[radius=1pt];
    %      % \filldraw[fill=red,draw=black] (bgbottom) circle[radius=1pt];
    %      % \filldraw[fill=green,draw=black] (bgleft) circle[radius=1pt];
    %      % \filldraw[fill=green,draw=black] (bgright) circle[radius=1pt];
    %      % \filldraw[fill opacity=0,draw=black] (O) circle[radius=2pt];
    \begin{scope}[on background layer]
      \node [background, fit= (bgleft) (bgright) (bgtop) (bgbottom)] {};
    \end{scope}
  \end{tikzpicture}
  \caption{A subset of $\symbb{R}^2$.}
  \label{fig:mf-three-lines}
\end{figure}

We consider the standard topology on $\symbb{R}^2$ and the inherited topology restricted to $M$
for the subset $\symcal{O}_{\text{std}}|_M$.
% \begin{center}
%   \def\scl{1}
%   \centering
%   \begin{tikzpicture}[%
%     scale=\scl,
%     curled/.style={%
%     green!50!black,
%     -{Stealth},
%     decorate,
%     decoration={snake,post length=4.3pt,amplitude=1.3pt,segment length=3.5pt},
%   },
%     ]
%     \node[baseline] (M) at (0,0) {$M$};
%     \node[right = -.4em of M,baseline] (subset) {$\subset$};
%     \node[right = -.4em of subset,baseline] (R2) {$\symbb{R}^2$};
%     \node[below = 1.5em of M] (bM) {\small $\symcal{O}_{\text{std}}|_M$};
%     \node[below = 1.5em of R2] (bR2) {\small $\symcal{O}_{\text{std}}$};
%     \draw[curled] (bM) -- (M);
%     \draw[curled] (bR2) -- (R2);
%   \end{tikzpicture}
% \end{center}
We claim that $(M,\symcal{O}_{\text{std}}|M)$ is a topological space but it fails to be
a topological manifold.

Of course, for every point on $M$, we could find an open set $U$ which contains it.
See figure~\ref{fig:mf-three-lines-pU}.
\begin{figure}[ht]
  \def\scl{.55}
  \pgfmathsetmacro{\OPENENDRADIUS}{2.0}
  \pgfmathsetmacro{\POINTRADIUS}{1.4}
  \pgfmathsetmacro{\RAD}{1.0}
  \pgfmathsetmacro{\RX}{3.0}
  \pgfmathsetmacro{\LX}{-3.0}
  \pgfmathsetmacro{\LY}{0.4}
  \pgfmathsetmacro{\BGTOP}{2.6}
  \pgfmathsetmacro{\BGBOTTOM}{-2.3}
  \pgfmathsetmacro{\BGRIGHT}{3.8}
  \pgfmathsetmacro{\BGLEFT}{-3.8}
  \centering
  \begin{tikzpicture}[%
    scale=\scl,
    point/.style={fill=black,draw=black},
    open end/.style={fill=\bgcolor,draw=black},
    open set/.style={green!80!black,line width=1.7pt},
    line/.style={line width=1.3pt,draw=black!50},
    white/.style={line width=7pt,draw=\bgcolor},
    background/.style={
      line width=\bgborderwidth,
      draw=\bgbordercolor,
      fill=\bgcolor,
    },
    ]
    \pgfmathsetmacro{\PX}{-1.4}
    \pgfmathsetmacro{\PY}{0.3}
    % Coordinates
    \coordinate (O) at (0,0);
    \coordinate (L) at (\LX,\LY);
    \coordinate (p) at (\PX,\PY);
    \coordinate (RT) at (\RX-\RAD,\RAD);
    \coordinate (RB) at (\RX-\RAD,-\RAD);
    \coordinate (lo) at (-2.0,0.48);
    \coordinate (ro) at (-0.8,0.08);
    % 
    \coordinate (Mtext) at (2.5,1.8);
    % 
    \coordinate (bgtop) at (0,\BGTOP);
    \coordinate (bgleft) at (\BGLEFT,0);
    \coordinate (bgright) at (\BGRIGHT,0);
    \coordinate (bgbottom) at (0,\BGBOTTOM);
    % Lines
    \draw[line] (L) to[out=20,in=160] (p);
    \draw[line] (p) to[out=335,in=180] (O);
    \draw[line] (O) to[out=60,in=170] (RT);
    \draw[line] (O) to[out=350,in=170] (RB);
    % Open set
    \filldraw[open set] (lo) to[out=348,in=160] (p);
    \filldraw[open set] (p) to[out=335,in=170] (ro);
    \node[below left=4pt and 0pt,\greentext] at (p) {\footnotesize $U\subset M$};
    % Ends
    \filldraw[open end] (L) circle[radius=\OPENENDRADIUS pt];
    \filldraw[open end] (RT) circle[radius=\OPENENDRADIUS pt];
    \filldraw[open end] (RB) circle[radius=\OPENENDRADIUS pt];
    \filldraw[point] (O) circle[radius=.8pt];
    \filldraw[point] (p) circle[radius=\OPENENDRADIUS pt]
    node[above] {\scriptsize $p$};
    \filldraw[open end] (lo) circle[radius=\OPENENDRADIUS pt];
    \filldraw[open end] (ro) circle[radius=\OPENENDRADIUS pt];
    % Text
    \node at (Mtext) {\footnotesize $M\subset\symbb{R}^2$};
    % YELLOW BACKGROUND
    % \filldraw[fill=red,draw=black] (bgtop) circle[radius=1pt];
    % \filldraw[fill=red,draw=black] (bgbottom) circle[radius=1pt];
    % \filldraw[fill=green,draw=black] (bgleft) circle[radius=1pt];
    % \filldraw[fill=green,draw=black] (bgright) circle[radius=1pt];
    % \filldraw[fill opacity=0,draw=black] (O) circle[radius=2pt];
    \begin{scope}[on background layer]
      \node [background, fit= (bgleft) (bgright) (bgtop) (bgbottom)] {};
    \end{scope}
  \end{tikzpicture}
  \hspace{2em}
  \begin{tikzpicture}[%
    scale=\scl,
    point/.style={fill=black,draw=black},
    open end/.style={fill=\bgcolor,draw=black},
    open set/.style={green!80!black,line width=1.7pt},
    line/.style={line width=1.3pt,draw=black!50},
    white/.style={line width=7pt,draw=\bgcolor},
    background/.style={
      line width=\bgborderwidth,
      draw=\bgbordercolor,
      fill=\bgcolor,
    },
    ]
    \pgfmathsetmacro{\MX}{-1.4}
    \pgfmathsetmacro{\MY}{0.3}
    % \pgfmathsetmacro{\ANG}{45}
    
    \coordinate (O) at (0,0);
    \coordinate (L) at (\LX,\LY);
    \coordinate (m) at (\MX,\MY);
    \coordinate (p) at (O);
    \coordinate (RT) at (\RX-\RAD,\RAD);
    \coordinate (RB) at (\RX-\RAD,-\RAD);
    \coordinate (lo) at (-0.8,0.08);
    \coordinate (rto) at (0.5,0.61);
    \coordinate (rbo) at (0.6,-0.225);
    
    \coordinate (Mtext) at (2.5,1.8);
    % \coordinate (ptext) at (LB);
    % \coordinate (Utext) at (LB);
    % 
    \coordinate (bgtop) at (0,\BGTOP);
    \coordinate (bgleft) at (\BGLEFT,0);
    \coordinate (bgright) at (\BGRIGHT,0);
    \coordinate (bgbottom) at (0,\BGBOTTOM);
    % Lines
    % \draw[line] (L) to[out=20,in=170] (O);
    \draw[line] (L) to[out=20,in=160] (m);
    \draw[line] (m) to[out=335,in=180] (O);
    \draw[line] (O) to[out=60,in=170] (RT);
    \draw[line] (O) to[out=350,in=170] (RB);
    % 
    \filldraw[open set] (lo) to[out=345,in=180] (p);
    \filldraw[open set] (p) to[out=60,in=218] (rto);
    \filldraw[open set] (p) to[out=350,in=145] (rbo);
    \node[below left=4pt and 0pt,\greentext] at (p) {\footnotesize $U\subseteq M$};
    % Ends
    \filldraw[open end] (L) circle[radius=\OPENENDRADIUS pt];
    \filldraw[open end] (RT) circle[radius=\OPENENDRADIUS pt];
    \filldraw[open end] (RB) circle[radius=\OPENENDRADIUS pt];
    \filldraw[point] (O) circle[radius=.8pt];
    \filldraw[point] (p) circle[radius=\OPENENDRADIUS pt]
    node[above left=-0.8pt and -3pt] {\scriptsize $p$};
    \filldraw[open end] (lo) circle[radius=\OPENENDRADIUS pt];
    \filldraw[open end] (rto) circle[radius=\OPENENDRADIUS pt];
    \filldraw[open end] (rbo) circle[radius=\OPENENDRADIUS pt];
    \node at (Mtext) {\footnotesize $M\subset\symbb{R}^2$};
    % Map
    % YELLOW BACKGROUND
    % \filldraw[fill=red,draw=black] (bgtop) circle[radius=1pt];
    % \filldraw[fill=red,draw=black] (bgbottom) circle[radius=1pt];
    % \filldraw[fill=green,draw=black] (bgleft) circle[radius=1pt];
    % \filldraw[fill=green,draw=black] (bgright) circle[radius=1pt];
    % \filldraw[fill opacity=0,draw=black] (O) circle[radius=2pt];
    \begin{scope}[on background layer]
      \node [background, fit= (bgleft) (bgright) (bgtop) (bgbottom)] {};
    \end{scope}
  \end{tikzpicture}
  \caption{For every point in $M$ we cand find an open set which contains it.}
  \label{fig:mf-three-lines-pU}
\end{figure}

For most points $p$ in $M$, like the one represented at the left side of
figure~\ref{fig:mf-three-lines-pU}, we could always find an invertible map $x$, from every point
in the open set $U$ into some subset of $\symbb{R}$, so that both, $x$ and $x^{-1}$ are
continuous under the considered topologies. See figure~\ref{fig:mf-three-lines-invertible-map}.
\begin{figure}[ht]
  \def\scl{.50}
  \pgfmathsetmacro{\POINTRADIUS}{1.4}
  \pgfmathsetmacro{\OPENENDRADIUS}{2.0}
  \pgfmathsetmacro{\POINTRADIUS}{1.4}
  \pgfmathsetmacro{\RAD}{1.0}
  \pgfmathsetmacro{\RX}{3.0}
  \pgfmathsetmacro{\LX}{-3.0}
  \pgfmathsetmacro{\LY}{0.4}
  \pgfmathsetmacro{\PX}{-1.4}
  \pgfmathsetmacro{\PY}{0.3}
  % \pgfmathsetmacro{\ANG}{45}
  \pgfmathsetmacro{\BGTOP}{2.6}
  \pgfmathsetmacro{\BGBOTTOM}{-5.6}
  \pgfmathsetmacro{\BGRIGHT}{4.0}
  \pgfmathsetmacro{\BGLEFT}{-5.8}
  \centering
  \begin{tikzpicture}[%
    scale=\scl,
    point/.style={fill=black,draw=black},
    open end/.style={fill=\bgcolor,draw=black},
    open set/.style={green!80!black,line width=1.7pt},
    line/.style={line width=1.3pt,draw=black!50},
    white/.style={line width=7pt,draw=\bgcolor},
    map/.style={-{Latex[round]},shorten >=2pt,shorten <=2pt},
    background/.style={
      line width=\bgborderwidth,
      draw=\bgbordercolor,
      fill=\bgcolor,
    },
    ]
    \coordinate (O) at (0,0);
    \coordinate (L) at (\LX,\LY);
    \coordinate (p) at (\PX,\PY);
    \coordinate (RT) at (\RX-\RAD,\RAD);
    \coordinate (RB) at (\RX-\RAD,-\RAD);
    \coordinate (lo) at (-2.0,0.48);
    \coordinate (ro) at (-0.8,0.08);
    \coordinate (xlo) at (-4.5,-4.5);
    \coordinate (xro) at (-2.0,-4.5);
    
    \coordinate (Mtext) at (2.5,1.8);
    % \coordinate (ptext) at (LB);
    % \coordinate (Utext) at (LB);
    % 
    \coordinate (bgtop) at (0,\BGTOP);
    \coordinate (bgleft) at (\BGLEFT,0);
    \coordinate (bgright) at (\BGRIGHT,0);
    \coordinate (bgbottom) at (0,\BGBOTTOM);
    
    % Lines
    % \draw[line] (L) to[out=20,in=170] (O);
    \draw[line] (L) to[out=20,in=160] (p);
    \draw[line] (p) to[out=335,in=180] (O);
    \draw[line] (O) to[out=60,in=170] (RT);
    \draw[line] (O) to[out=350,in=170] (RB);
    
    \filldraw[open set] (lo) to[out=348,in=160] (p);
    \filldraw[open set] (p) to[out=335,in=170] (ro);
    \node[above left=8pt and 0pt,\greentext] at (p) {\footnotesize $U\subset M$};
    % \filldraw[open set] (p) to[out=335,in=180] (ro);
    
    % Ends
    \filldraw[open end] (L) circle[radius=\OPENENDRADIUS pt];
    \filldraw[open end] (RT) circle[radius=\OPENENDRADIUS pt];
    \filldraw[open end] (RB) circle[radius=\OPENENDRADIUS pt];
    \filldraw[point] (O) circle[radius=.8pt];
    \filldraw[point] (p) circle[radius=\OPENENDRADIUS pt]
    node[above] {\scriptsize $p$};
    \filldraw[open end] (lo) circle[radius=\OPENENDRADIUS pt];
    \filldraw[open end] (ro) circle[radius=\OPENENDRADIUS pt];
    
    % Chart
    \draw[open set] (xlo) -- (xro);
    % Right open end
    \filldraw[open end] (xro) circle[radius=\OPENENDRADIUS pt]
    node[right=9pt,\greentext] {\footnotesize $x(U)\subseteq\symbb{R}$};
    % Left open end
    \filldraw[open end] (xlo) circle[radius=\OPENENDRADIUS pt];
    % x(p)
    \filldraw[point] (xp) circle[radius=\POINTRADIUS pt]
    node[below] {\scriptsize $x(p)$};
    % Map
    \draw[map] (lo) to[bend right] (xlo);
    \draw[map] (ro) to[bend right] (xro);
    \draw[map] (p) to[bend right] (xp);
    % Text
    \node at (Mtext) {\footnotesize $M\subset\symbb{R}^2$};
    % YELLOW BACKGROUND
    % 
    % \filldraw[fill=red,draw=black] (bgtop) circle[radius=1pt];
    % \filldraw[fill=red,draw=black] (bgbottom) circle[radius=1pt];
    % \filldraw[fill=green,draw=black] (bgleft) circle[radius=1pt];
    % \filldraw[fill=green,draw=black] (bgright) circle[radius=1pt];
    % \filldraw[fill opacity=0,draw=black] (O) circle[radius=2pt];
    \begin{scope}[on background layer]
      \node [background, fit= (bgleft) (bgright) (bgtop) (bgbottom)] {};
    \end{scope}
  \end{tikzpicture}
  \caption{For every point in $M$, \emph{except one}, we could find an open set $U$ containing the
    point, and invent a map $x$ that takes every point in $U$ to the real line, such that this
    map is invertible and continuous, and its inverse is also continuous.}
  \label{fig:mf-three-lines-invertible-map}
\end{figure}

But, when we choose the intersection point $p$ as in the right side of
figure~\ref{fig:mf-three-lines-pU}, we could not invent such a map:

\begin{itemize}
\item
  If we target $x(U)$ was a single open set in the real line, as at the left side of
  figure~\ref{fig:mf-three-lines-wrong-map}, such a map would not be \emph{one-to-one},
  because there were some points in $x(U)$ that would be the image of various points in $U$.
  So the map would not be invertible.
\item
  But, if the target $x(U)$ was a collection of open sets in $\symbb{R}$, then the map would
  be continuous and invertible, but the inverse would not be continuous, as there were some
  open sets in $U$ (involving parts of both, the green and the red subsets, whose preimages are not
  in $\symcal{O}_{\text{std}}|_{x(U)}$. See the right side of
  figure~\ref{fig:mf-three-lines-wrong-map}.
\end{itemize}

\begin{figure}[ht]
  \def\scl{.50}
  \pgfmathsetmacro{\POINTRADIUS}{1.4}
  \pgfmathsetmacro{\OPENENDRADIUS}{2.0}
  \pgfmathsetmacro{\POINTRADIUS}{1.4}
  \pgfmathsetmacro{\RAD}{1.0}
  \pgfmathsetmacro{\RX}{3.0}
  \pgfmathsetmacro{\LX}{-3.0}
  \pgfmathsetmacro{\LY}{0.4}
  \pgfmathsetmacro{\MX}{-1.4}
  \pgfmathsetmacro{\MY}{0.3}
  \pgfmathsetmacro{\ANG}{45}
  \pgfmathsetmacro{\BGTOP}{2.6}
  \pgfmathsetmacro{\BGBOTTOM}{-5.6}
  \pgfmathsetmacro{\BGRIGHT}{4.0}
  \pgfmathsetmacro{\BGLEFT}{-5.8}
  \centering
  \begin{tikzpicture}[%
    scale=\scl,
    point/.style={fill=black,draw=black},
    open end/.style={fill=\bgcolor,draw=black},
    open set/.style={green!80!black,line width=1.7pt},
    line/.style={line width=1.3pt,draw=black!50},
    white/.style={line width=7pt,draw=\bgcolor},
    map/.style={-{Latex[round]},shorten >=2pt,shorten <=1.6pt},
    background/.style={
      line width=\bgborderwidth,
      draw=\bgbordercolor,
      fill=\bgcolor,
    },
    ]
    \coordinate (O) at (0,0);
    \coordinate (L) at (\LX,\LY);
    \coordinate (m) at (\MX,\MY);
    \coordinate (p) at (O);
    \coordinate (RT) at (\RX-\RAD,\RAD);
    \coordinate (RB) at (\RX-\RAD,-\RAD);
    \coordinate (lo) at (-0.8,0.08);
    \coordinate (rto) at (0.5,0.61);
    \coordinate (rtm) at (0.2,0.3);
    \coordinate (rbo) at (0.6,-0.225);
    \coordinate (rbm) at (0.4,-0.111);
    \coordinate (xp) at (-3.4,-4.5);
    \coordinate (xm) at (-2.5,-4.5);
    
    \coordinate (xlo) at (-4.5,-4.5);
    \coordinate (xro) at (-2.0,-4.5);
    
    \coordinate (Mtext) at (2.5,1.8);
    % 
    \coordinate (bgtop) at (0,\BGTOP);
    \coordinate (bgleft) at (\BGLEFT,0);
    \coordinate (bgright) at (\BGRIGHT,0);
    \coordinate (bgbottom) at (0,\BGBOTTOM);
    
    % Lines
    \draw[line] (L) to[out=20,in=160] (m);
    \draw[line] (m) to[out=335,in=180] (O);
    \draw[line] (O) to[out=60,in=170] (RT);
    \draw[line] (O) to[out=350,in=170] (RB);
    
    \filldraw[open set] (lo) to[out=345,in=180] (p);
    \filldraw[open set,red!85!black] (p) to[out=60,in=218] (rto);
    \filldraw[open set] (p) to[out=350,in=145] (rbo);
    
    % \filldraw[fill=blue] (rtm) circle[radius=.5pt];
    % \filldraw (rbm) circle[radius=.5pt];
    
    % Ends
    \filldraw[open end] (L) circle[radius=\OPENENDRADIUS pt];
    \filldraw[open end] (RT) circle[radius=\OPENENDRADIUS pt];
    \filldraw[open end] (RB) circle[radius=\OPENENDRADIUS pt];
    \filldraw[point] (O) circle[radius=.8pt];
    \filldraw[point] (p) circle[radius=\OPENENDRADIUS pt];
    % node[above] {\scriptsize $p$};
    \filldraw[open end] (lo) circle[radius=\OPENENDRADIUS pt];
    \filldraw[open end] (rto) circle[radius=\OPENENDRADIUS pt];
    \filldraw[open end] (rbo) circle[radius=\OPENENDRADIUS pt];
    
    % Chart
    \draw[open set,line width=2pt] (xlo) -- (xro);
    % Right open end
    \filldraw[open end] (xro) circle[radius=\OPENENDRADIUS pt]
    node[right=9pt,\greentext] {\footnotesize $x(U)\subseteq\symbb{R}$};
    % Left open end
    \filldraw[open end] (xlo) circle[radius=\OPENENDRADIUS pt];
    % x(p)
    \filldraw[point] (xp) circle[radius=\POINTRADIUS pt]
    node[below] {\scriptsize $x(p)$};
    
    % Map
    \draw[map,black!20] (lo) to[bend right] (xlo);
    \draw[map,black!20] (rbo) to[bend right] (xro);
    % \draw[map] (p) to[bend right] (xp);
    \draw[map,red!70!black] (rtm) to[bend right] (xm);
    \draw[map,shorten <=0pt,green!50!black] (rbm) to[bend right] (xm);
    % Text
    \node at (Mtext) {\footnotesize $M\subset\symbb{R}^2$};
    
    \filldraw (rtm) circle[radius=.5pt];
    \filldraw (rbm) circle[radius=.5pt];
    % YELLOW BACKGROUND
    % 
    % \filldraw[fill=red,draw=black] (bgtop) circle[radius=1pt];
    % \filldraw[fill=red,draw=black] (bgbottom) circle[radius=1pt];
    % \filldraw[fill=green,draw=black] (bgleft) circle[radius=1pt];
    % \filldraw[fill=green,draw=black] (bgright) circle[radius=1pt];
    % \filldraw[fill opacity=0,draw=black] (O) circle[radius=2pt];
    \begin{scope}[on background layer]
      \node [background, fit= (bgleft) (bgright) (bgtop) (bgbottom)] {};
    \end{scope}
  \end{tikzpicture}
  \hspace{2em}
  \begin{tikzpicture}[%
    scale=\scl,
    point/.style={fill=black,draw=black},
    open end/.style={fill=\bgcolor,draw=black},
    open set/.style={green!80!black,line width=1.7pt},
    line/.style={line width=1.3pt,draw=black!50},
    white/.style={line width=7pt,draw=\bgcolor},
    map/.style={-{Latex[round]},shorten >=2pt,shorten <=1.6pt},
    background/.style={
      line width=\bgborderwidth,
      draw=\bgbordercolor,
      fill=\bgcolor,
    },
    ]
    \coordinate (O) at (0,0);
    \coordinate (L) at (\LX,\LY);
    \coordinate (m) at (\MX,\MY);
    \coordinate (p) at (O);
    \coordinate (RT) at (\RX-\RAD,\RAD);
    \coordinate (RB) at (\RX-\RAD,-\RAD);
    \coordinate (lo) at (-0.8,0.08);
    \coordinate (rto) at (0.5,0.61);
    \coordinate (rtm) at (0.2,0.3);
    \coordinate (rbo) at (0.6,-0.225);
    \coordinate (rbm) at (0.4,-0.111);
    \coordinate (xp) at (-3.4,-4.5);
    \coordinate (xm) at (-2.5,-4.5);
    
    \coordinate (xlo) at (-4.5,-4.5);
    \coordinate (xro) at (-2.0,-4.5);
    \coordinate (xlo2) at (0.0,-4.5);
    \coordinate (xro2) at (2.0,-4.5);
    
    \coordinate (Mtext) at (2.5,1.8);
    % 
    \coordinate (bgtop) at (0,\BGTOP);
    \coordinate (bgleft) at (\BGLEFT,0);
    \coordinate (bgright) at (\BGRIGHT,0);
    \coordinate (bgbottom) at (0,\BGBOTTOM);
    
    % Lines
    \draw[line] (L) to[out=20,in=160] (m);
    \draw[line] (m) to[out=335,in=180] (O);
    \draw[line] (O) to[out=60,in=170] (RT);
    \draw[line] (O) to[out=350,in=170] (RB);
    \filldraw[open set] (lo) to[out=345,in=180] (p);
    \filldraw[open set,red!85!black] (p) to[out=60,in=218] (rto);
    \filldraw[open set] (p) to[out=350,in=145] (rbo);
    
    % Ends of M
    \filldraw[open end] (L) circle[radius=\OPENENDRADIUS pt];
    \filldraw[open end] (RT) circle[radius=\OPENENDRADIUS pt];
    \filldraw[open end] (RB) circle[radius=\OPENENDRADIUS pt];
    \filldraw[point] (O) circle[radius=.8pt];
    
    % Point
    \filldraw[point,fill=black,draw=red] (p) circle[radius=2.7pt];
    
    % node[above] {\scriptsize $p$};
    % Open ends of U
    \filldraw[open end] (lo) circle[radius=\OPENENDRADIUS pt];
    \filldraw[open end] (rto) circle[radius=\OPENENDRADIUS pt];
    \filldraw[open end] (rbo) circle[radius=\OPENENDRADIUS pt];
    
    % Target 1
    \draw[open set,line width=2pt] (xlo) -- (xro);
    % Right open end
    \filldraw[open end] (xro) circle[radius=\OPENENDRADIUS pt];
    % node[right=9pt,\greentext] {\footnotesize $x(U)\subseteq\symbb{R}$};
    % Left open end
    \filldraw[open end] (xlo) circle[radius=\OPENENDRADIUS pt];
    % x(p)
    \filldraw[point] (xp) circle[radius=\POINTRADIUS pt]
    node[below] {\scriptsize $x(p)$};
    
    % Target 2
    \draw[open set,line width=2pt,red!85!black] (xlo2) -- (xro2);
    % Right open end
    \filldraw[open end] (xro2) circle[radius=\OPENENDRADIUS pt];
    % Left open end
    \filldraw[open end] (xlo2) circle[radius=\OPENENDRADIUS pt];
    
    % Map
    % Target 1
    \draw[map,black!20] (lo) to[bend right] (xlo);
    \draw[map,black!20] (rbo) to[bend right] (xro);
    % Target 2
    % draw[map,black!20] (p) to[bend left] (xlo2);
    % draw[map,black!20] (rto) to[bend left] (xro2);
    \draw[map,black!20] (p) to[out=20,in=60] (xlo2);
    \draw[map,black!20] (rto) to[out=0,in=60] (xro2);
    
    \draw[map] (p) to[bend right] (xp);
    % \draw[map,red!70!black] (rtm) to[bend right] (xm);
    % \draw[map,shorten <=0pt,green!50!black] (rbm) to[bend right] (xm);
    % Text
    \node at (Mtext) {\footnotesize $M\subset\symbb{R}^2$};
    
    % \filldraw (rtm) circle[radius=.5pt];
    % \filldraw (rbm) circle[radius=.5pt];
    % YELLOW BACKGROUND
    % 
    % \filldraw[fill=red,draw=black] (bgtop) circle[radius=1pt];
    % \filldraw[fill=red,draw=black] (bgbottom) circle[radius=1pt];
    % \filldraw[fill=green,draw=black] (bgleft) circle[radius=1pt];
    % \filldraw[fill=green,draw=black] (bgright) circle[radius=1pt];
    % \filldraw[fill opacity=0,draw=black] (O) circle[radius=2pt];
    \begin{scope}[on background layer]
      \node [background, fit= (bgleft) (bgright) (bgtop) (bgbottom)] {};
    \end{scope}
  \end{tikzpicture}    
  \caption{When we choose the point in the intersection of the lines, we cannot possibly find
    a \emph{one-to-one} map. This map would fail to be invertible or the inverse would not be
    continuous, so $(M,\symcal{O}_{\text{std}}|_M)$ is not a manifold.}
  \label{fig:mf-three-lines-wrong-map}
\end{figure}

So $M$ is not a topological manifold.

It is not difficult to come up with stuff that is not a topological manifold. String theorists,
for instance, like to think about d-branes with a little string attached to them as in
figure~\ref{fig:mf-dbrane-string}. But this the whole object is not a manifold, as the brane is
two dimensional, while the string is one dimensional, but the definition requires the same
dimension applied everywhere.
\begin{figure}[ht]
  \def\scl{.5}
  % 
  \pgfmathsetmacro{\PoneX}{-2.0}
  \pgfmathsetmacro{\PoneY}{2.0}
  \pgfmathsetmacro{\PtwoX}{0.3}
  \pgfmathsetmacro{\PtwoY}{3.0}
  \pgfmathsetmacro{\PthreeX}{-.9}
  \pgfmathsetmacro{\PthreeY}{-2.8}
  \pgfmathsetmacro{\PfourX}{2.5}
  \pgfmathsetmacro{\PfourY}{-1.0}
  \pgfmathsetmacro{\PX}{0}
  \pgfmathsetmacro{\PY}{.8}
  \pgfmathsetmacro{\MX}{0.9}
  \pgfmathsetmacro{\MY}{1.5}
  \pgfmathsetmacro{\EX}{3.5}
  \pgfmathsetmacro{\EY}{1.7}
  \pgfmathsetmacro{\BGTOP}{4.0}
  \pgfmathsetmacro{\BGBOTTOM}{-4.0}
  \pgfmathsetmacro{\BGRIGHT}{5.4}
  \pgfmathsetmacro{\BGLEFT}{-4.4}
  \centering
  \begin{tikzpicture}[%
    scale=\scl,
    dbrane/.style={dotted,fill=black!12,draw=black},
    string/.style={draw=black,line width=1.2pt},
    background/.style={
      line width=\bgborderwidth,
      draw=\bgbordercolor,
      fill=\bgcolor,
    },      
    ]
    % Coordinates
    \coordinate (O) at (0,0);
    % 
    \coordinate (P1) at (\PoneX,\PoneY);
    \coordinate (P2) at (\PtwoX,\PtwoY);
    \coordinate (P3) at (\PthreeX,\PthreeY);
    \coordinate (P4) at (\PfourX,\PfourY);
    % 
    \coordinate (P) at (\PX,\PY);
    \coordinate (M) at (\MX,\MY);
    \coordinate (E) at (\EX,\EY);
    % 
    \coordinate (bgtop) at (0,\BGTOP);
    \coordinate (bgleft) at (\BGLEFT,0);
    \coordinate (bgright) at (\BGRIGHT,0);
    \coordinate (bgbottom) at (0,\BGBOTTOM);
    % Plane
    % \filldraw[fill=red,draw=black] (P1) circle [radius=1pt];
    % \filldraw[fill=red,draw=black] (P2) circle [radius=1pt];
    % \filldraw[fill=red,draw=black] (P3) circle [radius=1pt];
    % \filldraw[fill=red,draw=black] (P4) circle [radius=1pt];
    % d-Brane
    \filldraw[dbrane]
    (P1) to[bend right]
    (P2) to[out=290,in=110]
    (P4) to[bend right]
    (P3) to[out=100,in=300]
    (P1);
    % String
    \filldraw[fill=black,draw=black] (P) circle [radius=1.5pt];
    % \filldraw[fill=black,draw=black] (M) circle [radius=1pt];
    % \filldraw[fill=black,draw=black] (E) circle [radius=1pt];
    \draw[string] (P) to[out=60,in=200] (M);
    \draw[string] (M) to[out=20,in=250] (E);
    \filldraw[fill=\bgcolor,draw=black] (E) circle[radius=2pt];
    % YELLOW BACKGROUND
    % 
    % \filldraw[fill=red,draw=black] (bgtop) circle[radius=1pt];
    % \filldraw[fill=red,draw=black] (bgbottom) circle[radius=1pt];
    % \filldraw[fill=green,draw=black] (bgleft) circle[radius=1pt];
    % \filldraw[fill=green,draw=black] (bgright) circle[radius=1pt];
    % \filldraw[fill opacity=0,draw=black] (O) circle[radius=2pt];
    \begin{scope}[on background layer]
      \node [background, fit= (bgleft) (bgright) (bgtop) (bgbottom)] {};
    \end{scope}
  \end{tikzpicture}
  \caption{A string attached to a d-brane is not a manifold.}
  \label{fig:mf-dbrane-string}
\end{figure}


\subsection{Terminology review}
\begin{itemize}
\item The open set $U$ and the map $x$ is called a chart
  \[
    (U,x) \text{ is a chart of } (M,\symcal{O}_M)
  \]
\item An atlas $\symcal{A}$ of the topological manifold is a possibly huge collection of charts
  with the condition that the union of all the chart domains give the manifold $M$
  \[
    \symcal{A} = \set{(U_{(\alpha)},x_{(\alpha)} | \alpha\in A}
    \text{ is called an atlas if } M = \bigcup_{\alpha\in A} U_\alpha
  \]
  $\alpha$ is just a label, which comes from an arbitrary index set $A$.
\item A chart map is one which takes any open set in the manifold into a subset of $\symbb{R}^d$
  \[
    x : U \longrightarrow x(U)\subseteq\symbb{R}^d
    \text{ is a chart map}
  \]
  
  The map image of any point $p$ in the open set is a d-tuple
  \[
    x(p) = \left(x^1(p), x^2(p),\cdots,x^d(p)\right)
    \text{ where }
    x^i : U \longrightarrow \symbb{R}
    \hspace{1em}
    i = 1, 2, \cdots, d
  \]
  So the map $x$ give the same information as $d$-many maps $x^i$.
  The $x^i$ are called the \emph{coordinate maps}
  \[
    \left.
      \begin{array}{c}
        x^1 : U \longrightarrow \symbb{R}\\
        x^2 : U \longrightarrow \symbb{R}\\
        \cdots\\
        x^d : U \longrightarrow \symbb{R}
      \end{array}
    \right\}
    \iff
    x : U \longrightarrow \symbb{R}^d
  \]
  
  The surface of the Earth is two dimensional, so we can print charts on sheets of paper.
  Unfortunately you cannot print an atlas of spacetime on sheets of paper. You need to print it
  on four-dimensional sheets of paper, which mathematically we can do but practically we can't.
  
\item If $p$ is any point in the open set $U$ ($p\in U$), then:
  \vspace{-4ex}
  \begin{center}
    \begin{tabular}{l}
      $x^1$ is the first coordinate of point $p$ with respecto to the chosen chart $(U,x)$\\
      $x^2$ is the second coordinate of point $p$ with respecto to the chosen chart $(U,x)$\\
      $\cdots$
    \end{tabular}
  \end{center}
  
  It is very important to understand what is the real world and what is a concept in our head.
  In physics, topological manifolds and their open sets are part of the \emph{real world}.
  But charts have nothing to do with physics, they are only in our heads, although they may be
  very useful. An atlas is extremely useful, but nobody would say that, because the publisher
  prints an atlas upside-down, it would be less comfortable to live in that part of the world.
  
  An atlas is something imagined. Most of undergraduate physics students tend to confuse what
  is imagined and what is real. It took three years to Einstein to clarify this. Possibly it
  could take us three days, so that's excellent.
  
  This is a key point. We know from lagrangian mechanics, that if we have a physical system
  that moves in a configuration space, we can choose generalized coordinates. Choosing
  coordinates is like choosing charts. A stone thrown around here is not affected by how I
  imagine charts or if we deform them more or less.
\end{itemize}

\subsubsection{Example}
First, we must understand that $\symbb{R}^2$ is just a set. For example, if $\set{1,2,3,4}$
is a set, then element permutations like $\set{2,1,4,3}$ or $\set{3,4,1,2}$ represent the same
set. Many people think of the a subset of $M\subseteq\symbb{R}^2$ as a piece of paper sheet,
but they should not. Instead, imagine that we take that sheet of paper, then rip it appart and
grind it into powder, put it in a box, shake it, and inside we have a better image for
$\symbb{R}^2$ (at least a subset of it).

But if we give it a little more structure. For example, if we equip it with the standard
topology $(M,\theta_{\text{std}}|_M)$, then we could take this topological space like a rubber
sheet. We could deform it the way we like but we could not rip it appart because it would be
non-continuous. So the $\symbb{R}^2$ that we usually think of is, at least, a topological
space with the standard topology. If we add more structure, a dot product, then we could
measure lengths and distances, and our image of the subset of $\symbb{R}^2$ would be that of
a rigid plane which we couldn't deform.

Now we take for the real world the points in $\symbb{R}^2$. We are now going to chart these
points except the origin, $U=\symbb{R}^2\setminus\set{(0,0)}$, for example:
\[
  \begin{array}{rcl}
    x : U & \longrightarrow & x(U)\subset\symbb{R}^2\\
    (m,n) & \longmapsto & (-m,-n)
  \end{array}
\]
which is represented in figure~\ref{fig:mf-cartesian-chart-map-example}.
\begin{figure}[ht]
  \def\scl{.75}
  % 
  % Rectangle
  \pgfmathsetmacro{\RECLENGTH}{3.5}
  \pgfmathsetmacro{\RECHEIGHT}{3.0}
  \pgfmathsetmacro{\RECX}{-1.4}
  \pgfmathsetmacro{\RECY}{-1.5}
  % F base X and Y coordinates
  \pgfmathsetmacro{\FBASEX}{.25}
  \pgfmathsetmacro{\FBASEY}{.55}
  % F's vertical bar length
  \pgfmathsetmacro{\FHEIGHT}{.5}
  % F's relative medium position (0-1)
  \pgfmathsetmacro{\MED}{.55}
  % F's top horizontal bar length
  \pgfmathsetmacro{\FTOPLENGTH}{.4}
  % F's medium horizontal bar length
  \pgfmathsetmacro{\FMEDLENGTH}{.32}
  % MAP ORIGIN
  \pgfmathsetmacro{\MAPORIGX}{6.0}
  \pgfmathsetmacro{\MAPORIGY}{-2.0}
  % MAP x length
  \pgfmathsetmacro{\MAPXLENGTH}{2.0}
  \pgfmathsetmacro{\MAPNXLENGTH}{.5}
  % MAP y length
  \pgfmathsetmacro{\MAPYLENGTH}{2.0}
  \pgfmathsetmacro{\MAPNYLENGTH}{.5}
  % Background
  \pgfmathsetmacro{\BGTOP}{2.0}
  \pgfmathsetmacro{\BGBOTTOM}{-3.6}
  \pgfmathsetmacro{\BGRIGHT}{9.0}
  \pgfmathsetmacro{\BGLEFT}{-2.0}
  \centering
  \begin{tikzpicture}[%
    scale=\scl,
    R2/.style={%
      fill=green!50,draw=black,dotted,
    },
    letter/.style={%
      line width=1.8pt,
      draw=black,
    },
    arrow/.style={%
      -{Latex[round]},shorten >=2pt, shorten <=1pt,draw=black!40,
    },
    background/.style={%
      line width=\bgborderwidth,
      draw=\bgbordercolor,
      fill=\bgcolor,
    },
    ]
    % COORDINATES
    % Origin
    \coordinate (O) at (0,0);
    % R2
    \coordinate (bl) at (\RECX, \RECY);
    \coordinate (tl) at (\RECX,\RECY+\RECHEIGHT);
    \coordinate (br) at (\RECX+\RECLENGTH,\RECY);
    \coordinate (tr) at (\RECX+\RECLENGTH,\RECY+\RECHEIGHT);
    \path (bl) -- (tl) coordinate[pos=.15] (R2);
    % F
    \coordinate (Fbase) at (\FBASEX,\FBASEY);
    \path (Fbase) -- +(up:\FHEIGHT) coordinate (Ftop);
    \path (Fbase) -- +(up:\FHEIGHT) coordinate[pos=\MED] (Fmed);
    % Map
    \coordinate (Omap) at (\MAPORIGX,\MAPORIGY);
    \coordinate (Fbasemap) at ($(Omap) + (-\FBASEX,-\FBASEY)$);
    \path (Fbasemap) -- +(down:\FHEIGHT) coordinate (Ftopmap);
    \path (Fbasemap) -- +(down:\FHEIGHT) coordinate[pos=\MED] (Fmedmap);
    % Background coordinates
    \coordinate (bgtop) at (0,\BGTOP);
    \coordinate (bgleft) at (\BGLEFT,0);
    \coordinate (bgright) at (\BGRIGHT,0);
    \coordinate (bgbottom) at (0,\BGBOTTOM);
    % DRAWING
    % R2
    \filldraw[R2] (bl) rectangle (tr);
    \node[right=2pt] at (R2) {\small $\symbb{R}^2$};
    % F domain
    % F's vertical and top horizontal lines
    \draw[letter] (Fbase) -- (Ftop) -- +(right:\FTOPLENGTH) coordinate (FtopRight);
    % F's horizontal medium line
    \draw[letter] (Fmed) -- +(right:\FMEDLENGTH) coordinate (FmedRight);
    % Origin
    \filldraw[fill=\bgcolor,draw=black] (O) circle [radius=1.5pt];
    \node[below right=0pt and -4pt] at (O) {$\scriptscriptstyle (0,0)$};
    % MAP
    % Map axes
    \draw[->] (Omap) -- +(right:\MAPXLENGTH) node[below] {\scriptsize $x^1(p)$};
    \draw (Omap) -- +(left:\MAPNXLENGTH);
    \draw[->] (Omap) -- +(up:\MAPYLENGTH) node[left] {\scriptsize $x^2(p)$};
    \draw (Omap) -- +(down:\MAPNYLENGTH);
    % Map origin
    \filldraw[fill=black,draw=black] (Omap) circle [radius=.5pt];
    % Map F
    \draw[letter] (Fbasemap) -- (Ftopmap) -- +(left:\FTOPLENGTH) coordinate (FtopRightmap);
    \draw[letter] (Fmedmap) -- +(left:\FMEDLENGTH) coordinate (FmedRightmap);
    \filldraw[fill=\bgcolor,draw=black] (Omap) circle [radius=1.5pt];
    % Arrows
    \draw[arrow] (FmedRight) to[bend left] (FmedRightmap);
    \draw[arrow] (Ftop) to[out=40,in=0] node[above=5pt,pos=.35] {\small $x$} (Ftopmap);
    % YELLOW BACKGROUND
    % 
    % \filldraw[fill=red,draw=black] (bgtop) circle[radius=1pt];
    % \filldraw[fill=red,draw=black] (bgbottom) circle[radius=1pt];
    % \filldraw[fill=green,draw=black] (bgleft) circle[radius=1pt];
    % \filldraw[fill=green,draw=black] (bgright) circle[radius=1pt];
    % \filldraw[fill opacity=0,draw=black] (O) circle[radius=2pt];
    \begin{scope}[on background layer]
      \node [background, fit= (bgleft) (bgright) (bgtop) (bgbottom)] {};
    \end{scope}
  \end{tikzpicture}
  \caption{A chart map $(U,x)$ from $\symbb{R}^2\setminus\set{(0,0)}$ into itself, $x(U)$.}
  \label{fig:mf-cartesian-chart-map-example}
\end{figure}

In physics, we always ignore the real world. We only talk about charts representing it.
This is like placing little flags on the real world points with something like
``coordinate (1,2)'' on them and so on. Putting flags on every point of a real world zone is
inventing a chart map. But even if we \emph{invert} or distort $x(U)$ in our chart map, the
real world remains unaffected.

Let us take for the real world the set $\symbb{R}^2$, and consider a different chart map on
another domain, like $U = \symbb{R}^2 \setminus\set{(m,0)|m\in\symbb{R}_0^+}$, where we take
out the $m\geq 0$ from $\symbb{R}^2$
\[
  \begin{array}{rcl}
    y : U & \longrightarrow & x(U)\subset\symbb{R}^2\\
    (m,n) & \longmapsto & \left(\sqrt{m^2+n^2},\arctan{\left(\dfrac{n}{m}\right)}\right)
  \end{array}
\]
where the first coordinate is the distance from the origin and the second one is the angle which
forms the position vector with the $x$ axes.
Now we draw a circle, the result can be seen in figure~\ref{fig:mf-polar-chart-map-example}.

Charts may look very different from the real world. We need to know how the charts are
related to the real world.

\begin{figure}[ht]
  \def\scl{.75}
  % Rectangle
  \pgfmathsetmacro{\RECLENGTH}{3.5}
  \pgfmathsetmacro{\RECHEIGHT}{3.5}
  \pgfmathsetmacro{\RECX}{-1.4}
  \pgfmathsetmacro{\RECY}{-1.5}
  % R2 ORIGIN
  \pgfmathsetmacro{\ORIGX}{\RECX + .5*\RECLENGTH}
  \pgfmathsetmacro{\ORIGY}{\RECY + .5*\RECHEIGHT}
  \pgfmathsetmacro{\ORIGRADIUS}{1.5}
  % MYCIRCLE
  \pgfmathsetmacro{\MYCIRCLERADIUS}{.95}
  % MAP ORIGIN
  \pgfmathsetmacro{\MAPORIGX}{4.2}
  \pgfmathsetmacro{\MAPORIGY}{-1.0}
  % MAP x length
  \pgfmathsetmacro{\MAPXLENGTH}{5.2}
  \pgfmathsetmacro{\MAPNXLENGTH}{0}
  % MAP y length
  \pgfmathsetmacro{\MAPYLENGTH}{2.5}
  \pgfmathsetmacro{\MAPNYLENGTH}{0}
  % 
  \pgfmathsetmacro{\MAPIMAGELENGTH}{2.9}
  \pgfmathsetmacro{\MAPIMAGEHEIGHT}{1.2}
  % Infinity arrows length
  \pgfmathsetmacro{\MAPINFTYARROWLENGTH}{.90}
  \pgfmathsetmacro{\INFTYPOS}{\MAPINFTYARROWLENGTH + .2}
  % MAP circle image
  \pgfmathsetmacro{\MAPCIRCLERELPOS}{.4}
  % Background
  \pgfmathsetmacro{\BGTOP}{2.6}
  \pgfmathsetmacro{\BGBOTTOM}{-2.2}
  \pgfmathsetmacro{\BGRIGHT}{10.3}
  \pgfmathsetmacro{\BGLEFT}{-2.0}
  \centering
  \begin{tikzpicture}[%
    scale=\scl,
    R2/.style={%
      fill=green!40,draw=black,dotted,
    },
    mycircle/.style={%
      black, line width=1pt,
    },
    auxiliary line/.style={%
      fill=\bgcolor,
      draw=\bgcolor,
      ultra thin,
    },
    open end circle/.style={%
      fill=\bgcolor,
      draw=black,
      ultra thin,
    },
    infinity arrow/.style={%
      -{Latex[round]},
      green!90,
    },
    arrow/.style={%
      -{Latex[round]},shorten >=2pt, shorten <=1pt,draw=black!40,
    },
    background/.style={%
      line width=\bgborderwidth,
      draw=\bgbordercolor,
      fill=\bgcolor,
    },
    ]
    % COORDINATES
    % R2
    \coordinate (bl) at (\RECX, \RECY);
    \coordinate (tl) at (\RECX,\RECY+\RECHEIGHT);
    \coordinate (br) at (\RECX+\RECLENGTH,\RECY);
    \coordinate (tr) at (\RECX+\RECLENGTH,\RECY+\RECHEIGHT);
    \path (bl) -- (tl) coordinate[pos=.15] (R2);
    % Origin
    \coordinate (O) at (\ORIGX,\ORIGY);
    \coordinate (Otop) at ($(O) + (0,\ORIGRADIUS pt)$);
    \coordinate (Obottom) at ($(O) - (0,\ORIGRADIUS pt)$);
    % Intersections
    \path [name path=topaux] (Otop) -- +(right:10);
    \path [name path=bottomaux] (Obottom) -- +(right:10);
    \path [name path=recright] ($(br)-(.02pt,0)$) -- ($(tr)-(.02pt,0)$);
    \path [name intersections={of=topaux and recright,by=T}];
    \path [name intersections={of=bottomaux and recright,by=B}];
    \path [name path=mycircle] (O) circle[radius=\MYCIRCLERADIUS];
    \path [name intersections={of=mycircle and topaux,by=CT}];
    \path [name intersections={of=mycircle and bottomaux,by=CB}];
    % Map
    \coordinate (Omap) at (\MAPORIGX,\MAPORIGY);
    \coordinate (mapx) at (\MAPORIGX+\MAPIMAGELENGTH,\MAPORIGY);
    \coordinate (mapy) at (\MAPORIGX,\MAPORIGY+\MAPIMAGEHEIGHT);
    \coordinate (mapxy) at (\MAPORIGX+\MAPIMAGELENGTH,\MAPORIGY+\MAPIMAGEHEIGHT);
    % 
    \coordinate (mapcirclebottom) at ($(Omap)!\MAPCIRCLERELPOS!(mapx)$);
    \coordinate (mapcircletop) at ($(mapy)!\MAPCIRCLERELPOS!(mapxy)$);
    % Background coordinates
    \coordinate (bgtop) at (0,\BGTOP);
    \coordinate (bgleft) at (\BGLEFT,0);
    \coordinate (bgright) at (\BGRIGHT,0);
    \coordinate (bgbottom) at (0,\BGBOTTOM);
    % 
    % DRAWING
    % R2
    \filldraw[R2] (bl) rectangle (tr);
    \node[right=2pt] at (R2) {\small $\symbb{R}^2$};
    % mycircle
    \draw[mycircle] (O) circle[radius=\MYCIRCLERADIUS];
    % my circle's auxiliary lines
    \filldraw[auxiliary line] (Obottom) rectangle (T);
    \draw[\bgcolor,very thick] (T) -- (B);
    \filldraw[open end circle] (CT) circle[radius=.62*\ORIGRADIUS pt];
    \filldraw[open end circle] (CB) circle[radius=.62*\ORIGRADIUS pt];
    % Origin
    \filldraw[fill=\bgcolor,draw=black] (O) circle [radius=\ORIGRADIUS pt];
    % \node[below left=0pt and -4pt] at (O) {$\scriptscriptstyle (0,0)$};
    % 
    % MAP
    % Map axes
    \draw[->] (Omap) -- +(right:\MAPXLENGTH) node[below=2pt] {\scriptsize $x^1(p)$};
    \draw (Omap) -- +(left:\MAPNXLENGTH);
    \draw[->] (Omap) -- +(up:\MAPYLENGTH) node[left] {\scriptsize $x^2(p)\,(\si{\radian})$};
    \draw (Omap) -- +(down:\MAPNYLENGTH);
    % Chart map
    \filldraw[fill=green!40,draw=green!40] (Omap) rectangle (mapxy);
    \draw[dotted,black] (Omap) -- (mapy);
    \draw[dotted,black] (Omap) -- (mapx);
    \draw[dotted,black] (mapy) -- (mapxy);
    \foreach \y in {.20,.40,...,.80}
    \draw[infinity arrow] ($(mapx)!\y!(mapxy)$) -- +(right:\MAPINFTYARROWLENGTH);
    
    \node[right=\INFTYPOS,\greentext] at ($(mapx)!.5!(mapxy)$) {\small $\infty$};
    % Circle image
    \draw[mycircle] (mapcirclebottom) -- (mapcircletop);
    \filldraw[open end circle] (mapcirclebottom) circle[radius=.7*\ORIGRADIUS pt];
    \filldraw[open end circle] (mapcircletop) circle[radius=.7*\ORIGRADIUS pt];
    % Map origin
    \filldraw[fill=\bgcolor,draw=black] (Omap) circle [radius=\ORIGRADIUS pt];
    % Values
    \node[left] at (mapy) {\footnotesize $2\pi$};
    % Arrows
    % \draw[arrow] (FmedRight) to[bend left] (FmedRightmap);
    % \draw[arrow] (Ftop) to[out=20,in=0] node[above=5pt,pos=.35] {\small $x$} (Ftopmap);
    
    % YELLOW BACKGROUND
    % 
    % \filldraw[fill=red,draw=black] (bgtop) circle[radius=1pt];
    % \filldraw[fill=red,draw=black] (bgbottom) circle[radius=1pt];
    % \filldraw[fill=green,draw=black] (bgleft) circle[radius=1pt];
    % \filldraw[fill=green,draw=black] (bgright) circle[radius=1pt];
    % \filldraw[fill opacity=0,draw=black] (O) circle[radius=2pt];
    \begin{scope}[on background layer]
      \node [background, fit= (bgleft) (bgright) (bgtop) (bgbottom)] {};
    \end{scope}
  \end{tikzpicture}
  \caption{A chart map $(U,y)$ which uses something like polar coordinates, from
    $U=\symbb{R}^2\setminus\set{(0,0)}$ into $y(U)\subset\symbb{R}^2$, in which we can see a
    circle drawn in the real world and its image on the chart map.}
  \label{fig:mf-polar-chart-map-example}
\end{figure}

\subsection{Chart transition maps}
These are objects derived from certain charts we already have, which are heavily used by
physicists.

Imagine two charts $(U,x)$ and $(V,y)$ with overlapping regions, so $U\cap V \neq\emptyset$.
So $U$ and $V$ come with some chart maps which take any point in the set to some point in
$\symbb{R}^d$. The charts have a little overlap in order for us to navigate through them.

Now we are going to restrict the $x$ and $y$ maps to the intersection region $U\cap V$
and their images $x(U\cap V)$ and $y(U\cap V)$ with dark green in
figure~\ref{fig:mf-chart-transition-maps-one}.

\begin{figure}[ht]
  \def\scl{.7}
  % 
  % Plane (bottom left)
  \pgfmathsetmacro{\PLANEBLX}{-3.0}
  \pgfmathsetmacro{\PLANEBLY}{-.5}
  % Plane (bottom right)
  \pgfmathsetmacro{\PLANEBRX}{2.1}
  \pgfmathsetmacro{\PLANEBRY}{-.8}
  % Plane (top left)
  \pgfmathsetmacro{\PLANETLX}{-1.5}
  \pgfmathsetmacro{\PLANETLY}{2.0}
  % Plane (top right)
  \pgfmathsetmacro{\PLANETRX}{3.0}
  \pgfmathsetmacro{\PLANETRY}{2.0}
  % M text 
  \pgfmathsetmacro{\MX}{3.4}
  \pgfmathsetmacro{\MY}{1.9}
  % Ellipse a
  \pgfmathsetmacro{\EAX}{-.67}
  \pgfmathsetmacro{\EAY}{.80}
  \pgfmathsetmacro{\EAA}{.93}
  \pgfmathsetmacro{\EAB}{.60}
  \def\ellipseA{(\EAX,\EAY) ellipse (\EAA cm and \EAB cm)}
  % Ellipse b
  \pgfmathsetmacro{\EBX}{.40}
  \pgfmathsetmacro{\EBY}{.73}
  \pgfmathsetmacro{\EBA}{1.2}
  \pgfmathsetmacro{\EBB}{.73}
  \def\ellipseB{(\EBX,\EBY) ellipse (\EBA cm and \EBB cm)}
  % U and V text
  \pgfmathsetmacro{\UX}{-1.73}
  \pgfmathsetmacro{\UY}{.20}
  \pgfmathsetmacro{\VX}{1.30}
  \pgfmathsetmacro{\VY}{-.1}
  % Coordinate system (common)
  \pgfmathsetmacro{\ORIGRADIUS}{1.0}
  \pgfmathsetmacro{\ORIGY}{-5.0}
  \pgfmathsetmacro{\XAXISLENGTH}{4.0}
  \pgfmathsetmacro{\YAXISLENGTH}{3.2}
  % Left coordinate system
  \pgfmathsetmacro{\LORIGX}{-4.0}
  % Right coordinate system
  \pgfmathsetmacro{\RORIGX}{3.0}
  % Background
  \pgfmathsetmacro{\BGTOP}{2.8}
  \pgfmathsetmacro{\BGBOTTOM}{-6.0}
  \pgfmathsetmacro{\BGRIGHT}{8.0}
  \pgfmathsetmacro{\BGLEFT}{-5.5}
  \centering
  \begin{tikzpicture}[%
    scale=\scl,
    R2/.style={%
      fill=green!40,draw=black,dotted,
    },
    open set/.style={%
      fill=green!17,draw=\bgcolor,
    },
    intersection1/.style={%
      fill=green!60,draw=green!17,
    },
    intersection2/.style={%
      fill=green!60,draw=black,
    },
    arrow/.style={%
      -{Latex[round]},shorten >=2pt, shorten <=1pt,draw=black!40,
    },
    origin/.style={%
      fill=black, draw=black,
    },
    axis/.style={%
      ->,
    },
    arrow/.style={%
      -{Latex[round]},shorten >=2pt, shorten <=2pt,draw=black!40,
    },
    background/.style={%
      line width=\bgborderwidth,
      draw=\bgbordercolor,
      fill=\bgcolor,
    },
    ]
    % 
    % COORDINATES
    % Plane
    \coordinate (bl) at (\PLANEBLX,\PLANEBLY);
    \coordinate (br) at (\PLANEBRX,\PLANEBRY);
    \coordinate (tl) at (\PLANETLX,\PLANETLY);
    \coordinate (tr) at (\PLANETRX,\PLANETRY);
    % M text
    \coordinate (M) at (\MX,\MY);
    % Ellipses a and b
    \coordinate (EA) at (\EAX,\EAY);
    \coordinate (EB) at (\EBX,\EBY);
    % U and V text
    \coordinate (U) at (\UX,\UY);
    \coordinate (V) at (\VX,\VY);
    % Left coordinate system
    \coordinate (LO) at (\LORIGX,\ORIGY);
    \coordinate (LX) at (\LORIGX+\XAXISLENGTH,\ORIGY);
    \coordinate (LY) at (\LORIGX,\ORIGY+\YAXISLENGTH);
    % Right coordinate system
    \coordinate (RO) at (\RORIGX,\ORIGY);
    \coordinate (RX) at (\RORIGX+\XAXISLENGTH,\ORIGY);
    \coordinate (RY) at (\RORIGX,\ORIGY+\YAXISLENGTH);
    % Point p
    \coordinate (p) at (-.35,.76);
    \coordinate (xp) at (-2.9,-3.2);
    \coordinate (yp) at (4.2,-3.4);
    % Background coordinates
    \coordinate (bgtop) at (0,\BGTOP);
    \coordinate (bgleft) at (\BGLEFT,0);
    \coordinate (bgright) at (\BGRIGHT,0);
    \coordinate (bgbottom) at (0,\BGBOTTOM);
    % 
    % DRAWING
    % Plane and text
    \filldraw[dotted,fill=black!17,draw=black] (bl)
    to[out=50,in=260] (tl) to[out=350,in=185] (tr)
    to[bend right] (br) to[out=170,in=10] cycle;
    \draw[dotted] (bl)
    to[out=50,in=260] (tl) to[out=350,in=185] (tr)
    to[bend right] (br) to[out=170,in=10] cycle;
    \node at (M) {\footnotesize $M$};
    % Ellipses and text
    \filldraw[open set] \ellipseA;
    \node at (U) {\footnotesize $U$};
    \filldraw[open set] \ellipseB;
    \node at (V) {\footnotesize $V$};
    \begin{scope}[even odd rule]
      \clip \ellipseA;
      \filldraw[intersection1] \ellipseB;
    \end{scope}
    \draw[dotted] \ellipseA;
    \draw[dotted] \ellipseB;
    % Intersection text
    \node[\greentext] at (0,1.63) {\scriptsize $U\cap V$};
    % Left coordinate system
    \filldraw[origin] (LO) circle[radius=\ORIGRADIUS pt];
    \draw[axis] (LO) -- (LX);
    \draw[axis] (LO) -- (LY);
    % Right coordinate system
    \filldraw[origin] (RO) circle[radius=\ORIGRADIUS pt];
    \draw[axis] (RO) -- (RX);
    \draw[axis] (RO) -- (RY);
    % x(U)
    \begin{scope}
      \filldraw[dotted,open set] (-3.2,-2.9) to[bend left] (-2.2,-2.7)
      to [bend left] (-1.3,-3.7) to[bend left] (-1.4,-4.0) to[bend left] (-2.8,-4.4)
      to[bend left] (-3.42,-3.4) to[bend left] cycle;
      \draw[dotted] (-3.2,-2.9) to[bend left] (-2.2,-2.7) to [bend left] (-1.3,-3.7)
      to[bend left] (-1.4,-4.0) to[bend left] (-2.8,-4.4) to[bend left] (-3.42,-3.4)
      to[bend left] cycle;
    \end{scope}
    % Intersection on x(U)
    \begin{scope}
      \clip (-3.2,-2.9) to[bend left] (-2.2,-2.7) to [bend left] (-1.3,-3.7)
      to[bend left] (-1.4,-4.0) to[bend left] (-2.8,-4.4) to[bend left] (-3.42,-3.4)
      to[bend left] cycle;
      % x(intersection)
      \filldraw[dotted,intersection1] plot [smooth cycle] coordinates{%
        (-3.72,-2.65)
        (-2.72,-2.65) (-2.4,-3.0) (-2.43,-3.3) (-2.65,-3.7) (-3.2,-4.07)
        (-3.72,-4.07)};
      \draw[dotted] plot [smooth cycle] coordinates{%
        (-3.72,-2.65)
        (-2.72,-2.65) (-2.4,-3.0) (-2.43,-3.3) (-2.65,-3.7) (-3.2,-4.07)
        (-3.72,-4.07)};
      \draw[dotted] (-3.2,-2.9) to[bend left] (-2.2,-2.7) to [bend left] (-1.3,-3.7)
      to[bend left] (-1.4,-4.0) to[bend left] (-2.8,-4.4) to[bend left] (-3.42,-3.4)
      to[bend left] cycle;
    \end{scope}
    % y(V)
    \begin{scope}
      \clip (3.8,-4.3) to[out=0,in=165] (5.8,-4.5)
      to[out=100,in=260] (6.0,-3.0) to[out=195,in=20] (4.0,-3.2) to cycle;
      \filldraw[open set] (3.8,-4.3) to[out=0,in=165] (5.8,-4.5)
      to[out=100,in=260] (6.0,-3.0) to[out=195,in=20] (4.0,-3.2) to cycle;
      \draw[dotted] (3.8,-4.3) to[out=0,in=165] (5.8,-4.5)
      to[out=100,in=260] (6.0,-3.0) to[out=195,in=20] (4.0,-3.2) to cycle;
      %% Intersection on Y(V)
      \filldraw[intersection1]
      (3.4,-4.3) to (3.8,-4.3) to[out=60,in=220] (4.4,-3.5) to[out=40,in=215] (6.0,-3.0)
      to (3.4,-3.0) -- cycle;
      \draw[\bgcolor] (3.8,-4.3) to[out=0,in=165] (5.8,-4.5)
      to[out=100,in=260] (6.0,-3.0) to[out=195,in=20] (4.0,-3.2) to cycle;
    \end{scope}
    \draw[dotted] (3.8,-4.3) to[out=0,in=165] (5.8,-4.5)
    to[out=100,in=260] (6.0,-3.0) to[out=195,in=20] (4.0,-3.2) to cycle;
    % Point p in the intersection
    \filldraw (p) circle[radius=.9pt];
    \node[above=-1pt] at (p) {\scriptsize $p$};
    \filldraw (xp) circle[radius=.9pt];
    \filldraw (yp) circle[radius=.9pt];
    % Arrows
    \draw[arrow] (p) to[bend left] node[pos=.55,above left] {\small $x$} (xp);
    \draw[arrow] (p) to[bend right] node[pos=.5,above right] {\small $y$} (yp);
    % x(U) and x(instersection)
    \node[\greentext] at (-3.1,-2.4) {\scriptsize $x\,(U\cap V)$};
    \node at (-1.2,-4.5) {\scriptsize $x\,(U)$};
    % y(U) and y(instersection)
    \node[\greentext] at (5.0,-2.7) {\scriptsize $y\,(U\cap V)$};
    \node at (6.5,-4.3) {\scriptsize $y\,(U)$};
    % Chart transition map arrow
    % \draw[arrow] (xp) to[bend right] node[pos=.6,below] {\small $y\circ x^{-1}$} (yp);
    \draw[arrow,shorten >=5pt,line width=1.2pt]
    (xp) to[out=345,in=195] node[below] {\small $y\circ x^{-1}$} (yp);
    % 
    % YELLOW BACKGROUND 
    % \filldraw[fill=red,draw=black] (bgtop) circle[radius=1pt];
    % \filldraw[fill=red,draw=black] (bgbottom) circle[radius=1pt];
    % \filldraw[fill=green,draw=black] (bgleft) circle[radius=1pt];
    % \filldraw[fill=green,draw=black] (bgright) circle[radius=1pt];
    % \filldraw[fill opacity=0,draw=black] (O) circle[radius=2pt];
    \begin{scope}[on background layer]
      \node [background, fit= (bgleft) (bgright) (bgtop) (bgbottom)] {};
    \end{scope}      
  \end{tikzpicture}
  \caption{$y\circ x^{-1}$ is the chart transition map, which translates the coordinates of
    $p\in U\cup V$ from map $(U,x)$ to the other $(V,y)$.}
  \label{fig:mf-chart-transition-maps-one}
\end{figure}

Now imagine that we are studying the map $x(U)$, and when we are interested in point $p$ in the
intersection of $U$ and $V$. Point $p$ is mapped as $x(p)$ in the chart map $(U,x)$.
Then we would like to move to the chart $(V,y)$ map, and locate the point $y(p)$ in it.
In other words, it is often very useful to translate the coordinates of $p$ in the $(U,x)$
chart map to the ones of the same point in the $(V,y)$ map.

This can be achieved with the map $y\circ x^{-1}$ ($y$ \emph{after} $x$ \emph{inverse})
applied to point $p$
\[
  (y\circ x^{-1})\,(p) = y\, (x^{-1} \,(p))
\]
Note that $x$ is invertible\footnotemark{}, so $x^{-1}\,(p)$ represents the point $p$ in the
real world.
\footnotetext{Both, $x$ and $y$ are invertible by definition of chart map, and in both directions
  continuous. The inverse $y^{-1}$ would be necessary translating the coordinates from
  $y\kern1pt(p) $ to $x\kern1pt (p)$.}

The manifold picture can be represented formally as
\vspace{-1ex}
\begin{center}
  \def\dist{.6}
  \begin{tikzcd}
    & U\cap V \arrow[dl,"x" above left=0pt and 0pt,shift right=\dist ex]
    \arrow[dr,"y" above right=0pt and 0pt] & \\
    \symbb{R}^d\supseteq x\,(U\cap V)
    \arrow[ur, "x^{-1}" below right=-2pt and 0pt, shift right=\dist ex]
    \arrow{rr}[below] {y\,\circ\kern.8pt x^{-1}}
    &   &
    y\,(U\cap V)\subseteq\symbb{R}^d
  \end{tikzcd}
\end{center}
where both $x\,(U\cap  V)$ and $y\,(U\cap V)$ are open subsets of $\symbb{R}^d$.
This chart transition map $y\circ x^{-1}$ is continuous because both $x$ and $y$ are
continuous and the composition of continuous maps (remember the end of the last chapter)
is also continuous.

$U\cap V$ represents the real world, but $x\,(U\cap V)$ and $y\,(U\cap V)$ are just imaginations.
The chart transition map is $y\circ x^{-1}$, defined as 
\begin{center}
  \def\dist{.6}
  \begin{tikzcd}
    x\,(U\cap V)\arrow{rr}[below] {y\,\circ\kern.8pt x^{-1}}
    &   &
    y\,(U\cap V)
  \end{tikzcd}
\end{center}
When physicists talk about a change of coordinates, let's say, between cartesian to spherical
coordinates, they never touch the real world $U\cap V$ as mathematical object. They only work
with chart maps, which are just imaginations.

This can lead us to the philosophical problem of. somehow, imagining that there is some reality
to the coordinates of points. We can even think of cartesian coordinates as being
real, and the others being not so real (so we could name the others as
\emph{generalized coordinates} for a reason.)
But this means nothing. The real world doesn't care whether we use cartesian coordinates or
any other or whether we look at one chart $(U,x)$ or to another chart $(V,x)$.
They are just notions that we invent to represent the real world.
This is something that took Einstein three years to fully understand.
He seemed to understood it, but not completely, which was wrong. We need to understand it
\SI{100}{\percent}.

Informally, but a good way to think of it, chart transition maps contain the instructions to glue
together the different charts of an atlas, see figure~\ref{fig:mf-chart-transition-maps-two}.
\begin{figure}[ht]
  \def\scl{1}
  % Rectangles
  \pgfmathsetmacro{\RECWIDTH}{2.8}
  \pgfmathsetmacro{\RECHEIGHT}{1.8}
  \pgfmathsetmacro{\RECGAP}{.5}
  \pgfmathsetmacro{\RECTXT}{.5*\RECWIDTH}
  % Ghost rectangle
  \pgfmathsetmacro{\GHOSTRECWIDTH}{1.0}
  \pgfmathsetmacro{\GHOSTRECHEIGHT}{\RECHEIGHT}
  % Dots in the ghost rectangle
  \pgfmathsetmacro{\DOTSNUM}{3}
  \pgfmathsetmacro{\DOTS}{\DOTSNUM - 1}
  \pgfmathsetmacro{\DOTSXSHIFT}{\GHOSTRECWIDTH / \DOTS}
  \pgfmathsetmacro{\DOTSYSHIFT}{\GHOSTRECHEIGHT / 2}
  \pgfmathsetmacro{\DOTRADIUS}{.6}
  % Background
  \pgfmathsetmacro{\BGTOP}{2.6}
  \pgfmathsetmacro{\BGBOTTOM}{-1.0}
  \pgfmathsetmacro{\BGRIGHT}{11.3}
  \pgfmathsetmacro{\BGLEFT}{-.4}
  %
  \centering
  \begin{tikzpicture}[%
    scale=\scl,
    dot/.style={%
      fill=black!70, draw=black!70,
    },
    chart/.style={%
      fill=black!10,draw=black!10,
    },
    open set/.style={%
      fill=green!17,draw=black,dotted,
    },
    arrow/.style={%
      -{Latex[round]},shorten >=2pt, shorten <=2pt,draw=black!50,
    },
    background/.style={%
      line width=\bgborderwidth,
      draw=\bgbordercolor,
      fill=\bgcolor,
    },
    ]
    % COORDINATES
    \coordinate (O) at (0,0);
    % Plane 1
    \coordinate (p1bl) at (O);
    \coordinate (p1tr) at ($(p1bl) + (\RECWIDTH,\RECHEIGHT)$);
    \coordinate (p1txt) at ($(p1bl) + (\RECTXT,\RECHEIGHT)$);
    % Plane 2
    \coordinate (p2bl) at ($(p1tr) + (\RECGAP,-\RECHEIGHT)$);
    \coordinate (p2tr) at ($(p2bl) + (\RECWIDTH,\RECHEIGHT)$);
    \coordinate (p2txt) at ($(p2bl) + (\RECTXT,\RECHEIGHT)$);
    % Ghost plane
    \coordinate (pgbl) at ($(p2tr) + (\RECGAP,-\RECHEIGHT)$);
    \coordinate (pgtr) at ($(pgbl) + (\GHOSTRECWIDTH,\GHOSTRECHEIGHT)$);
    % Plane 3
    \coordinate (p3bl) at ($(pgtr) + (\RECGAP,-\RECHEIGHT)$);
    \coordinate (p3tr) at ($(p3bl) + (\RECWIDTH,\RECHEIGHT)$);
    \coordinate (p3txt) at ($(p3bl) + (\RECTXT,\RECHEIGHT)$);
    % Background coordinates
    \coordinate (bgtop) at (0,\BGTOP);
    \coordinate (bgleft) at (\BGLEFT,0);
    \coordinate (bgright) at (\BGRIGHT,0);
    \coordinate (bgbottom) at (0,\BGBOTTOM);
    % DRAWING
    % Chart 1 and text
    \filldraw[chart] (p1bl) rectangle (p1tr);
    \node[above] at (p1txt) {\small Chart 1};
    % Chart 2 and text
    \filldraw[chart] (p2bl) rectangle (p2tr);
    \node[above] at (p2txt) {\small Chart 2};
    % Ghost plane (dots)
    \foreach \n in {0,...,\DOTS}{%
      \filldraw[dot] ($(pgbl) + (\n * \DOTSXSHIFT,\DOTSYSHIFT)$) circle[radius=\DOTRADIUS pt];
    }
    % Chart 3 and text
    \filldraw[chart] (p3bl) rectangle (p3tr);
    \node[above] at (p3txt) {\small Chart n};
    % chart 1
    \begin{scope}[shift=(p1bl)]
      \clip (p1bl) rectangle (p1tr);
      % Zone 1
      \filldraw[open set] plot [smooth cycle] coordinates{%
        (.4,.8) (.5,1.2) (.9,1.4) (1.3,1.1) (.9,.7)
      };
      % Zone 2
      \filldraw[open set] plot [smooth cycle] coordinates{%
        (1.8,.3) (1.9,.7) (2.2,1.0) (2.5,1.05) (3.1,0.9) (3.1,.5) (2.5,.25) (2.2,.2)
      };
      % Point p in zone 2
      \filldraw (2.2,.45) circle [radius=.6pt] coordinate (p1);
      \node[above] at (p1) {\scriptsize $p$};
    \end{scope}
    \begin{scope}[shift=(p2bl)]
      \clip (p2bl) rectangle (p2tr);
      % Zone 1
      \filldraw[open set] plot [smooth cycle,xshift=-1cm] coordinates{%
        (.4,.8) (.5,1.2) (.9,1.4) (1.3,1.1) (.9,.7)
      };
      % Zone 2
      \filldraw[open set] plot [smooth cycle,xshift=-1cm] coordinates{%
        (1.8,.3) (1.9,.7) (2.2,1.0) (2.5,1.05) (3.1,.9) (3.1,.5) (2.5,.25) (2.2,.2)
      };
      % Point p in zone 2
      \filldraw[xshift=-1cm] (2.2,.45) circle [radius=.6pt] coordinate (p2)
      node[above] {\scriptsize $p$};
      % Zone 3
      \filldraw[open set] plot [smooth cycle,xshift=-1cm] coordinates{%
        (3.5,1.4) (3.7,1.5) (3.9,1.45) (4.2,1.35) (4.0,1.20) (3.8,1.1)
      };
    \end{scope}
    \begin{scope}[shift=(p3bl)]
      % Zone 4
      \filldraw[open set] plot [smooth cycle,xshift=-1cm] coordinates{%
        (2.1,1.6) (2.8,1.55) (3.0,1.5) (2.9,1.2) (3.2,.8) (2.5,.5) (1.9,1.1)
      };
    \end{scope}
    % Plane 1
    \draw[black!17] (p1bl) rectangle (p1tr);
    \draw[dotted] (p1bl) rectangle (p1tr);
    % Plane 2
    \draw[black!17] (p2bl) rectangle (p2tr);
    \draw[dotted] (p2bl) rectangle (p2tr);
    % Plane 3
    \draw[black!17] (p3bl) rectangle (p3tr);
    \draw[dotted] (p3bl) rectangle (p3tr);
    % Arrow
    \draw[arrow] (p1) to[out=280,in=260] node[below] {\footnotesize $y\circ x^{-1}(p)$} (p2);
    % 
    % YELLOW BACKGROUND 
    % \filldraw[fill=red,draw=black] (bgtop) circle[radius=1pt];
    % \filldraw[fill=red,draw=black] (bgbottom) circle[radius=1pt];
    % \filldraw[fill=green,draw=black] (bgleft) circle[radius=1pt];
    % \filldraw[fill=green,draw=black] (bgright) circle[radius=1pt];
    % \filldraw[fill opacity=0,draw=black] (O) circle[radius=2pt];
    \begin{scope}[on background layer]
      \node [background, fit= (bgleft) (bgright) (bgtop) (bgbottom)] {};
    \end{scope}      
  \end{tikzpicture}
  \caption{A chart transition map contains the instructions how to glue together the charts
  of an atlas, which overlap a little so that we can navigate across its different charts.}
  \label{fig:mf-chart-transition-maps-two}
\end{figure}

Imagine you are the captain of a ship with no windows to see the real world. It's a long cruise and
you have nothing to do, so you are bored. You have a single-sided printed atlas of the world,
you then pull out the charts and decide to glue together the charts in the overlapped regions
with the help of chart transition maps (the gluing instructions). When finished, you get
a paper model of a sphere, topologically speaking.

Chart transition maps have all the information about the real world globally.
This is why chart maps and chart transition maps are so beloved by physicists, because they
globally represent the real world.
Only when we glue together the different charts of a atlas of the Earth, for example,  with chart
transition maps, we get a sphere. If we understood the charts by themselves, with no chart transition
maps, we could not have inferred that the Earth is like a sphere.

% 01:00:00
\subsection{Manifold philosophy}
Now we come to a very important point. What is the whole philosophy of writing down manifolds?
We will only understand this once we start discussing differentiability of curves and functions.
So far we have only talked about continuity ---no jumps---, but differentiability of curves mean
that everywhere there is a tangent, roughly speaking.

\subsubsection{Continuity}
Let's first focus our discussion on continuity.
The basic idea is the following:
The navigator of a ship can see the bubbles that the ship creates in the sea when it moves.
He also have an atlas of the real world. So he puts the course of the ship in the charts.
Would you be surprised if, all of a sudden, this course would jump to another point and
continue from there? You would be very much surprised, so you would ask: ``How did you do that?''

What is the argument that you would make to explain that this is unusual? You know that the map
from the real world to the chart is bijective and continuous, and that for the charts they have
chosen the standard topology in $\symbb{R}^2$, etc., so the discontinuity of the curve in the
charts is the image of the discontinuity of the curve in the real world.

So instead of defining continuity directly in the real world, which of course you can, because the
real world is a topological space and you can define continuity in curves, you could push the real
world object you want to study down to a chart, study the object in the chart, and then you get the
idea of the property in the real world from the one you have studied in the chart.

\begin{quote}
  ``Often it is desirable (or indeed the only way) to define properties (like continuity, etc.) of
  real world objects (for instance a curve $\gamma$ in the real world,
  $\symbb{R} \overset{\gamma}{\longrightarrow} M$) by judging suitable conditions not on
  the real world object itself, but on a chart-representative of that real world object.''
\end{quote}

The advantage of the an approach are that the course of the ship in the real world is a curve from
some real parameter into $M$.
Now we choose an open subset $U$ of $M$, which has a portion $\gamma$ of the  curve.
\[
  \symbb{R} \overset{\gamma}{\longrightarrow} M
\]
Next, we invent a chart $(U,x)$ which takes the $\gamma$ curve and maps this curve into the chart
which might look as in figure~\ref{fig:mf-curve-chart}.
\begin{figure}[ht]
  \def\scl{.7}
  % Coordinate system (common)
  \pgfmathsetmacro{\ORIGRADIUS}{.6}
  \pgfmathsetmacro{\ORIGX}{-3.0}
  \pgfmathsetmacro{\ORIGY}{-4.0}
  \pgfmathsetmacro{\XAXISLENGTH}{6.0}
  \pgfmathsetmacro{\YAXISLENGTH}{3.5}
  % X(U)
  \pgfmathsetmacro{\XUWIDTH}{2.9}
  \pgfmathsetmacro{\XUHEIGHT}{2}
  \pgfmathsetmacro{\XUORIGX}{.8}
  \pgfmathsetmacro{\XUORIGY}{.5}
  % Curve on x(U) initial point
  \pgfmathsetmacro{\XUCURVEINITX}{.8}
  \pgfmathsetmacro{\XUCURVEINITY}{2.1}
  \pgfmathsetmacro{\XUCURVEENDX}{3.2}
  \pgfmathsetmacro{\XUCURVEENDY}{.5}
  % Background
  \pgfmathsetmacro{\BGTOP}{2.0}
  \pgfmathsetmacro{\BGBOTTOM}{-4.8}
  \pgfmathsetmacro{\BGRIGHT}{4.0}
  \pgfmathsetmacro{\BGLEFT}{-4.0}
  % 
  \centering
  \begin{tikzpicture}[%
    scale=\scl,
    manifold/.style={%
      fill=black!10,draw=black!10,
    },
    chart/.style={%
      fill=black!10,draw=black!10,
    },
    open set/.style={%
      fill=green!17,draw=black,dotted,
    },
    origin/.style={%
      fill=black, draw=black,
    },
    axis/.style={%
      ->,
    },
    xU/.style={%
      rounded corners=8pt,
      dotted,
      fill=green!17, draw=black,
    },
    arrow/.style={%
      -{Latex[round]},shorten >=1pt, shorten <=1pt,draw=black!40,
    },
    background/.style={%
      line width=\bgborderwidth,
      draw=\bgbordercolor,
      fill=\bgcolor,
    },
    ],
    % 
    % COORDINATES
    % Coordinate system
    \coordinate (O) at (\ORIGX,\ORIGY);
    \coordinate (X) at (\ORIGX+\XAXISLENGTH,\ORIGY);
    \coordinate (Y) at (\ORIGX,\ORIGY+\YAXISLENGTH);
    % X(U) rectangle
    \coordinate (xUbl) at ($(O) + (\XUORIGX,\XUORIGY)$);
    \coordinate (xUtr) at ($(xUbl) + (\XUWIDTH,\XUHEIGHT)$);
    \filldraw[red] (1.0,.55) circle[radius=.2pt];
    \filldraw[red] (1.86,.91) circle[radius=.2pt];
    % gamma curve initial and end points
    \coordinate (UcurveInit) at (1.0,.55);
    \coordinate (UcurveEnd) at (1.86,.91);
    % X(gamma) curve initial and end points
    \coordinate (xUcurveInit) at ($(O) + (\XUCURVEINITX,\XUCURVEINITY)$);
    \coordinate (xUcurveEnd) at ($(O) + (\XUCURVEENDX,\XUCURVEENDY)$);
    % Background coordinates
    \coordinate (bgtop) at (0,\BGTOP);
    \coordinate (bgleft) at (\BGLEFT,0);
    \coordinate (bgright) at (\BGRIGHT,0);
    \coordinate (bgbottom) at (0,\BGBOTTOM);
    % 
    % DRAWING
    % Manifold
    \draw[manifold] plot [smooth cycle] coordinates{%
      (.55,-.05) (.4,.5) (.9,1.2) (1.6,1.5) (2.6,1.4) (2.9,.8) (2.3,-.1) (1.3,-.4)
    };
    \begin{scope}
      % U filled
      \filldraw[open set] plot [smooth cycle]
      coordinates{%
        (1.0,.5) (1.05,.8) (1.2,1.1) (1.5,1.15) (1.85,1.0) (1.8,.6) (1.5,.35) (1.2,.3)
      };
      % Clipping U
      \clip plot [smooth cycle]
      coordinates{%
        (1.0,.5) (1.05,.8) (1.2,1.1) (1.5,1.15) (1.85,1.0) (1.8,.6) (1.5,.35) (1.2,.3)
      };
      % Curve with arrow head inside
      % \draw[sloped,allow upside down] plot [smooth] coordinates{%
      % (.6,.5) (.8,.6) (1.0,.55) (1.4,.61) (1.75,.85) (2.0,.95) (2.2,.9)
      % }[arrow inside={end=Straight Barb,opt={black,line width=.5pt,scale=1.15}}{.63}];
      \draw[sloped,allow upside down] plot [smooth] coordinates{%
        (.6,.5) (.8,.6) (1.0,.55) (1.4,.61) (1.75,.85) (2.0,.95) (2.2,.9)
      }[arrow inside={end=Straight Barb,opt={black,line width=.5pt,scale=.95}}{.63}];
    \end{scope}
    % Manifold border
    \draw[dotted] plot [smooth cycle] coordinates{%
      (.55,-.05) (.4,.5) (.9,1.2) (1.6,1.5) (2.6,1.4) (2.9,.8) (2.3,-.1) (1.3,-.4)
    };
    % U border
    \draw[dotted] plot [smooth cycle]
    coordinates{%
      (1.0,.5) (1.05,.8) (1.2,1.1) (1.5,1.15) (1.85,1.0) (1.8,.6) (1.5,.35) (1.2,.3)
    };
    % gamma text
    \node at (1.28,.8) {\scriptsize $\gamma$};
    % U text
    \node[\greentext] at (1.4,0) {\footnotesize $U$};
    % M text
    \node at (.3,1.0) {\small $M$};
    % Chart
    % Coordinate system
    \filldraw[origin] (O) circle[radius=\ORIGRADIUS pt];
    \draw[axis] (O) -- (X);
    \draw[axis] (O) -- (Y);
    % x(U)
    \filldraw[xU] (xUbl) rectangle (xUtr);
    % Curve with arrow head inside
    \begin{scope}[shift=(xUcurveInit)]
      \draw[sloped,allow upside down] plot[smooth] coordinates{%
        (0,-.01) (.5,.1) (1.0,.05) (1.5,-.15) (2.0,-.6) (2.23,-1.0) (2.35,-1.3) (2.4,-1.6)
      }[arrow inside={end=Straight Barb,opt={black,line width=.5pt,scale=1.15}}{.6}];
    \end{scope}
    % x o gamma text
    \node at (-1.15,-2.25) {\scriptsize $x\circ\gamma$};
    % x(U) text
    % \node[\greentext] at (1.70,-1.9) {\footnotesize $x(U)\subseteq\symbb{R}^d$};
    \node[\greentext] at (-.3,-1.2) {\footnotesize $x(U)\subseteq\symbb{R}^d$};
    % Arrows
    \draw[arrow] (UcurveInit) to[out=240,in=80] node[above] {\footnotesize $x$} (xUcurveInit);
    \draw[arrow] (UcurveEnd) to[bend left] (xUcurveEnd);
    % 
    % YELLOW BACKGROUND 
    % \filldraw[fill=red,draw=black] (bgtop) circle[radius=1pt];
    % \filldraw[fill=red,draw=black] (bgbottom) circle[radius=1pt];
    % \filldraw[fill=green,draw=black] (bgleft) circle[radius=1pt];
    % \filldraw[fill=green,draw=black] (bgright) circle[radius=1pt];
    % \filldraw[fill opacity=0,draw=black] (O) circle[radius=2pt];
    \begin{scope}[on background layer]
      \node [background, fit= (bgleft) (bgright) (bgtop) (bgbottom)] {};
    \end{scope}
  \end{tikzpicture}
  \caption{The curve $x\circ\gamma$ is the representation of the original curve $\gamma$ in a
    subset $U$ of the manifold $M$ into the chart $(U,x)$.}
  \label{fig:mf-curve-chart}
\end{figure}

This can also be represented formally as
%\vspace{-1ex}
\begin{center}
  \begin{tikzcd}[cramped]
    \symbb{R}\arrow[r,"\gamma" above] \arrow[dr,end anchor={[xshift=-.5em]},"x\circ\gamma" below left]
    & U\subseteq M
    \arrow[d,start anchor={[xshift=-1.2em]},end anchor={[xshift=-1.2em]},"x" right, pos=.35]\\
    & \hspace{.6em}x(U)\subseteq\symbb{R}^d
  \end{tikzcd}
\end{center}
where $x\circ y$ is a composite map.

So, instead of looking at the curve $\gamma$ in the real world, you could study the curve on the
chart $x\circ\gamma$, which goes from a real parameter into a part of $\symbb{R}^d$.
In undergraduate physics, this is called the trajectory of a particle in $\symbb{R}^3$, or
something
\vspace{-3ex}
\begin{center}
  \begin{equation}\label{eq:mf-chart-map-xgamma}
    \begin{tikzcd}[cramped]
      \symbb{R}\arrow[r,"x\circ\gamma" above] & x(U)\subseteq\symbb{R}^d
    \end{tikzcd}
  \end{equation}
\end{center}

This is only the chart representative of the curve $\gamma$ in the real world.
If we could prove that the trajectory $x\circ\gamma$ in the chart is continuous, then the real
trajectory $\gamma$ would also be continuous. This way we could define some property, like
continuity, of an object in the real world by judging the continuity of the image of the object
in a chart that represents it. This would be and advantage because we could use undergraduate
analysis in charts.

But we need to ensure that this works. I mean, the $x$ map is just a product of our imagination
or, even worse, superstition, so the chart may be ill-defined and we could not infer properties
of the real world from products of our imagination, like $x\circ\gamma$.
In the following diagram we have drawn in red these fantasy bits
%\vspace{-1ex}
\begin{center}
  \begin{tikzcd}[cramped]
    \symbb{R}
    \arrow[r,"\gamma" above]
    \arrow[dr,"x\circ\textcolor{black}{\gamma}" below left,red]
    & U \arrow[d,"x" right,pos=.40,red]\\
    & \textcolor{red}{x}(U)
  \end{tikzcd}
\end{center}

So we need to prove that this works \emph{for every chart map we could invent}.
Imagine two chart maps $(U,x)$ and $(U,y)$
\vspace{-5ex}
\begin{center}
  \begin{equation}\label{eq:mf-chart-maps-x-y-continuity}
    \begin{tikzcd}[cramped]
      & y(U)\\
      \symbb{R}
      \arrow[ur,"y\circ\gamma" above left]
      \arrow[r,"\gamma" above]
      \arrow[dr,"x\circ\gamma" below left]
      & U \arrow[u,"y" right,pos=.40] \arrow[d,"x" right,pos=.40]\\
      & x(U)
    \end{tikzcd}
  \end{equation}
\end{center}

We then need to ensure that the defined property does not change if we choose another chart map.
In particular, we have to prove that the same properties can be deduced from whatever map
$x$ or $y$ we could invent. Then, instead of going directly from $\symbb{R}$ to $x(U)$ through
$x\circ\gamma$ as in~(\ref{eq:mf-chart-maps-x-y-continuity}), we start from the $y\circ\gamma$,
then $y^{-1}$, and finally $x$. If we get the same result $x\circ\gamma$ as in the direct road,
then we would be sure that this method is chart independent
\vspace{-1ex}
\[
  x \circ y^{-1} \circ (y\circ\gamma)
  = x \circ (y^{-1}\circ y) \circ\gamma
  = x \circ I \circ \gamma
  = x\circ \gamma
\]
where $I$ is the identity map and the composition of continuous maps is continuous.
So this ensures us that this is chart independent.

But there is a danger in the $(U,x)$ chart map. We might start confusing what an object in the
real world is, as it happened to Einstein.

\subsubsection{Differentiability}
With a topology on the manifold and the standard topology in $\symbb{R}^d$, we could study
continuity of the curve, on both the chart map and on the manifold.
But there may be cases where we need more structure than that.

Unlike continuity, in order to define differentiability we need more structure than a topology.
We cannot define differentiability at the manifold level. It can be done only at the chart level.
Suppose we have $x\circ\gamma$, which is a map from $\symbb{R}$ into $\symbb{R}^d$
%\vspace{-1ex}
\[
  x\circ\gamma : \symbb{R} \longrightarrow \symbb{R}^d
\]
and we would like to define differentiability for the real world curve $\gamma$ in the manifold,
based on the differentiability of $x\circ\gamma$ in the chart.

In order to study differentiability we need to define addition and multiplication (we need to
substract and divide). It can be easily done for $x\circ\gamma$ in $\symbb{R}^d$ as we learned in
undergraduate real analysis.

Now we have a problem.
We know from the continuity definition of the curve in the real world, that the definition
in the manifold ought to be chart independent, but now this doesn't work with differentiability.
Continuity is preserved, but the chart transition map $y\circ x^{-1}$, see
equation~(\ref{eq:mf-chart-maps-x-y-differentiability}), does not necessarily keep
differentiability.
The chart transition map $y\circ x^{-1}$ can turn a nice smooth curve $x\circ\gamma$ into another
one $y\circ\gamma$ with an edge, which is continuous but there is no unique tangent at the edge.
\vspace{-3ex}
\begin{center}
  \begin{equation}\label{eq:mf-chart-maps-x-y-differentiability}
    \begin{tikzcd}[cramped]
      & y(U)\\
      \symbb{R}
      \arrow[ur,"y\kern.3pt\circ\gamma" above left]
      \arrow[r,"\gamma" above]
      \arrow[dr,"x\circ\gamma" below left]
      & U \arrow[u,"y" right,pos=.40] \arrow[d,"x" right,pos=.40]\\
      & x(U) \arrow[uu,out=20,in=340,"y\kern.8pt\circ x^{-1}" right]
    \end{tikzcd}
  \end{equation}
\end{center}

This is because $y\circ\gamma$ is not guaranteed to be differentiable because of the $x^{-1}$ term in
the chart transition map $y\circ x^{-1}$ which is only guaranteed to be continuous.
If we have a differentiable function, $x\circ\gamma$, and we compose it with a continuous function,
$y\circ x^{-1}$ the result is only guaranteed to be continuous, not differentiable.
See figure~\ref{fig:mf-curve-failed-differentiation}.

%This is because, when the curve $x\circ\gamma$ is differentiable, $y\circ\gamma$ is not guaranteed
%to be differentiable because the chart transition map $y\circ x^{-1}$ only guarantees
%continuity. The inverse of a differentiable map is not necessarily differentiable.

\begin{figure}[ht]
  \def\scl{1}
  % Open end radius
  \pgfmathsetmacro{\OPENENDRADIUS}{1}
  % Background
  \pgfmathsetmacro{\BGTOP}{1.8}
  \pgfmathsetmacro{\BGBOTTOM}{-1.0}
  \pgfmathsetmacro{\BGRIGHT}{4.3}
  \pgfmathsetmacro{\BGLEFT}{-.7}
  \centering
  \begin{tikzpicture}[%
    scale=\scl,
    open end circle/.style={%
      fill=\bgcolor,
      draw=black,
      ultra thin,
    },
    arrow/.style={%
      -{Latex[round]},shorten >=1pt, shorten <=1pt,draw=black!50,
    },
    background/.style={%
      line width=\bgborderwidth,
      draw=\bgbordercolor,
      fill=\bgcolor,
    },
    ]
    % COORDINATES
    \coordinate (A1) at (0,0);
    \coordinate (B1) at (.5,1);
    \coordinate (C1) at (1,1.5);
    \coordinate (A2) at (2,0);
    \coordinate (B2) at (2.5,1);
    \coordinate (C2) at (3.6,1.5);
    % Background coordinates
    \coordinate (bgtop) at (0,\BGTOP);
    \coordinate (bgleft) at (\BGLEFT,0);
    \coordinate (bgright) at (\BGRIGHT,0);
    \coordinate (bgbottom) at (0,\BGBOTTOM);
    % DRAWING
    %
    % First curve
    \draw (A1) to[out=0,in=260] (B1)
    node[sloped,below left=2pt and 2pt] {\footnotesize $x\circ\gamma$} to[out=80,in=260] (C1)
    [arrow inside={end=Straight Barb,opt={black,line width=.5pt,scale=.95}}{.5}];
    \filldraw[open end circle] (A1) circle[radius=\OPENENDRADIUS pt];
    % \filldraw[red] (B1) circle[radius=.5pt];
    \filldraw[open end circle] (C1) circle[radius=\OPENENDRADIUS pt];
    %
    % Second curve 
    \draw (A2) to[out=20,in=270] (B2)
    node[sloped,below left=2pt and 2pt] {\footnotesize $y\circ\gamma$}  to[out=-20,in=260] (C2)
    [arrow inside={end=Straight Barb,opt={black,line width=.5pt,scale=.95}}{.25}]
    [arrow inside={end=Straight Barb,opt={black,line width=.5pt,scale=.95}}{.75}];
    \filldraw[open end circle] (A2) circle[radius=\OPENENDRADIUS pt];
    \filldraw[red] (B2) circle[radius=1.2pt];
    \filldraw[black] (B2) circle[radius=.2pt];
    \filldraw[open end circle] (C2) circle[radius=\OPENENDRADIUS pt];
    %
    % Chart transition map
    \draw[arrow] ($(A1)+(0,-.4)$) -- node[below] {\footnotesize $y\circ x^{-1}$} ($(A2)+(0,-.4)$);
    % 
    % YELLOW BACKGROUND 
    % \filldraw[fill=red,draw=black] (bgtop) circle[radius=1pt];
    % \filldraw[fill=red,draw=black] (bgbottom) circle[radius=1pt];
    % \filldraw[fill=green,draw=black] (bgleft) circle[radius=1pt];
    % \filldraw[fill=green,draw=black] (bgright) circle[radius=1pt];
    % \filldraw[fill opacity=0,draw=black] (O) circle[radius=2pt];
    \begin{scope}[on background layer]
      \node [background, fit= (bgleft) (bgright) (bgtop) (bgbottom)] {};
    \end{scope}
  \end{tikzpicture}
%  \caption{Imagine the curve $x\circ\gamma\in x(U)$ on the chart left being smooth enough
%    (differentiable). The other curve $y\circ\gamma\in y(U)$ on the right, although continuous,
%    may be non-differentiable (note the red vertex).}
  \caption{The chart transition map $y\circ x^{-1}$ might turn a nice smooth differentiable curve
    $x\circ\gamma$, into another one $y\circ\gamma$, with an edge (see the red point) which is
    not.}
  \label{fig:mf-curve-failed-differentiation}
\end{figure}

So it seems that we cannot define differentiability on the manifold through chart maps. But if we
\emph{restrict chart maps to those whose inverse is differentiable}\footnotemark{} then the chart
transition map $y\circ x^{-1}$ would be the composition of two differentiable maps, and the result
would be differentiable.
So we get an artificially restricted atlas where all the chart transition maps are differentiable
and then we could define differenciability on the manifold.
\footnotetext{This is like taking a huge atlas and ripping off the pages that contain chart maps
  whose inverse is not differentiable.}

A maximal topological atlas is an atlas which contains every chart that can possibly contain
(lots of redundancy). Then ripping certain charts off give us information (we can send a
secret message, like \emph{I love you} or something like that).
Ripping out charts is information, so reducing a maximal atlas by ripping out certain charts that
make trouble for the notion we want to define (like differentiability) means providing information.
This additional information is called \emph{the prescription of the differentiable atlas}.

This process of restricting an atlas is a major step for understanding spacetime.





  
  % 1:15:50


% janr
%TORUS IN 3D SHADED
%\begin{figure}[ht]
%  \def\scl{1}
%  \begin{tikzpicture}
%    % \x runs over the  angles at which to draw the circles defining the torus
%    \foreach \x in {90,89,...,-90} {%
%      \pgfmathsetmacro\elrad{20*max(cos(\x),.1)}
%      \pgfmathsetmacro\ltint{.9*abs(\x-60)/180 - .1}
%      \pgfmathsetmacro\rtint{.9*(1-abs(\x+60)/180) - .1}
%      \definecolor{currentcolor}{rgb}{\ltint, \ltint, \ltint}
%      % Third draws the right-hand circle
%      \draw[color=currentcolor,fill=currentcolor]
%      (xyz polar cs:angle=\x,y radius=.75,x radius=1.5)
%      ellipse (\elrad pt and 20pt);
%      % This sets the colour correctly for the left-hand circle ...
%      \definecolor{currentcolor}{rgb}{\rtint, \rtint, \rtint}
%      % ... and draws it
%      \draw[color=currentcolor,fill=currentcolor]
%      (xyz polar cs:angle=180-\x,radius=.75,x radius=1.5)
%      ellipse (\elrad pt and 20pt);
%    % End of foreach statement
%    }
%  \end{tikzpicture}
%\end{figure}

%janr - 10:40
  

  

  








%%% Local Variables:
%%% coding: utf-8
%%% mode: latex
%%% TeX-engine: luatex
%%% TeX-master: "../spacetime.tex"
%%% End:

