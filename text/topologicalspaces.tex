% topologicalspaces.tex
%
% Copyright (C) 2020-2025 José A. Navarro Ramón <janr.devel@gmail.com>

\chapter{Topological spaces}

\section{Introduction}
At the coarsest level, spacetime is a set. It just consists of points which are
the elements of the set. However, this level is not enough to talk even about
the simplest notions that we would like to talk about in classical physics, as
the notion of continuity of maps.

Why would we want to talk about continuity of maps? Well, in classical physics
there is the idea of curves in which there are no jumps. Some particle is
running somewhere and we do not have the situation that, all of a sudden, there
is a jump and the trajectory of the particle continues abruptly in a different
place. We do not want that, so we need to require continuity of maps.
\begin{figure}[ht]
  \centering
  % Scale
  \def\scl{1}
  % SCALE FACTOR
  \pgfmathsetmacro{\SCALE}{.60}
  % 
  \begin{tikzpicture}[%
    scale=\scl,
    baseline,
    every node/.style={black,font=\large},
    box/.style={fill=green!50, draw=green!55!black},
    arrow/.style={%
      {Latex[bend]}-{Latex[bend]}, shorten >=4pt, shorten <=4pt, line width=2.5pt
    }, background/.style={%
      line width=\bgborderwidth,
      draw=\bgbordercolor,
      fill=\bgcolor,
    },
    ]
    % COORDINATES
    % Origin
    \coordinate (O) at (0,0);
    % First semicurve
    \coordinate (a) at (0,0);
    \coordinate (b) at (.6,.5);
    \coordinate (c) at (1,.3);
    \coordinate (d) at (3,.4);
    % Second semicurve
    \coordinate (a') at (3.3,-.5);
    \coordinate (b') at (4.2,-.2);
    \coordinate (c') at (5.2,-.1);
    \coordinate (d') at (5.8,.4);
    %
    % DRAWING
    % First semicurve
    \draw [ultra thick, black] (a) to [out=90,in=180] (b) to[out=0,in=150] (c)
    to[out=330,in=210] (d)
    [arrow inside={end=Straight Barb,opt={black,line width=1.2pt,scale=1.2}}{.6}];
    \filldraw [fill=white, draw=red] (d) circle[radius=2pt];
    % Second semicurve
    \draw[ultra thick, black] (a') to[out=30,in=180] (b') to[out=350,in=210] (c')
    to[out=30,in=230] (d')
    [arrow inside={end=Straight Barb,opt={black,line width=1.2pt,scale=1.2}}{.6}];
    \filldraw [fill=red,draw=red] (a') circle[radius=2pt];
    %
    % Yellow background
    \begin{scope}[on background layer]
      \coordinate (left) at (-.5,0);
      \coordinate (right) at (6.3,0);
      \coordinate (above) at (0,1.3);
      \coordinate (below) at (0,-1);
      \node [background, fit= (left) (right) (below) (above)] {};
    \end{scope}
  \end{tikzpicture}
  \caption{A particle is not allowed to jump all of a sudden from one point to another
    in classical physics.}
  \label{fig:topspac-nojumps}
\end{figure}

It turns out that a set is not enough structure to be able to talk about curves
being continuous or not on that set. You could, of course, imagine all kinds of
structures on a set that allows you to talk about continuity. For instance, you
could implement a \emph{distance measure} of some kind.
But we need to be very minimal and very economic in order to not introduce
undue asumptions.
So \emph{we are interested in the weakest structure that we can establish on a set}
which allows a good definition of continuity on a set. Mathematicians know the
weakest such structure, and it is called \emph{a topology}.
This is our motivation to study topology in these lectures.

\section{Topology}
\subsection{Definition}
Let $M$ be a set. A \emph{topology}\footnotemark{} $\symcal{O}$ is a subset of
the power set of M
\footnotetext{$\symcal{O}$ stands for \emph{open}. This terminology will be
introduced soon.}
\begin{equation}
  \symcal{O} \subseteq \symcal{P}(M)
\end{equation}
that satisfies three axioms
\begin{enumerate}
\item The empty set and $M$ must be always part of the collection
  \begin{equation}
    \emptyset , M \in \symcal{O}
  \end{equation}
 
\item For every two elements of the collection, their intersection is always in
  it
  \begin{equation}
    \forall\, U, V \in\symcal{O}  \Longrightarrow U \cap V \in \symcal{O}
  \end{equation}
  
\item Given a finite or infinite number of elements in the collection, their
  union is always in the collection
  \begin{equation}
    \forall\, U_\alpha \in \symcal{O}  \Longrightarrow \bigcup_{\alpha\in A}
    U_\alpha \in \symcal{O}
  \end{equation}
  where $\alpha \in A$ is an arbitrary index and $A$ is any set, finite,
  countable infinite or even uncountable. In particular, the set $A$ may be the
  whole real line, $\symbb{R}$.
\end{enumerate}

\subsection{Comments on the defition of topology}
\begin{itemize}
\item The power set $\symcal{P}(M)$ of a set $M$ is the set containing all
  subsets of $M$.
  So to get a topology of $M$ you start by choosing a certain subset of the
  power set of $M$, $\symcal{O} \subseteq\symcal{P}(M)$.
\item The empty set and $M$ must be elements of $\symcal{P}(M)$.
\item The intersection of \emph{any} two sets of $\symcal{O}$ is also in
  $\symcal{O}$.
  To check for this, you have to check for all $\symcal{O}$ pairs.
\item The last axiom is deceptively similar to the preceding one. But in this
  case we are not restricted to check for every pair of elements in
  $\symcal{O}$, but we can choose any subset of elements of $\symcal{O}$,
  whether finite, countable-infinite or even uncountable-infinite elements of
  the power set. In either case, the union of all these elements must also be
  an element of $\symcal{O}$.
\item So, why we cannot check for the intersection of an infinite number of
  elements in $\symcal{O}$ in the second axiom?
  \[
    \forall\, U_\alpha \in \symcal{O}  \Longrightarrow \bigcap_{\alpha\in A}
    U_\alpha \in \symcal{O}
  \]
  Well, the answer is that it would yield to undesirable results.
  Take, for example the standard topology in the real line $\symbb{R}$,
  containing the sequence of open intervals
  \[
    \symcal{O}\equiv\left\{-\frac{1}{n}, \frac{1}{n}\right\},
    \forall n \in \symbb{Z}^{+}
  \]
  If we take the infinite intersection of these sets
  \[
    \bigcap_{n=1}^{\infty}U_n
    = \bigcap_{n=1}^{\infty} \left(-\frac{1}{n}, \frac{1}{n}\right)
    = \Set{0}
  \]
  The result would be a single point, which is not an element of this topology.
\end{itemize}

\subsection{Examples}
\subsubsection{A first try}
Let's have a set $M$ that just consists of three elements,
$M\equiv\Set{1,2,3}$.
We write down the entire power set of $M$, just for reference. It has $2^3 = 8$
elements
\[
  \symcal{P}(M)
  \equiv
  \Set{\emptyset, \{1\}, \{2\}, \{3\}, \{1,2\}, \{1,3\}, \{2,3\}, \{1,2,3\}}
\]

We now choose an arbitrary subset, $\symcal{O}$ of this power set
\[
  \symcal{O} \equiv \Set{\emptyset, \{1\}, \{2\}, \{1,2,3\}}
\]

And we want to find if $\symcal{O} \subseteq \symcal{P}(M)$ is a topology of
$M$. Our subset choice obey the first two axioms, but it fails on the third:
\[
  \{1\}\cup\{2\} = \{1,2\}\notin\symcal{O}
\]

\subsubsection{Extreme examples of topology}
\begin{itemize}
\item The most simple example of topology of a set $M$ is called the
  \emph{chaotic topology}, and only contains the empty set and the set
  \[
    \symcal{O}_{\text{chaotic}} \equiv \Set{\emptyset, M}
  \]
  It is very easy to prove that this is a topology.
\item The other example of topology of a set $M$ is the whole power set of $M$.
  It is called the \emph{discrete topology}
  \[
    \symcal{O}_{\text{discrete}} \equiv \symcal{P}(M)
  \]
  It is also very easy to prove that this is a topology. The power set contains
  all subsets of $M$, including $\emptyset$ and $M$. The intersection and union
  of subsets of the power set is also a subset of $M$.
\end{itemize}
Pity that these two topologies are utterly useless. It is worth introducing
them as they are extreme cases (the topology with the minimum and maximum
number of elements).

\subsection{Standard topology}
This is a very important example that will reconcile our intuition aboout
continuity built from undergraduate analysis courses. The standard topology
will be used throughout these lectures.

Contrary to the chaotic and discrete topologies, which can be defined for any
set $M$, the standard topology can only be defined in $M = \symbb{R}^d$, which
is the set of all d-tuples of the form
\[
  M = \symbb{R}^d
  = \underbrace{\symbb{R}\times\symbb{R}\cdots\times\symbb{R}}_{\text{d times}}
  \equiv \Set{(p_1, \cdots, p_d) | p_i\in\symbb{R}}
\]

Then, a standard topology is a subset of the power set of $\symbb{R}^d$
\[
  \symcal{O}_{\text{std}} \subseteq \symcal{P}(\symbb{R}^d)
\]

The standard topology may contain non-countable many elements, so it cannot be
defined explicitly.

The definition proceeds in two steps:

\begin{description}
\item[$\alpha$. Soft ball definition:]\ 

  A soft ball is a set $B_r(\vvv{p})$ where $p\in\symbb{R}^d$ is a point and
  $r$ is an element of the positive reals, $r\in\symbb{R}^+$, is called the
  soft ball radius. You can think of $\vvv{p}$ as the center of the ball
  \begin{equation}\label{eq:tplg-softball}
    B_r(\vvv{p})
    = \Set{(q_1, \cdots, q_d) | \sum_{i=1}^d (q_i - p_i)^2 < r^2}
  \end{equation}
  
  We might argue that we have just written down the Euclidian norm. But in this
  case it is just a formula.
  To talk about the Euclidian norm we need a vector space structure and a dot
  product, which is not necessary the case.

\item[$\beta$. Standard topology definition:]\ 

  $\symcal{O}_{\text{std}}$ is a standard topology if for any
  $U\in\symcal{O}_{\text{std}}$ and for all $\vvv{p}\in U$, there exists at
  least one soft ball $B_r(\vvv{p})$ which is included in $U$
\begin{equation}
  \forall U\in\symcal{O}_{\text{std}}, \forall \vvv{p}\in U
  \Longrightarrow \exists r\in\symbb{R}^+ | B_r(\vvv{p})\subseteq U
\end{equation}
\end{description}

\subsubsection{Comments}
\begin{itemize}
\item A soft ball, $B_r(\vvv{p})$ in $\symbb{R}^n$ is the multidimensional
  equivalent of an open interval in $\symbb{R}$. This is because the
  \emph{less than} relational operator in the
  definition~\eqref{eq:tplg-softball}. Open intervals in $\symbb{R}$ do not
  include their endpoints, whereas closed intervals include them. Accordingly,
  a soft ball consists only of \emph{inner points} and does not include the
  points in its surface, although we con get as near as we like to it.
  In figure~\eqref{fig:topspac-softball-closedball} we represent both, a soft
  ball example and a collection of points which is not considered a soft ball
  in a $\symbb{R}^2$ plane.
  \begin{figure}[ht]
    \def\scl{1}
    % 
    \pgfmathsetmacro{\FACTOR}{0.6}
    % Radius of the soft ball
    \pgfmathsetmacro{\RADIUS}{2 * \FACTOR}
    % Center of soft ball
    \pgfmathsetmacro{\XSB}{0 * \FACTOR}
    \pgfmathsetmacro{\YSB}{0 * \FACTOR}
    % Center of closed ball
    \pgfmathsetmacro{\XCB}{4.5 * \RADIUS * \FACTOR}
    \pgfmathsetmacro{\YCB}{0 * \FACTOR}
    % R2
    \pgfmathsetmacro{\XRTWO}{7.7 * \FACTOR}
    \pgfmathsetmacro{\YRTWO}{3.3 * \FACTOR}
    % Background
    \pgfmathsetmacro{\XCENTER}{(\XSB + \XCB)/2.0}
    \pgfmathsetmacro{\YCENTER}{0 * \FACTOR} 
    \pgfmathsetmacro{\XSPACE}{4 * \FACTOR}
    \pgfmathsetmacro{\YSPACE}{2 * \FACTOR}
    \pgfmathsetmacro{\XBOTTOMLEFT}{\XCENTER - (\XSPACE + \RADIUS)}
    \pgfmathsetmacro{\YBOTTOMLEFT}{(\YCENTER - (\YSPACE + \RADIUS)}
    \pgfmathsetmacro{\XTOPRIGHT}{\XCENTER + (\XSPACE + \RADIUS)}
    \pgfmathsetmacro{\YTOPRIGHT}{\YCENTER + (\YSPACE + \RADIUS)}
    % 
    \centering
    \begin{tikzpicture}[%
      open set/.style={%
        fill=green!50, draw=green!70!black, dashed, closed, line width=.8pt
      }, closed set/.style={%
        fill=green!50, draw=green!40!black, solid, closed, line width=.8pt
      }, point/.style={%
        fill=black!50, draw=black, radius=.8pt
      }, arrow/.style={%
        -{Latex}, ultra thin, black!70
      }, background/.style={%
        line width=\bgborderwidth,
        draw=\bgbordercolor,
        fill=\bgcolor,
      },
      ]
      % COORDINATES
      \coordinate (O1) at (\XSB,\YSB);
      \coordinate (O2) at (\XCB,\YCB);
      \coordinate (BOTTOMLEFT) at (\XBOTTOMLEFT,\YBOTTOMLEFT);
      \coordinate (TOPRIGHT) at (\XTOPRIGHT, \YTOPRIGHT);
      % Text
      \coordinate (R2) at (\XRTWO, \YRTWO);
      % 
      % DRAWING
      \filldraw[open set] (O1) circle [radius=\RADIUS];
      \draw[arrow] (O1) -- node[above right, black]
      {\footnotesize $r$} (135:\RADIUS);
      \filldraw[point] (O1) circle;
      \node[below right] at (O1) {\footnotesize $\vvv{p}$};
      \filldraw[closed set] (O2) circle [radius=\RADIUS];
      \filldraw[point] (O2) circle;
      %
      \node at (R2) {\large $\symbb{R}^2$};
      % YELLOW BACKGROUND
      \begin{scope}[on background layer]
        \node [background, fit= (BOTTOMLEFT) (TOPRIGHT)] {};
      \end{scope}
    \end{tikzpicture}
    \caption{%
      To the left a soft ball $B_r(\vvv{p})$ is represented.
      The one on the right is not a soft ball.}
    \label{fig:topspac-softball-closedball}
  \end{figure}

\item It's a simple exercise to prove that this two step definition of the
  standard topology satisfies the three axioms of a topology. We leave this
  task to the reader.
  
\item We can think of a standard topology $\theta_{\text{std}}$ in a very
  simple and graphical way.
  Let $U$ be any set in the topology, $U\subseteq\theta_{\text{std}}$. It can
  be proved that this leads to the fact that $U$ has no border, much like a
  soft ball. For this reason, any set belonging to the standard topology is
  also called as an \emph{open set}.
  We will define more properly an open set very soon.

  On the left side of picture~\ref{fig:topspac-openset}, a possible set of a
  standard topology is shown. We can see its border as a non-continuous line,
  meaning that all its elements are \emph{inner points}.
  
  According to the definition of standard topology, \emph{for every point} $p$
  in any $U\in\theta_{\text{std}}$ we can find at least one soft ball centered
  in this point that it is included in $U$.
  On the right side of the same figure an arbitrary set $U$ of this topology
  and two arbitrary points $p_1$ and $p_2$ are shown. $U$ is an element of
  $\theta_{\text{std}}$, because we can find a soft ball of small enough radius
  centered on any point of the set that is completely inside $U$. If the point
  is extremely near the border, the radius of the soft ball must be also
  extremely small.
  \begin{figure}[ht]
    \def\scl{1}
    % 
    \pgfmathsetmacro{\FACTOR}{.60}
    % A
    \pgfmathsetmacro{\XA}{0 * \FACTOR}
    \pgfmathsetmacro{\YA}{0 * \FACTOR}
    % B
    \pgfmathsetmacro{\XB}{3.0 * \FACTOR}
    \pgfmathsetmacro{\YB}{2.0 * \FACTOR}
    % C
    \pgfmathsetmacro{\XC}{2.0 * \FACTOR}
    \pgfmathsetmacro{\YC}{3.2 * \FACTOR}
    % D
    \pgfmathsetmacro{\XD}{-.3 * \FACTOR}
    \pgfmathsetmacro{\YD}{2.6 * \FACTOR}
    % E
    \pgfmathsetmacro{\XE}{.1 * \FACTOR}
    \pgfmathsetmacro{\YE}{2.2 * \FACTOR}
    % CLIPCENTER
    \pgfmathsetmacro{\XCLIPCENTER}{0 * \FACTOR}
    \pgfmathsetmacro{\YCLIPCENTER}{2.5 * \FACTOR}
    \pgfmathsetmacro{\RADIUS}{1.0 * \FACTOR}
    % R2
    \pgfmathsetmacro{\XRTWO}{5.0 * \FACTOR}
    \pgfmathsetmacro{\YRTWO}{3.5 * \FACTOR}
    % U text
    \pgfmathsetmacro{\XU}{4.0 * \FACTOR}
    \pgfmathsetmacro{\YU}{-1.0 * \FACTOR}
    % Point
    \pgfmathsetmacro{\XPOINTONE}{2.0 * \FACTOR}
    \pgfmathsetmacro{\YPOINTONE}{1.6 * \FACTOR}
    \pgfmathsetmacro{\RADIUSONE}{8}
    \pgfmathsetmacro{\XPOINTTWO}{.23 * \FACTOR}
    \pgfmathsetmacro{\YPOINTTWO}{0.0 * \FACTOR}
    \pgfmathsetmacro{\RADIUSTWO}{3}
    % Points text
    \pgfmathsetmacro{\XPTEXTONE}{2.6 * \FACTOR}
    \pgfmathsetmacro{\YPTEXTONE}{1.0 * \FACTOR}
    \pgfmathsetmacro{\XPTEXTTWO}{0.6 * \FACTOR}
    \pgfmathsetmacro{\YPTEXTTWO}{0.4 * \FACTOR}
    % Background
    \pgfmathsetmacro{\XBOTTOMLEFT}{-2.0 * \FACTOR}
    \pgfmathsetmacro{\YBOTTOMLEFT}{-1.8 * \FACTOR}
    \pgfmathsetmacro{\XTOPRIGHT}{6.0 * \FACTOR}
    \pgfmathsetmacro{\YTOPRIGHT}{4.2 * \FACTOR}
    %
    \begin{minipage}{.45\linewidth}
      %
      \hspace{2.4em}
      %
      \begin{tikzpicture}[%
        scale = \scl,
        use Hobby shortcut,
        open set/.style={%
          fill=green!50, draw=green!70!black, dashed, closed, line width=.8pt
        }, closed set/.style={%
          fill=green!50, draw=green!40!black, solid, closed, line width=.8pt
        }, background/.style={%
          line width=\bgborderwidth,
          draw=\bgbordercolor,
          fill=\bgcolor,
        },
        ]
        % COORDINATES
        \coordinate (a) at (\XA, \YA);
        \coordinate (b) at (\XB, \YB);
        \coordinate (c) at (\XC, \YC);
        \coordinate (d) at (\XD, \YD);
        \coordinate (e) at (\XE, \YE);
        % Text
        \coordinate (R2) at (\XRTWO, \YRTWO);
        \coordinate (U) at (\XU, \YU);
        % Clip center
        \coordinate (clipCENTER) at (\XCLIPCENTER, \YCLIPCENTER);
        % Background
        \coordinate (BOTTOMLEFT) at (\XBOTTOMLEFT, \YBOTTOMLEFT);
        \coordinate (TOPRIGHT) at (\XTOPRIGHT, \YTOPRIGHT);
        % 
        % DRAWING
        % Open set
        % \draw[green] (clipCENTER) circle[radius=\RADIUS];
        % \filldraw[red] (clipCENTER) circle[radius=1pt];
        \draw[open set] (a) .. (b) .. (c) .. (d) .. (e);
        %\clip (clipCENTER) circle[radius=\RADIUS];
        %\draw[closed set] (a) .. (b) .. (c) .. (d) .. (e);
        % Text
        \node at (R2) {\large $\symbb{R}^2$};
        \node at (U) {\small $U\in\symcal{O}_{\text{std}}$};
        % 
        % YELLOW BACKGROUND
        \begin{scope}[on background layer]
          \node [background, fit= (BOTTOMLEFT) (TOPRIGHT)] {};
        \end{scope}
      \end{tikzpicture}
    \end{minipage}
    \hspace{1.9em}
    \begin{minipage}{.45\linewidth}
      %
      \hspace{1.8em}
      %
      \begin{tikzpicture}[%
        scale = \scl,
        use Hobby shortcut,
        open set/.style={%
          fill=green!50, draw=green!70!black, dashed, closed, line width=.8pt
        }, closed set/.style={%
          fill=green!50, draw=green!40!black, solid, closed, line width=.8pt
        }, point/.style={fill=black!50, draw=black, radius=.3pt
        }, soft ball/.style={%
          fill=green!80!black, draw=black, densely dotted, line width=.5pt
        }, background/.style={%
          line width=\bgborderwidth,
          draw=\bgbordercolor,
          fill=\bgcolor,
        },
        ]
        % COORDINATES
        % Points
        \coordinate (a) at (\XA, \YA);
        \coordinate (b) at (\XB, \YB);
        \coordinate (c) at (\XC, \YC);
        \coordinate (d) at (\XD, \YD);
        \coordinate (e) at (\XE, \YE);
        % Text
        \coordinate (R2) at (\XRTWO, \YRTWO);
        \coordinate (U) at (\XU, \YU);
        \coordinate (p1text) at (\XPTEXTONE, \YPTEXTONE);
        \coordinate (p2text) at (\XPTEXTTWO, \YPTEXTTWO);
        % Points
        \coordinate (p1) at (\XPOINTONE, \YPOINTONE);
        \coordinate (p2) at (\XPOINTTWO, \YPOINTTWO);
        % Clip center
        \coordinate (clipCENTER) at (\XCLIPCENTER, \YCLIPCENTER);
        % Background
        \coordinate (BOTTOMLEFT) at (\XBOTTOMLEFT, \YBOTTOMLEFT);
        \coordinate (TOPRIGHT) at (\XTOPRIGHT, \YTOPRIGHT);
        % 
        % DRAWING
        % Open set
        \draw[open set] (a) .. (b) .. (c) .. (d) .. (e);
        % \draw[green] (clipCENTER) circle[radius=\RADIUS];
        % \filldraw[red] (clipCENTER) circle[radius=1pt];
        %\clip (clipCENTER) circle[radius=\RADIUS];
        %\draw[closed set] (a) .. (b) .. (c) .. (d) .. (e);
        % Point 1
        \filldraw[soft ball] (p1) circle [radius=\RADIUSONE pt];
        \filldraw[point] (p1) circle;
        % Point 2
        \filldraw[soft ball] (p2) circle [radius=\RADIUSTWO pt];
        \filldraw[point] (p2) circle;
        % Text
        \node at (R2) {\large $\symbb{R}^2$};
        \node at (U) {\small $U\in\symcal{O}_{\text{std}}$};
        \node at (p1text) {\footnotesize $p_1$};
        \node at (p2text) {\footnotesize $p_2$};
        % 
        % YELLOW BACKGROUND
        \begin{scope}[on background layer]
          \node [background, fit= (BOTTOMLEFT) (TOPRIGHT)] {};
        \end{scope}
      \end{tikzpicture}
    \end{minipage}
    \caption{On the left side, a possible set of the standard topology is
      represented.
      We see that the boundary doesn't belong to the set (roughly speaking).
      On the right side we suggest that for any point of $U$, we could find a
      radius small enough such that the soft ball $B_r(\vvv{p})$ lies entirely
      in the set $U$.}
    \label{fig:topspac-openset}
  \end{figure}

  In contrast, on the left side of
  figure~\ref{fig:topspac-nonopen-nonclosed-set} we represent a set $V$ that is
  not an element of the standard topology. This is because there are some
  points in its border, marked with a continuous line, that are elements of
  $V$. To the right of the figure we see that any soft ball
  ---no matter how small--- around the points in the border are not completely
  inside $V$.
  
  \begin{figure}[ht]
    \def\scl{1}
    %
    \pgfmathsetmacro{\FACTOR}{.60}
    % A
    \pgfmathsetmacro{\XA}{0 * \FACTOR}
    \pgfmathsetmacro{\YA}{0 * \FACTOR}
    % B
    \pgfmathsetmacro{\XB}{3.0 * \FACTOR}
    \pgfmathsetmacro{\YB}{2.0 * \FACTOR}
    % C
    \pgfmathsetmacro{\XC}{2.0 * \FACTOR}
    \pgfmathsetmacro{\YC}{3.2 * \FACTOR}
    % D
    \pgfmathsetmacro{\XD}{-.3 * \FACTOR}
    \pgfmathsetmacro{\YD}{2.6 * \FACTOR}
    % E
    \pgfmathsetmacro{\XE}{.1 * \FACTOR}
    \pgfmathsetmacro{\YE}{2.2 * \FACTOR}
    % CLIPCENTER
    \pgfmathsetmacro{\XCLIPCENTER}{0 * \FACTOR}
    \pgfmathsetmacro{\YCLIPCENTER}{2.5 * \FACTOR}
    \pgfmathsetmacro{\RADIUS}{1.0 * \FACTOR}
    % R2
    \pgfmathsetmacro{\XRTWO}{5.0 * \FACTOR}
    \pgfmathsetmacro{\YRTWO}{3.5 * \FACTOR}
    % V text
    \pgfmathsetmacro{\XV}{4.0 * \FACTOR}
    \pgfmathsetmacro{\YV}{-1.0 * \FACTOR}
    % Point
    \pgfmathsetmacro{\XPOINT}{2.0 * \FACTOR}
    \pgfmathsetmacro{\YPOINT}{1.6 * \FACTOR}
    \pgfmathsetmacro{\SOFTBALLRADIUS}{4}
    % BACKGROUND
    \pgfmathsetmacro{\XBOTTOMLEFT}{-2.0 * \FACTOR}
    \pgfmathsetmacro{\YBOTTOMLEFT}{-1.8 * \FACTOR}
    \pgfmathsetmacro{\XTOPRIGHT}{6.0 * \FACTOR}
    \pgfmathsetmacro{\YTOPRIGHT}{4.2 * \FACTOR}
    %
    \begin{minipage}{.45\linewidth}
      %
      \hspace{4.4em}
      %
      \begin{tikzpicture}[%
        scale = \scl,
        use Hobby shortcut,
        open set/.style={%
          fill=green!50, draw=green!70!black, dashed, closed, line width=.8pt
        }, closed set/.style={%
          fill=green!50, draw=green!40!black, solid, closed, line width=.8pt
        }, background/.style={%
          line width=\bgborderwidth,
          draw=\bgbordercolor,
          fill=\bgcolor,
        },      
        ]
        % COORDINATES
        \coordinate (a) at (\XA, \YA);
        \coordinate (b) at (\XB, \YB);
        \coordinate (c) at (\XC, \YC);
        \coordinate (d) at (\XD, \YD);
        \coordinate (e) at (\XE, \YE);
        % Text
        \coordinate (R2) at (\XRTWO, \YRTWO);
        \coordinate (V) at (\XV, \YV);
        % Clip center
        \coordinate (clipCENTER) at (\XCLIPCENTER, \YCLIPCENTER);
        % Background
        \coordinate (BOTTOMLEFT) at (\XBOTTOMLEFT, \YBOTTOMLEFT);
        \coordinate (TOPRIGHT) at (\XTOPRIGHT, \YTOPRIGHT);
        % 
        % DRAWING
        % Text
        \node at (R2) {\large $\symbb{R}^2$};
        %\node at (U) {$V$};
        \node at (V) {\small $V\notin\symcal{O}_{\text{std}}$};
        % Set
        % \draw[green] (clipCENTER) circle[radius=\RADIUS];
        % \filldraw[red] (clipCENTER) circle[radius=1pt];
        \draw[open set] (a) .. (b) .. (c) .. (d) .. (e);
        \clip (clipCENTER) circle[radius=\RADIUS];
        \draw[closed set] (a) .. (b) .. (c) .. (d) .. (e);
        % 
        % YELLOW BACKGROUND
        \begin{scope}[on background layer]
          \node [background, fit= (BOTTOMLEFT) (TOPRIGHT)] {};
        \end{scope}
      \end{tikzpicture}
    \end{minipage}
    \hspace{2em}
    \begin{minipage}{.45\linewidth}
      %
      \hspace{1.8em}
      %
      \begin{tikzpicture}[%
        scale = \scl,
        use Hobby shortcut,
        open set/.style={%
          fill=green!50, draw=green!70!black, dashed, closed, line width=.4pt
        }, closed set/.style={%
          fill=green!50, draw=green!40!black, solid, closed, line width=.4pt
        }, point/.style={fill=black!50, draw=black, radius=.3pt
        }, inner/.style={fill=green!80!black, draw=black, densely dotted
        }, outer/.style={fill=red,draw=black,densely dotted
        }, background/.style={%
          line width=\bgborderwidth,
          draw=\bgbordercolor,
          fill=\bgcolor,
        },
        ]
        % COORDINATES
        % Points
        \coordinate (a) at (\XA, \YA);
        \coordinate (b) at (\XB, \YB);
        \coordinate (c) at (\XC, \YC);
        \coordinate (d) at (\XD, \YD);
        \coordinate (e) at (\XE, \YE);
        % Text
        \coordinate (R2) at (\XRTWO, \YRTWO);
        \coordinate (V) at (\XV, \YV);
        % Clip center
        \coordinate (clipCENTER) at (\XCLIPCENTER, \YCLIPCENTER);
        % Point
        \coordinate (p) at (\XPOINT, \YPOINT);
        % Background
        \coordinate (BOTTOMLEFT) at (\XBOTTOMLEFT, \YBOTTOMLEFT);
        \coordinate (TOPRIGHT) at (\XTOPRIGHT, \YTOPRIGHT);
        % 
        % DRAWING
        % Open set zone
        \draw[open set] (a) .. (b) .. (c) .. (d) .. (e);
        % Text
        \node at (R2) {\large $\symbb{R}^2$};
        \node at (V) {\small $V\notin\symcal{O}_{\text{std}}$};
        % Outer points of soft ball
        %\filldraw[outer] (-.2*\FACTOR,2.6*\FACTOR) circle [radius=4pt];
        \filldraw[outer] (d) circle [radius=4pt];
        \begin{scope}
          % Clip the set and draw a continuous line in the border
          \clip(clipCENTER) circle[radius=\RADIUS];
          \draw[closed set] (a) .. (b) .. (c) .. (d) .. (e);
        \end{scope}
        \begin{scope}
        \clip[closed] (a) .. (b) .. (c) .. (d) .. (e);
        % Inner points of soft ball
        %\filldraw[inner] (-.2*\FACTOR,2.6*\FACTOR) circle [radius=4pt];
        \filldraw[inner] (d) circle [radius=4pt];
        \end{scope}
        % Soft ball center
        \filldraw[point] (d) circle;
        %
        % YELLOW BACKGROUND
        \begin{scope}[on background layer]
          \node [background, fit= (BOTTOMLEFT) (TOPRIGHT)] {};
        \end{scope}
      \end{tikzpicture}
    \end{minipage}  
    \caption{On the left side of the picture we outline a set $V$ which does
      not belong to the standard topology, since some points of its border are
      elements of $V$ (the ones in the solid line).
      On the right we show a soft ball centered on one point of this border,
      revealing that some parts of the soft ball, the ones in red, are not in
      $V$.}
    \label{fig:topspac-nonopen-nonclosed-set}
  \end{figure}

\item Probably, when you see the pictures in figures
  \ref{fig:topspac-nonopen-nonclosed-set} and
  \ref{fig:topspac-nonopen-nonclosed-set}
  you may recognize an open set in the first, and a non-open set in the last
  one from, perhaps the open intervals in $\symbb{R}$, as $(a,b)$.
  This is precisely the reason for defining the standard topology, this way
  we use the properties on $\symbb{R}$ that we know by heart.


\item Although the standard topology is very important, it doesn't mean that it
  is the only topology.
  There might be opportunities to equip $M=\symbb{R}^d$ with a different
  topology if you have other than the standard purposes in mind.
  Its name is somewhat deceptive, but there is nothing special about the
  standard topology, other than you like it because you met it at the beginning
  of your mathematical studies. But there is no reason for spacetime to carry
  it. Furthermore, the standard topology can only be defined in some sets of
  the form $\symbb{R}^d$.

\end{itemize}


\subsection{Open and closed sets}
Later on, we are going to think of $M = \symbb{R}^d$ as the points of
spacetime, and we will ask ourselves what topology should the physical
spacetime carry.

Remember that we define topology in order to talk about continuity of maps, so
please, do not be deterred if you find that this is a little abstract. It is
necessary to make explicit some notions that we may take for granted.

When Einstein wrote down general relativity, he didn't understand many explicit
assumptions, and it took him many years to overcome intuitive notions
(some would say obvious notions) that nevertheless didn't apply, making
implicit things explicit to really understand the concepts needed for general
relativity.

\subsubsection{Some terminology}
We've talked aobut $M$ being a \emph{set}, meaning a \emph{collection} of
\emph{elements}. But to really understand these concepts (set and collection),
we need to postulate some axioms, although in this text we are not going that
far, and we will be satisfied with the intuitive meaning.

\begin{itemize}
  \item $M$ is some \emph{set} that we need to study.
  \item $\symcal{O}$ is a \emph{topology}, which is some set of subsets of $M$.
    These subsets are also called \emph{open sets}. So a topology $\symcal{O}$
    is a set of open sets. Note that the term \emph{open} is defined only by
    choosing a certain topology (if we would choose a different one, the open
    sets would be different).
  \item The pair $(M, \symcal{O})$ is called a \emph{topological space}.
    It is more than a set. It is a set with more information about it.
  \item If $U$ is a subset of $M$ that lies in $\symcal{O}$, $U\in\symcal{O}$,
    then we call $U$ an \emph{open set}.
  \item If $A$ is as subset of $M$, it is called a \emph{closed set} if the
    complement of $A$ lies on the topology
    \[
      A\subseteq M \text{ is a closed set} \iff  M\backslash A\in\symcal{O}
    \]
  \item The \emph{open/closed} terminology is somewhat deceptive:
    Some would say that \emph{open is the opposite of closed in some way}.

    Well no, not in any way. There are open sets that are closed at the same
    time (think of the empty set $\emptyset$).
    There are sets that are open, but not closed (see
    figure~\ref{fig:topspac-openset}).
    There are sets that are closed, but not open (Set on the right of
    figure~\ref{fig:topspac-softball-closedball}).
    And there are sets that are not open and not closed (look at the set
    pictured in figure~\ref{fig:topspac-nonopen-nonclosed-set}).

    But of course, when we choose a certain topology these two terms
    \emph{open} and \emph{closed} are connected in this way
    \begin{align*}
      & \text{If } U\in\symcal{O}
        \iff\hspace{-0.4em}: \text{ call } U\subseteq M
        \text{ an \emph{open set}}\\
      & \text{If } M\backslash A\in\symcal{O}
        \iff\hspace{-0.4em} : \text{ call } A\subseteq M
        \text{ a \emph{closed set}}
    \end{align*}
    
  \end{itemize}
  
  \section{Maps}
  Topology yields to the notion of continuous maps.
  \subsection{Types of maps}
  A map $f$ connects every element of a set $M$, called \emph{domain}, to some
  element of the set $N$, called \emph{target}
  \[
    f: M \longrightarrow N
  \]

  In figure~\ref{fig:topspac-map-example} we picture two sets
  $M=\set{a, b, c, d}$ and $N=\set{i, j, k, l, m}$ by the collection of their
  elements (as points), and consider an arbitrary map $f$ which takes every
  point in the domain to some point in the target.
  
  If every point in the target set is the image of one element at most, then
  the map is said to be \emph{injective} or \emph{one to one}. In our example
  we can see that the target point $i$ is being hit twice, so this map is
  \emph{non-injective}.

  If every point in the target is the image of one or more elements in the
  domain, the map is called \emph{surjective}. But there is no need that every
  point is hit, in our example, points $j$ and $l$ are not hit, so the map is
  \emph{non-surjective}.

  A map is called \emph{bijective} if it is injective and surjective.
  \begin{figure}[ht]
    \def\scl{1}
    % 
    \pgfmathsetmacro{\FACTOR}{1.0}
    % 
    % M set
    \pgfmathsetmacro{\XMSET}{0.0 * \FACTOR}
    \pgfmathsetmacro{\YMSET}{0.0 * \FACTOR}
    % M text
    \pgfmathsetmacro{\XMTEXT}{0.8 * \FACTOR}
    \pgfmathsetmacro{\YMTEXT}{4.6 * \FACTOR}
    % Domain text
    \pgfmathsetmacro{\XDOMAIN}{1.7 * \FACTOR}
    \pgfmathsetmacro{\YDOMAIN}{1.0 * \FACTOR}
    % Target text
    \pgfmathsetmacro{\XTARGET}{6.3 * \FACTOR}
    \pgfmathsetmacro{\YTARGET}{1.0 * \FACTOR}
    % a point
    \pgfmathsetmacro{\XA}{2.0 * \FACTOR}
    \pgfmathsetmacro{\YA}{4.0 * \FACTOR}
    % b point
    \pgfmathsetmacro{\XB}{1.7 * \FACTOR}
    \pgfmathsetmacro{\YB}{3.5 * \FACTOR}
    % c point
    \pgfmathsetmacro{\XC}{2.2 * \FACTOR}
    \pgfmathsetmacro{\YC}{2.8 * \FACTOR}
    % d point
    \pgfmathsetmacro{\XD}{1.5 * \FACTOR}
    \pgfmathsetmacro{\YD}{2.2 * \FACTOR}    
    % N set
    \pgfmathsetmacro{\XNSET}{4.0 * \FACTOR}
    \pgfmathsetmacro{\YNSET}{0.0 * \FACTOR}
    % N text
    \pgfmathsetmacro{\XNTEXT}{7.1 * \FACTOR}
    \pgfmathsetmacro{\YNTEXT}{4.6 * \FACTOR}    
    % i point
    \pgfmathsetmacro{\XI}{6.5 * \FACTOR}
    \pgfmathsetmacro{\YI}{4.0 * \FACTOR}
    % j point
    \pgfmathsetmacro{\XJ}{6.8 * \FACTOR}
    \pgfmathsetmacro{\YJ}{3.5 * \FACTOR}    
    % k point
    \pgfmathsetmacro{\XK}{5.9 * \FACTOR}
    \pgfmathsetmacro{\YK}{3.3 * \FACTOR}
    % l point
    \pgfmathsetmacro{\XL}{6.3 * \FACTOR}
    \pgfmathsetmacro{\YL}{3.0 * \FACTOR}
    % m point
    \pgfmathsetmacro{\XM}{6.1 * \FACTOR}
    \pgfmathsetmacro{\YM}{2.2 * \FACTOR}
    % Background
    \pgfmathsetmacro{\XCENTER}{0.5 * (\XMTEXT + \XNTEXT)}
    \pgfmathsetmacro{\YCENTER}{0.5 * (\YMTEXT + \YDOMAIN)}
    \pgfmathsetmacro{\XBACKLEFT}{\XMTEXT - 1.0 * \FACTOR}
    \pgfmathsetmacro{\YBACKBOTTOM}{\YDOMAIN - 0.5 * \FACTOR}
    \pgfmathsetmacro{\XBACKRIGHT}{\XNTEXT + 1.0 * \FACTOR}
    \pgfmathsetmacro{\YBACKTOP}{\YMTEXT + 0.8 * \FACTOR}
    %
    \centering
    %
    \begin{tikzpicture}[%
      scale=\scl,
      every label/.append style={text=black!80, font=\footnotesize
      }, M/.style={%
        ellipse, draw,
        fill=green!40, draw=black,
        very thick,
        minimum height=8em, minimum width=6em,
        fit=(a) (b) (c) (d),
      },
      N/.style={%
        ellipse, draw,
        fill=green!50, draw=black,
        minimum height=8em, minimum width=6em,
        fit=(i) (j) (k) (l) (m),
      },
      point/.style={circle, fill, minimum width=2.5pt, inner sep=0pt
      }, arrow/.style={-{Latex}, shorten >=3pt, shorten <=4pt, line width=.8pt
      }, background/.style={
        line width=\bgborderwidth,
        draw=\bgbordercolor,
        fill=\bgcolor,
      },     
      ]
      % COORDINATES
      \coordinate (O) at (0, 0);
      % M and N sets
      \coordinate (M) at (\XMSET, \YMSET);
      \coordinate (N) at (\XNSET, \YNSET);
      % M and N texts
      \coordinate (Mtext) at (\XMTEXT, \YMTEXT);
      \coordinate (Ntext) at (\XNTEXT, \YNTEXT);
      % Domain and target
      \coordinate (domain) at (\XDOMAIN,\YDOMAIN);
      \coordinate (target) at (\XTARGET,\YTARGET);
      % M elements
      \coordinate (a) at (\XA, \YA);
      \coordinate (b) at (\XB, \YB);
      \coordinate (c) at (\XC, \YC);
      \coordinate (d) at (\XD, \YD);
      % N elements
      \coordinate (i) at (\XI, \YI);
      \coordinate (j) at (\XJ, \YJ);
      \coordinate (k) at (\XK, \YK);
      \coordinate (l) at (\XL, \YL);
      \coordinate (m) at (\XM, \YM);
      % Background coordinates
      \coordinate (backleft) at (\XBACKLEFT, \YCENTER);
      \coordinate (backright) at (\XBACKRIGHT, \YCENTER);
      \coordinate (backbottom) at (\XCENTER, \YBACKBOTTOM);
      \coordinate (backtop) at (\XCENTER, \YBACKTOP);
      % 
      % DRAWING
      % Sets M, N
      \node[M] {};
      \node[N]{};
      % Arrows
      \draw[arrow] (a) to[out=50, in=130] (i);
      \draw[arrow] (b) to[out=340, in=200] (i);
      \draw[arrow] (c) to[out=340, in=210] (k);
      \draw[arrow] (d) to[out=330, in=230] (l);
      % a, b, c, d
      \node[point, label=left:a] at (a) {};
      \node[point, label=left:b] at (b) {};
      \node[point, label=left:c] at (c) {};
      \node[point, label=left:d] at (d) {};
      % e, f, g, h, i
      \node[point, label=right:i] at (i) {};
      \node[point, label=right:j] at (j) {};
      \node[point, label=right:k] at (k) {};
      \node[point, label=right:l] at (l) {};
      \node[point, label=right:m] at (m) {};
      % M and N texts
      \node at (Mtext) {$M$};
      \node at (Ntext) {$N$};
      % Domain and N texts
      \node at (domain) {\small Domain};
      \node at (target) {\small Target};
      % 
      % YELLOW BACKGROUND
      \begin{scope}[on background layer]
        \node [background, fit= (backleft) (backright) (backbottom) (backtop)] {};
      \end{scope}
    \end{tikzpicture}
    \caption{An example of map $f:M\rightarrow N$.}
    \label{fig:topspac-map-example}
  \end{figure}

  \subsection{Concept of preimage}
  Given a certain map $f:M\longrightarrow N$, the \emph{preimage} of some
  subset of the target $V\subseteq N$ with respect to $f$ is the set of
  elements of the domain whose image is $V$
  \[
    \text{preim}_f\,(V) \coloneq \set{m\in M | f(m)\in V}
  \]

  Where
  \[
  \begin{array}{cccc}
    f:& M & \longrightarrow & N\\
      & m & \mapsto     & f(m)
  \end{array}
  \]
  
  In figure~\ref{fig:topspac-preimagefV} we can see the preimage with respect
  to $f$ of the set of the set $V\in N$, $V \coloneq \set{i}$, which is
  $\text{preim}_f\,(V) \coloneq \set{a, b}$ in our example.
  \begin{figure}[ht]
    \def\scl{1}
    % 
    \pgfmathsetmacro{\FACTOR}{1.0}
    % 
    % M set
    \pgfmathsetmacro{\XMSET}{0.0 * \FACTOR}
    \pgfmathsetmacro{\YMSET}{0.0 * \FACTOR}
    % M text
    \pgfmathsetmacro{\XMTEXT}{0.8 * \FACTOR}
    \pgfmathsetmacro{\YMTEXT}{4.6 * \FACTOR}
    % Domain text
    \pgfmathsetmacro{\XDOMAIN}{1.7 * \FACTOR}
    \pgfmathsetmacro{\YDOMAIN}{1.0 * \FACTOR}
    % Target text
    \pgfmathsetmacro{\XTARGET}{6.3 * \FACTOR}
    \pgfmathsetmacro{\YTARGET}{1.0 * \FACTOR}
    % a point
    \pgfmathsetmacro{\XA}{2.0 * \FACTOR}
    \pgfmathsetmacro{\YA}{4.0 * \FACTOR}
    % b point
    \pgfmathsetmacro{\XB}{1.7 * \FACTOR}
    \pgfmathsetmacro{\YB}{3.5 * \FACTOR}
    % c point
    \pgfmathsetmacro{\XC}{2.2 * \FACTOR}
    \pgfmathsetmacro{\YC}{2.8 * \FACTOR}
    % d point
    \pgfmathsetmacro{\XD}{1.5 * \FACTOR}
    \pgfmathsetmacro{\YD}{2.2 * \FACTOR}    
    % N set
    \pgfmathsetmacro{\XNSET}{4.0 * \FACTOR}
    \pgfmathsetmacro{\YNSET}{0.0 * \FACTOR}
    % N text
    \pgfmathsetmacro{\XNTEXT}{7.1 * \FACTOR}
    \pgfmathsetmacro{\YNTEXT}{4.6 * \FACTOR}    
    % i point
    \pgfmathsetmacro{\XI}{6.5 * \FACTOR}
    \pgfmathsetmacro{\YI}{4.0 * \FACTOR}
    % j point
    \pgfmathsetmacro{\XJ}{6.8 * \FACTOR}
    \pgfmathsetmacro{\YJ}{3.5 * \FACTOR}    
    % k point
    \pgfmathsetmacro{\XK}{5.9 * \FACTOR}
    \pgfmathsetmacro{\YK}{3.3 * \FACTOR}
    % l point
    \pgfmathsetmacro{\XL}{6.3 * \FACTOR}
    \pgfmathsetmacro{\YL}{3.0 * \FACTOR}
    % m point
    \pgfmathsetmacro{\XM}{6.1 * \FACTOR}
    \pgfmathsetmacro{\YM}{2.2 * \FACTOR}
    % Map f text
    \pgfmathsetmacro{\XF}{3.9 * \FACTOR}
    \pgfmathsetmacro{\YF}{3.8 * \FACTOR}    
    % Background
    \pgfmathsetmacro{\XCENTER}{0.5 * (\XMTEXT + \XNTEXT)}
    \pgfmathsetmacro{\YCENTER}{0.5 * (\YMTEXT + \YDOMAIN)}
    \pgfmathsetmacro{\XBACKLEFT}{\XMTEXT - 1.0 * \FACTOR}
    \pgfmathsetmacro{\YBACKBOTTOM}{\YDOMAIN - 0.5 * \FACTOR}
    \pgfmathsetmacro{\XBACKRIGHT}{\XNTEXT + 1.0 * \FACTOR}
    \pgfmathsetmacro{\YBACKTOP}{\YMTEXT + 0.8 * \FACTOR}
    %
    \centering
    %
    \begin{tikzpicture}[%
      scale=\scl,
      every label/.append style={text=black!80, font=\footnotesize
      }, M/.style={%
        ellipse, draw,
        fill=green!40, draw=black,
        very thick,
        minimum height=8em, minimum width=6em,
        fit=(a) (b) (c) (d),
      }, N/.style={%
        ellipse, draw,
        fill=green!40, draw=black,
        very thick,
        minimum height=8em, minimum width=6em,
        fit=(i) (j) (k) (l) (m),
      }, V/.style={%
        circle,
        draw,
        fill=green!85!black,
        draw=green!85!black,
        minimum size=2.4em,
        fit=(i),
      }, preimV/.style={%
        ellipse,
        draw,
        fill=green!85!black,
        draw=green!85!black,
        minimum height=0.8em,
        minimum width=3.7em,
        fit=(a) (b),
      }, point/.style={circle, fill, minimum width=2.5pt, inner sep=0pt
      }, arrow/.style={-{Latex}, shorten >=3pt, shorten <=4pt, line width=.8pt
      }, background/.style={
        line width=\bgborderwidth,
        draw=\bgbordercolor,
        fill=\bgcolor,
      },      
      ]
      % COORDINATES
      \coordinate (O) at (0, 0);
      % M and N sets
      \coordinate (M) at (\XMSET, \YMSET);
      \coordinate (N) at (\XNSET, \YNSET);
      % M and N texts
      \coordinate (Mtext) at (\XMTEXT, \YMTEXT);
      \coordinate (Ntext) at (\XNTEXT, \YNTEXT);
      % Domain and target
      \coordinate (domain) at (\XDOMAIN,\YDOMAIN);
      \coordinate (target) at (\XTARGET,\YTARGET);
      % M elements
      \coordinate (a) at (\XA, \YA);
      \coordinate (b) at (\XB, \YB);
      \coordinate (c) at (\XC, \YC);
      \coordinate (d) at (\XD, \YD);
      % N elements
      \coordinate (i) at (\XI, \YI);
      \coordinate (j) at (\XJ, \YJ);
      \coordinate (k) at (\XK, \YK);
      \coordinate (l) at (\XL, \YL);
      \coordinate (m) at (\XM, \YM);
      % Map f text
      \coordinate (mapf) at (\XF, \YF);
      % Background coordinates
      \coordinate (backleft) at (\XBACKLEFT, \YCENTER);
      \coordinate (backright) at (\XBACKRIGHT, \YCENTER);
      \coordinate (backbottom) at (\XCENTER, \YBACKBOTTOM);
      \coordinate (backtop) at (\XCENTER, \YBACKTOP);
      %
      % DRAWING
      % Sets M, N
      \node[M] {};
      \node[N] {};
      %
      \node[V, label=left:$V$] {};
      \node[preimV, rotate=45, label={[label distance=-2pt]310:$U$}] {};
      %
      % a, b, c, d
      \node[point, label=left:a] at (a) {};
      \node[point, label=left:b] at (b) {};
      \node[point, label=left:c] at (c) {};
      \node[point, label=left:d] at (d) {};
      % e, f, g, h, i
      \node[point, label=right:i] at (i) {};
      \node[point, label=right:j] at (j) {};
      \node[point, label=right:k] at (k) {};
      \node[point, label=right:l] at (l) {};
      \node[point, label=right:m] at (m) {};
      % Arrows
      \draw[arrow] (a) to[out=50, in=130] (i);
      \draw[arrow] (b) to[out=340, in=200] (i);
      \draw[arrow] (c) to[out=340, in=210] (k);
      \draw[arrow] (d) to[out=330, in=230] (l);
      % Map f text
      \node at (mapf) {$f$};
      % M and N texts
      \node at (Mtext) {$M$};
      \node at (Ntext) {$N$};
      % Domain and N texts
      \node at (domain) {\small Domain};
      \node at (target) {\small Target};
      % 
      % YELLOW BACKGROUND
      \begin{scope}[on background layer]
        \node [background, fit= (backleft) (backright) (backbottom) (backtop)]
        {};
      \end{scope}
    \end{tikzpicture}
    \caption{The preimage of $V\in N$ with respect of map $f:M\rightarrow N$ is
      the set marked as $U\in M$ in the domain.}
    \label{fig:topspac-preimagefV}
  \end{figure}

  You may say: The preimage is kind of an inverse $f^{-1}$!
  Well, careful, because if $f$ is not bijective, then $f^{-1}$ does not exist
  as a map. Look at figure~\ref{fig:topspac-preimagefV}. There, $f$ is not
  bijective, so the element $i$ would have to images, $a$ and $b$,
  $f^{-1}(i) = \set{a, b}$, but in a map an element cannot have two images.
  If this weren't enough, the element $m\in N$ does not have an image $f^{-1}$
  in $M$. Only, if $f$ is bijective we can write
  $\text{preim}_f\,(V) = f^{-1}(V)$.
  The point is that although $f^{-1}$ is not a map, there exists a preimage of
  every set of $N$ in $M$.
  
  \subsection{Continuous maps}
  A map is only defined by two sets. We need no further structure on a set to
  define what a map is.
  But we now ask ourselves if some arbitrary map, like the one in
  figure~\ref{fig:topspac-map-example} is continuous.
  The answer is\dots \emph{it depends}.

  The answer to the question of whether a map,
%  \[
    $f: M\longrightarrow N$,
 % \]
  is continuous, depends ---by definition--- on which topologies are chosen on
  the domain $M$ and on the target space $N$.

  So in order to decide continuity of a map, you need to decide a topology for
  the domain and another one for for the target. These topologies are different
  if the sets are different. If they are the same set, we could choose an equal
  or a different topology for both, the domain $M$ and the target $N$ sets.

  \subsubsection{Definition}
  Let $(M,\symcal{O}_M)$ and $(N,\symcal{O}_N)$ be two topological spaces.
  Then, a map $f:M\longrightarrow N$
  %\[
  %  f:M\longrightarrow N
  %\]
  is continuous ---with respect to the topologies
  $\symcal{O}_M$ and $\symcal{O}_N$--- if for every set open set $V$ in the
  target\footnotemark{} ($V\in\symcal{O}_N$) then,
  \footnotetext{We must check for every open set in the target.}
  the preimage with respect to $f$ of this chosen set is an open set in the
  domain
  \[
    \forall V\in\symcal{O}_N : \text{preim}_f\,(V) \in \symcal{O}_M
  \]
  and that's it, no epsilons, no deltas, it's very simple.

  If the map was bijective (\emph{one-to-one}), then it is invertible, and the
  preimage could have been written as the inverse of the map. And if the map
  was continuous in both directions, then not only the preimages of open sets
  are open, but also the images of open sets are open because the preimage of
  the inverse is the image of the map.

  In the map of figure~\ref{fig:topspac-map-example} map, the preimage of
  element $j$\/ is the empty set $\emptyset$, and the empty set is always an
  element of a topology.
  So if $j$\/ is an open set in $N$, then it doesn't destroy the notion of
  continuity of this map (of course, we must also check for every other sets in
  the topology).

  If maps were bijective and continuous in both directions, the topological
  structure is preserved and they have a specific name called homeomorphisms
  and they are structure preserving maps of topology. Mathematicians always
  clasify structures by identifying in them if certain structure preserving
  maps exist.V

  I would like to give a mnemonic phrase to remember this definition:
  \begin{quote}
    ``A map is continuous if and only if the preimages of all open sets (in the
    target space) are open sets (in the domain).''
  \end{quote}
 
  \subsubsection{First example}
  Let's say that set $M$ and $N$ are the same set ---why not?---
  \[
    M = \set{1, 2}
    \hspace{2em}
    N = \set{1,2}
  \]

  Now we invent a very simple map $f:M\longrightarrow N$. Just for fun, we map
  the $1$ to the $2$ and the $2$ to the $1$, as represented in
  figure~\ref{fig:topspac-simple-map-one}
  \begin{figure}[ht]
    \def\scl{1}
    %
    \pgfmathsetmacro{\FACTOR}{1.0}
    %
    % M set
    \pgfmathsetmacro{\XMSET}{0.0 * \FACTOR}
    \pgfmathsetmacro{\YMSET}{0.0 * \FACTOR}
    % M text
    \pgfmathsetmacro{\XMTEXT}{0.9 * \FACTOR}
    \pgfmathsetmacro{\YMTEXT}{3.6 * \FACTOR}
    % Domain text
    \pgfmathsetmacro{\XDOMAIN}{1.8 * \FACTOR}
    \pgfmathsetmacro{\YDOMAIN}{1.0 * \FACTOR}
    % Target text
    \pgfmathsetmacro{\XTARGET}{5.0 * \FACTOR}
    \pgfmathsetmacro{\YTARGET}{1.0 * \FACTOR}
    % a point
    \pgfmathsetmacro{\XA}{1.8 * \FACTOR}
    \pgfmathsetmacro{\YA}{3.0 * \FACTOR}
    % b point
    \pgfmathsetmacro{\XB}{1.8 * \FACTOR}
    \pgfmathsetmacro{\YB}{2.2 * \FACTOR}
    % M set
    \pgfmathsetmacro{\XNSET}{4.0 * \FACTOR}
    \pgfmathsetmacro{\YNSET}{0.0 * \FACTOR}
    % N text
    \pgfmathsetmacro{\XNTEXT}{5.8 * \FACTOR}
    \pgfmathsetmacro{\YNTEXT}{3.6 * \FACTOR}    
    % c point
    \pgfmathsetmacro{\XC}{5.0 * \FACTOR}
    \pgfmathsetmacro{\YC}{3.0 * \FACTOR}
    % d point
    \pgfmathsetmacro{\XD}{5.0 * \FACTOR}
    \pgfmathsetmacro{\YD}{2.2 * \FACTOR}
    % Map f text
    \pgfmathsetmacro{\XF}{3.5 * \FACTOR}
    \pgfmathsetmacro{\YF}{2.9 * \FACTOR}    
    % Background
    \pgfmathsetmacro{\XCENTER}{0.5 * (\XMTEXT + \XNTEXT)}
    \pgfmathsetmacro{\YCENTER}{0.5 * (\YMTEXT + \YDOMAIN)}
    \pgfmathsetmacro{\XBACKLEFT}{\XMTEXT - 0.7 * \FACTOR}
    \pgfmathsetmacro{\YBACKBOTTOM}{\YDOMAIN - 0.5 * \FACTOR}
    \pgfmathsetmacro{\XBACKRIGHT}{\XNTEXT + 0.7 * \FACTOR}
    \pgfmathsetmacro{\YBACKTOP}{\YMTEXT + 0.6 * \FACTOR}
    %
    \centering
    %
    \begin{tikzpicture}[%
      scale=\scl,
      every label/.append style={text=black!80, font=\footnotesize
      }, M/.style={%
        ellipse, draw,
        fill=green!40, draw=black,
        very thick,
        minimum height=5.5em, minimum width=4em,
        fit=(a) (b),
      }, N/.style={%
        ellipse, draw,
        fill=green!40, draw=black,
        very thick,
        minimum height=5.5em, minimum width=4em,
        fit=(c) (d),
      }, point/.style={circle, fill, minimum width=2.5pt, inner sep=0pt
      }, arrow/.style={-{Latex}, shorten >=3pt, shorten <=4pt, line width=.8pt
      }, background/.style={
        line width=\bgborderwidth,
        draw=\bgbordercolor,
        fill=\bgcolor,
      },     
      ]
      % COORDINATES
      \coordinate (O) at (0, 0);
      % M and N sets
      \coordinate (M) at (\XMSET, \YMSET);
      \coordinate (N) at (\XNSET, \YNSET);
      % M and N texts
      \coordinate (Mtext) at (\XMTEXT, \YMTEXT);
      \coordinate (Ntext) at (\XNTEXT, \YNTEXT);
      % Domain and target
      \coordinate (domain) at (\XDOMAIN,\YDOMAIN);
      \coordinate (target) at (\XTARGET,\YTARGET);
      % M elements
      \coordinate (a) at (\XA, \YA);
      \coordinate (b) at (\XB, \YB);
      % N elements
      \coordinate (c) at (\XC, \YC);
      \coordinate (d) at (\XD, \YD);
      % Map f text
      \coordinate (mapf) at (\XF, \YF);
      % Background coordinates
      \coordinate (backleft) at (\XBACKLEFT, \YCENTER);
      \coordinate (backright) at (\XBACKRIGHT, \YCENTER);
      \coordinate (backbottom) at (\XCENTER, \YBACKBOTTOM);
      \coordinate (backtop) at (\XCENTER, \YBACKTOP);
      %
      % DRAWING
      % Sets M, N
      \node[M] {};
      \node[N] {};
      %
      %\node[V, label=left:$V$] {};
      %\node[preimV, rotate=45, label={[label distance=-2pt]310:$U$}] {};
      %
      % a, b
      \node[point, label=left:1] at (a) {};
      \node[point, label=left:2] at (b) {};
      % c, d
      \node[point, label=right:1] at (c) {};
      \node[point, label=right:2] at (d) {};
      % Arrows
      \draw[arrow] (a) to[out=10, in=190] (d);
      \draw[arrow] (b) to[out=340, in=200] (c);
      % M and N texts
      \node at (Mtext) {$M$};
      \node at (Ntext) {$N$};
      % Domain and N texts
      \node at (domain) {\small Domain};
      \node at (target) {\small Target};
      % Map f text
      \node at (mapf) {\small $f$};
      % 
      % YELLOW BACKGROUND
      \begin{scope}[on background layer]
        \node [background, fit= (backleft) (backright) (backbottom) (backtop)]
        {};
      \end{scope}
    \end{tikzpicture}
    \caption{A very simple example map $f:M\longrightarrow N$ to discuss map
      continuity.}
    \label{fig:topspac-simple-map-one}
  \end{figure}

  We choose some topologies for each map\footnotemark{}
  \footnotetext{Note that, although the sets are the same, we are free choose a
    different topology for each one.}
  \begin{align*}
    &\text{Domain: }
      \symcal{O}_M \coloneq \Set{\emptyset, \set{1}, \set{2},
      \set{1, 2}}\\
    &\text{Target:}
      \hspace{1em} \symcal{O}_N \coloneq \Set{\emptyset, \set{1, 2}}
  \end{align*}
  
  Now we would like to check whether the map $f$ is continuous ---not all by
  itself, but with respect to the chosen topologies---.

  We must check that for every open subset of $N$, its preimage with respect to
  $f$ is an open set in $M$. It's easy in this example:
  \begin{itemize}
  \item The empty set is an open subset in the target
    $N$, $\emptyset\in\symcal{O}_N$. Its preimage is the empty set in the
    domain $M$, which is also an open subset
    \[
      \text{preim}_f\,(\emptyset) = \emptyset \in \symcal{O}_M 
    \]
    
  \item The whole $N=\set{1,2}$ is the other open set in the target $N$,
    $\set{1,2}\in\symcal{O}_N$,
    and its preimage is $M=\set{1,2}$, which is an open set in $M$.
    \[
      \text{preim}_f\,(\set{1,2}) = \set{1,2} \in \symcal{O}_M
    \]
  \end{itemize}
  Thus we conclude that $f$ is continuous.
  
  \subsubsection{Second example}
  The map in the first example was bijective, so we can talk about its inverse
  $f^{-1}$, and we keep the same topologies for the sets
    \begin{figure}[ht]
    \def\scl{1}
    % 
    \pgfmathsetmacro{\FACTOR}{1.0}
    % 
    % N set
    \pgfmathsetmacro{\XNSET}{0.0 * \FACTOR}
    \pgfmathsetmacro{\YNSET}{0.0 * \FACTOR}
    % N text
    \pgfmathsetmacro{\XNTEXT}{0.9 * \FACTOR}
    \pgfmathsetmacro{\YNTEXT}{3.6 * \FACTOR}
    % Domain text
    \pgfmathsetmacro{\XDOMAIN}{1.8 * \FACTOR}
    \pgfmathsetmacro{\YDOMAIN}{1.0 * \FACTOR}
    % Target text
    \pgfmathsetmacro{\XTARGET}{5.0 * \FACTOR}
    \pgfmathsetmacro{\YTARGET}{1.0 * \FACTOR}
    % a point
    \pgfmathsetmacro{\XA}{1.8 * \FACTOR}
    \pgfmathsetmacro{\YA}{3.0 * \FACTOR}
    % b point
    \pgfmathsetmacro{\XB}{1.8 * \FACTOR}
    \pgfmathsetmacro{\YB}{2.2 * \FACTOR}
    % M set
    \pgfmathsetmacro{\XMSET}{4.0 * \FACTOR}
    \pgfmathsetmacro{\YMSET}{0.0 * \FACTOR}
    % M text
    \pgfmathsetmacro{\XMTEXT}{5.8 * \FACTOR}
    \pgfmathsetmacro{\YMTEXT}{3.6 * \FACTOR}    
    % c point
    \pgfmathsetmacro{\XC}{5.0 * \FACTOR}
    \pgfmathsetmacro{\YC}{3.0 * \FACTOR}
    % d point
    \pgfmathsetmacro{\XD}{5.0 * \FACTOR}
    \pgfmathsetmacro{\YD}{2.2 * \FACTOR}
    % Map f-1 text
    \pgfmathsetmacro{\XF}{3.7 * \FACTOR}
    \pgfmathsetmacro{\YF}{2.9 * \FACTOR}    
    % Background
    \pgfmathsetmacro{\XCENTER}{0.5 * (\XNTEXT + \XMTEXT)}
    \pgfmathsetmacro{\YCENTER}{0.5 * (\YNTEXT + \YDOMAIN)}
    \pgfmathsetmacro{\XBACKLEFT}{\XNTEXT - 0.7 * \FACTOR}
    \pgfmathsetmacro{\YBACKBOTTOM}{\YDOMAIN - 0.5 * \FACTOR}
    \pgfmathsetmacro{\XBACKRIGHT}{\XMTEXT + 0.7 * \FACTOR}
    \pgfmathsetmacro{\YBACKTOP}{\YNTEXT + 0.6 * \FACTOR}
    %
    \centering
    %
    \begin{tikzpicture}[%
      scale=\scl,
      every label/.append style={text=black!80, font=\footnotesize
      }, M/.style={%
        ellipse, draw,
        fill=green!40, draw=black,
        very thick,
        minimum height=5.5em, minimum width=4em,
        fit=(a) (b),
      }, N/.style={%
        ellipse, draw,
        fill=green!40, draw=black,
        very thick,
        minimum height=5.5em, minimum width=4em,
        fit=(c) (d),
      }, point/.style={circle, fill, minimum width=2.5pt, inner sep=0pt
      }, arrow/.style={-{Latex}, shorten >=3pt, shorten <=4pt, line width=.8pt
      }, background/.style={
        line width=\bgborderwidth,
        draw=\bgbordercolor,
        fill=\bgcolor,
      },      
      ]
      % COORDINATES
      \coordinate (O) at (0, 0);
      % M and N sets
      \coordinate (M) at (\XMSET, \YMSET);
      \coordinate (N) at (\XNSET, \YNSET);
      % M and N texts
      \coordinate (Mtext) at (\XMTEXT, \YMTEXT);
      \coordinate (Ntext) at (\XNTEXT, \YNTEXT);
      % Domain and target
      \coordinate (domain) at (\XDOMAIN,\YDOMAIN);
      \coordinate (target) at (\XTARGET,\YTARGET);
      % M elements
      \coordinate (a) at (\XA, \YA);
      \coordinate (b) at (\XB, \YB);
      % N elements
      \coordinate (c) at (\XC, \YC);
      \coordinate (d) at (\XD, \YD);
      % Map f text
      \coordinate (mapf) at (\XF, \YF);
      % Background coordinates
      \coordinate (backleft) at (\XBACKLEFT, \YCENTER);
      \coordinate (backright) at (\XBACKRIGHT, \YCENTER);
      \coordinate (backbottom) at (\XCENTER, \YBACKBOTTOM);
      \coordinate (backtop) at (\XCENTER, \YBACKTOP);
      %
      % DRAWING
      % Sets M, N
      \node[M] {};
      \node[N] {};
      %
      %\node[V, label=left:$V$] {};
      %\node[preimV, rotate=45, label={[label distance=-2pt]310:$U$}] {};
      %
      % a, b
      \node[point, label=left:1] at (a) {};
      \node[point, label=left:2] at (b) {};
      % c, d
      \node[point, label=right:1] at (c) {};
      \node[point, label=right:2] at (d) {};
      % Arrows
      \draw[arrow] (a) to[out=10, in=190] (d);
      \draw[arrow] (b) to[out=340, in=200] (c);
      % M and N texts
      \node at (Mtext) {$M$};
      \node at (Ntext) {$N$};
      % Domain and N texts
      \node at (domain) {\small Domain};
      \node at (target) {\small Target};
      % Map f text
      \node at (mapf) {\small $f^{-1}$};
      % 
      % YELLOW BACKGROUND
      \begin{scope}[on background layer]
        \node [background, fit= (backleft) (backright) (backbottom) (backtop)]
        {};
      \end{scope}
    \end{tikzpicture}
    \caption{The inverse of the map in example one, $f^{-1}:N\longrightarrow B$
      to discuss map continuity.}
    \label{fig:topspac-simple-map-two}
  \end{figure}

  We keep the same topologies as before
  \begin{align*}
    \text{Domain: }
    \symcal{O}_N &\coloneq \Set{\emptyset, \set{1, 2}}\\
    \text{Target:}\hspace{1em}
    \symcal{O}_M &\coloneq \Set{\emptyset, \set{1}, \set{2}, \set{1, 2}}
  \end{align*}

  As always, we must check if the preimage of all open subsets in the target is
  an open set in the domain. We start with the open subset
  $\set{1}\in\symcal{O}_M$ in the target. Its preimage is the subset set
  $\set{2}$ in the domain. But this subset is not an open set in the domain,
  $\set{2}\notin\symcal{O}_N$. So we conclude that $f^{-1}$ is not
  continuous\footnotemark{}.
  \footnotetext{Although $f$ itself was continuous with the same topological
    spaces, $(M,\symcal{O}_M)$ and $(N,\symcal{O}_N)$.}
  \[
    \text{preim}_{f^{-1}}\,(\set{1}) = \set{2} \notin \symcal{O}_M
  \]

  In particular, if $1$ was mapped to $1$ and $2$ was mapped to $2$, then $f$
  would be the identity map as $f^{-1}$, but the identity $f$ would be
  continuous while the identity $f^{-1}$ would not\footnotemark{}.
  \footnotetext{%
    As you may guess, it all depends on the topologies associated to the sets.
    You see how dangerous it is to not explicitly talk about the topologies.}

  \begin{quote}
    ``If a map is continuous and invertible, its inverse might not be
    continuous. It all depends on the topologies chosen for the sets.''
  \end{quote}

  Although the next thing we are going to talk is not about continuity, let us
  remember something about open an closed sets. In particular we are going to
  recall that an open subset may be also closed. Look at subset $\set{2}$ in
  $M$. This subset is in
  the topology $\symcal{O}_M$, so it is an open subset
  \[
    \set{2}\in\symcal{O}_M
    \Longrightarrow \set{2}
    \text{is an open subset in } M
  \]
  
  But the entire set without the subset $\set{2}$ is also in the topology, so
  it is closed at the same time
  \[
    M \backslash\set{2} = \set{1}\in\symcal{O}_M
    \Longrightarrow
    \set{2} \text{is an closed subset in } M
  \]

  Also, remember that being open or closed has nothing to do with the image of
  the boundary being there or not. We can define the boundary for every
  topology, so these notions carry over, but our images of the boundary, as in
  figure~\ref{fig:topspac-softball-closedball} not necessarily do.

  So we have introduced the notion of continuity for general maps between any
  two topological spaces. In the beginning we drew this curve in
  figure~\ref{fig:topspac-nojumps} in a set that makes a jump all of a sudden
  in order to picture a non-discontinuity. Well the curve could go from the
  real line into some other set wich may be another real line, but we need to
  equip these two sets with a topology, which may be the same or different. So
  in order to start talking about continuity we must establish the two
  topological spaces first.

  In the rest of these lectures, every time we have a topological space
  involving the real numbers as $\symbb{R}, \symbb{R}^2 \cdots \symbb{R}^d$,
  and nothing else is said, \emph{we assume they carry the standard topology}:
  the standard topology on the real line, the standard topology on the real
  plane, \dots But in the other spaces we will have, in particular, spacetime
  will not be $\symbb{R}^3$, nor $\symbb{R}^4$, or something like that.
  Spacetime is a topological space all by itself.

  \subsection{Composition of continuous maps}
  Composition of maps is defined at the set level ---we don't need to define a
  topology on the sets in order to define map composition---.
  But now we are interested in the composition of continuous maps ---so we need
  to define a topology on the sets in order to talk about continuity---. We
  want to know whether the composition of two continuous maps is continuous or
  not.

  Let's have a map $f$ which goes from $M$ to $N$ and another map $g$ which goes
  from $N$ go $P$
  \[
    M
    \xrightarrow{\hspace{4pt}f\hspace{4pt}}
    N \xrightarrow{\hspace{5pt}g\hspace{5pt}}
    P
  \]

  We can define a new map $g\circ f$ ($g$ after $f$) which is a map from $M$ to
  $P$
  \[
    \begin{array}{rccl}
      g\circ f:& \hspace{-6pt}M & \longrightarrow & P \\
      &  m & \mapsto & (g\circ f)(m)
    \end{array}
  \]

  Where the composition is defined as $(g\circ f) (m) \coloneq g(f(m))$

  \subsubsection{Theorem and proof}
  \begin{quote}
    ``If $f$ and $g$ are continuous, then $g\circ f$ is also continuous.''
  \end{quote}

  Let $V$\/ be an open subset in $P$, that is, $V\in\symcal{O}_P$.
  The preimage of $V$\/ by $g\circ f$ is
  \begin{align*}
    \text{preim}_{g\circ f\/}\kern 2pt (V) \coloneq
    &\set{m\in M | (g\circ f)\in V}\\
    = &\set{m\in M | g(f(m))\in V}\\
    = &\set{m\in M | f(m)\in \text{preim}_g (V)}\\
    = &\text{preim}_f \kern 2pt(\text{preim}_g (V))
    %\in &\symcal{O}_M
  \end{align*}
  But $g$ is continuous, so the preimage of $V$ with respect to $g$ is an open
  subset
  \[
    \text{preim}_g(V)\in\symcal{O}_N
  \]
  and $f$ is also continuous which tells us that the preimage of an open subset ($\text{preim}_g(v)$)
  in $N$, with respec to $f$ is also an open subset in $M$
  \[
    \text{preim}_f \kern 2pt(\text{preim}_g(V))\in\symcal{O}_M
  \]
  So the theorem has been proved, since
  \[
    \text{preim}_{g\circ f} \kern 2pt(V) \in\symcal{O}_M
  \]

  % In words, as $g$ is continuous, the preimage of an open set $V$ in $P$ is an
  %open set in $N$.
  % Finally, as $f$ is continuous, it ensures us than the preimage of an open
  %set in $N$ is an open set in $M$. And that is what we wanted to prove.

  \subsection{Inheriting a topology}
  In this section we are interested in constructing a new topological space
  from other or others that we've already got.
  There are many useful ways to inherit a topology from some given topological
  space ---or spaces---.

  For spacetime physics it is important the following:
  Assume we have the set $M$, on which we are happy enough to already be given
  a topology $\symcal{O}_M$. So we already have a topological space.
  Now we consider a subset $S$ of this topological space
  $(M,\symcal{O}_M)$, where $S$ \emph{is any subset of} $M$, not necessarily an
  open subset
  %\vspace{-.5ex}
  \begin{center}
    \begin{tikzpicture}[%
      scale=1,
      curled/.style={%
        green!50!black,
        -{Stealth},
        decorate,
        decoration=
        {snake,post length=4.3pt,amplitude=1.3pt,segment length=3.5pt},
      },
      ]
      \coordinate (M) at (0,0);
      \coordinate (T) at +(30:1.3);
      \coordinate (A) at +(33:1.07);
      \coordinate (B) at (0.45,0.16);
    \node at (M)  {$S \subseteq M$};
    \node at (T) {\small $\symcal{O}_M$};
    \draw[curled] (A) -- (B);
  \end{tikzpicture}
 \end{center}
 Question: Can we make on $S$ a topology from the topology $\symcal{O}_M$ on
 $M$? Yes, we can always invent a topology on $S$, but maybe we want a special
 topology that in some way inherits from $\symcal{O}_M$ ---we'll see soon
 enough why this is interesting---.

 \subsubsection{Definition of subset topology}
 We define a topology $\symcal{O}|_S \subseteq \symcal{P} (S)$, called
 \emph{subset topology} (inherited from the topology on the superset $M$).
 Every element of $\symcal{O}|_S$ must be writeable as $U\cap S$, where $U$ is
 any set in the given topology $\symcal{O}_M$
 \[
   \symcal{O}|_S \coloneq \set{U\cap S | U\in\symcal{O}_M}
 \]
 
 We claim that $\symcal{O}|_S$ is a topology. To prove that, we need to check
 the axioms for a topology:
 \begin{enumerate}
 \item \emph{The empty set $\emptyset$ and the whole set $S$ must be part of
     $\symcal{O}|_S$\,:}
   \begin{itemize}
   \item The empty set is an element of the given topology
     $\emptyset\in\symcal{O}_M$, so the intersection of the empty set with $S$
     is the empty set, which is also an element of $\symcal{O}|_S$
   \[
     \emptyset = \emptyset\cap S \in \symcal{O}|_S
   \]

 \item The whole set $S$ needs to be part of $\symcal{O}|_s$. We know that $M$
   is an element of $\symcal{O}_M$, so this is also true
   \[
     S = M\cap S \in \symcal{O}|_S
   \]
   \end{itemize}

 \item \emph {The intersection of any two elements of the topology}
   $\symcal{O}|_S$ \emph{must be in the topology:}
   Let's take two elements of the topology
   \[
     A, B\in\symcal{O}|_S
     \Longrightarrow
     \exists \tilde{A}, \tilde{B}\in\symcal{O}_M
     \text{ such that }
     A = \tilde{A}\cap S, B = \tilde{B}\cap S
   \]
   So
   \[
     A\cap B = (\tilde{A}\cap S) \cap (\tilde{B}\cap S)
     = (\tilde{A}\cap\tilde{B})\cap S \in\symcal{O}|_S
   \]
   because
   \[
     \symcal{O}_M \text{ is a topology }
     \Longrightarrow
     \tilde{A}\cap\tilde{B}\in \symcal{O}_M
   \]

 \item \emph{Given a finite or infinite number of elements in} $\symcal{O}|_S$,
   \emph{their union is always in $\symcal{O}|_S$\,:}
   \begin{equation}
     \forall\, U_\alpha \in \symcal{O}|_S
     \Longrightarrow
     \bigcup_{\alpha\in A} U_\alpha \in \symcal{O}|_S
   \end{equation}

   Let's take a number ---finite or infinite--- of elements of
   $U_\alpha\in\symcal{O}|_S$.
   There must be a number of elements $\tilde{U}_\alpha\in\symcal{O}_M$ such
   that $U_\alpha = \tilde{U}_\alpha \cap S$.
   Now
   \[
     \bigcup_\alpha U_\alpha
     = \bigcup_\alpha \,(\tilde{U}_\alpha\cap S)
     = \left(\bigcup_\alpha \tilde{U}_\alpha\right) \cap S\in\symcal{O}|_S
   \]
   because $\bigcup_\alpha \tilde{U}_\alpha\in\symcal{O}_M$.
 \end{enumerate}

 \subsubsection{%
   Use of this specific way to inherit a topology from a superset
 }
 Very often, it is very easy for us to judge whether a certain map $f$ from $M$
 to $N$, with topologies $\symcal{O}_M$ and $\symcal{O}_N$, is continuous
% \vspace{-2ex}
% \begin{center}
%   %\begin{figure}[ht]
%   \def\scl{1}
%   \centering
%   \begin{tikzpicture}[%
%     scale=\scl,
%     curled/.style={%
%       green!50!black,
%       -{Stealth},
%       decorate,
%       decoration={%
%         snake,post length=4.3pt,amplitude=1.3pt,segment length=3.5pt
%       },
%     },
%     ]
%     \node (M) at (0,0) {$M$};
%     \node[right = -.4em of M]
%     (arrow) {$\xrightarrow[\phantom{.}]{\hspace{1.2em} f \hspace{1.2em}}$};
%     \node[right = -.4em of arrow] (N) {$N$};
%     \node[below = 1.5em of M] (OM) {\small $\symcal{O}_M$};
%     \node[below = 1.5em of N] (ON) {\small $\symcal{O}_N$};
%     \draw[curled] (OM) -- (M);
%     \draw[curled] (ON) -- (N);
%   \end{tikzpicture}
%   % \end{figure}
%   \end{center}
 \vspace{-2ex}
 \begin{center}
   % \begin{figure}[ht]
   \def\scl{1}
   \centering
   \begin{tikzpicture}[%
     scale=\scl,
     curled/.style={%
       green!50!black,
       -{Stealth},
       decorate,
       decoration={%
         snake,post length=4.3pt,amplitude=1.3pt,segment length=3.5pt
       },
     },
     ]
     \node (M) at (0,0) {$M$};
     \node[left = -0.4em of M] {$f\!: $};
     % \node[right = -.4em of M]
     % (arrow) {$\xrightarrow[\phantom{.}]{\hspace{1.2em} f \hspace{1.2em}}$};
     \node[right = -.4em of M]
     (arrow) {$\xrightarrow[\phantom{.}]{\hspace{1.2em}\hspace{1.2em}}$};
     \node[right = -.4em of arrow] (N) {$N$};
     \node[below = 1.5em of M] (OM) {\small $\symcal{O}_M$};
     \node[below = 1.5em of N] (ON) {\small $\symcal{O}_N$};
     \draw[curled] (OM) -- (M);
     \draw[curled] (ON) -- (N);
   \end{tikzpicture}
   % \end{figure}
   \end{center}

   We'd like to know what happens when we change $M$ with a certain subset of
   $M$, say ($S\subseteq M$) as the domain, keeping the same target space $N$
   and its topology $\symcal{O}_N$, This new map $f|_S$ is called
   \emph{restriction of $f$ only to the subset $S$}.
   Then, if we decide to equip $S$ with an arbitrary topology, we wouldn't know
   beforehand if it
   is continuous or not, because it depends on the chosen topology.
   But if we choose the subset topology $\symcal{O}|_S$ in the domain, then we
   are guaranteed that the restriction map $f|_S : S \longrightarrow N$ is
   continuous.
   \vspace{-1.0ex}
   \begin{center}
     % \begin{figure}[ht]
     \def\scl{1}
     \centering
     \begin{tikzpicture}[%
       scale=\scl,
       curled/.style={%
         green!50!black,
         -{Stealth},
         decorate,
         decoration={%
           snake,post length=4.3pt,amplitude=1.3pt,segment length=3.5pt
         },
       },
       ]
       \node (S) at (0,0) {$S$};
       \node[left = -0.4em of S] {$f|_S\!: $};       
       % \node[right = -.4em of S]
       % (arrow) {$\xrightarrow[\phantom{.}]{\hspace{1.2em} f|_S \hspace{1.2em}}$};
       \node[right = -.4em of S]
       (arrow) {$\xrightarrow[\phantom{.}]{\hspace{1.2em}\hspace{1.2em}}$};
       \node[right = -.4em of arrow] (N) {$N$};
       \node[below = 1.5em of S] (OS) {\small $\symcal{O}|_S$};
       \node[below = 1.5em of N] (ON) {\small $\symcal{O}_N$};
       \draw[curled] (OS) -- (S);
       \draw[curled] (ON) -- (N);
     \end{tikzpicture}
     % \end{figure}
   \end{center}
   and we'd like to do this very often.

   As a simple example, imagine we have a plane (hot plate). Every point of it
   has a certain temperature. This temperature field is a map from a subset of
   $\symbb{R}^2$ into $\symbb{R}$.
   \[
     f : \symbb{R}^2 \longrightarrow \symbb{R}
   \]

   We know that this map is continuous. But imagine that some mouse runs across
   the plate along some trajectory. The mouse is only interested in the
   temperature on the line (a subset $S$ of the points in the plane). Now we
   have a restriction of $f$ only to the subset $S$, $f|_S$.
   \[
     f|_S : S\subseteq\symbb{R}^2 \longrightarrow \symbb{R}
   \]

   If we keep the topology on the target, then the subset topology
   $\symcal{O}|_S$ guarantees that the new map is also continuous.
 

%janr - 1:17:00




%%% Local Variables:
%%% coding: utf-8
%%% mode: latex
%%% TeX-engine: luatex
%%% TeX-master: "../spacetime.tex"
%%% End:

